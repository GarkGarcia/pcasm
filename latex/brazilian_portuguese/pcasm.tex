% To create PDF version, type
%   pdflatex pcasm.tex
% This will produce errors the first time, type R at the error prompt
% Then rerun again (twice to get all the references.

\documentclass[11pt]{book}
\typeout{-----------------------------------------}
\typeout{Enter files to be included. (*=all)}
\typeout{(pcasm1,pcasm2,...)}
\typeout{-----------------------------------------}
\typein[\infiles]{ }
\if*\infiles\else\includeonly{\infiles}\fi


\newif\ifmypdf
\ifx\pdfoutput\undefined
    \pdffalse          % we are not running PDFLaTeX
\else
%    \pdfoutput=1       % we are running PDFLaTeX
%    \pdftrue
\fi
%----------------------------------------------------------------------
% The following is the construct that interests us in the end:
%\ifpdf
%   % Put PDF-specific stuff here
%\else
%   % Put LaTeX-specific stuff here
%\fi

\usepackage{indentfirst}  % indent first paragraph of sections
%\usepackage{graphicx}
\usepackage{listings}
\usepackage{epsfig}
\usepackage{longtable}
\usepackage{color}
\usepackage{makeidx}
\ifmypdf
\usepackage[pdftex,
            bookmarks=true,
            bookmarksnumbered=true,
            pdftitle={PC Assembly Language},
            pdfauthor={Paul A. Carter},
            pdfsubject={80x86 Assembly Language Programming},
	    pdfkeywords={80x86 assembly programming}]{hyperref}
\fi
\author{Paul~A.~Carter}
\title{PC Assembly Language}
\usepackage{lecnote}
\makeindex


\hyphenation{num-bers SF OF CF DF fact dif-fer-ence inter-rupts op-er-ands}
\begin{document}
\maketitle
\newlength{\AsmMargin}
\setlength{\AsmMargin}{-1cm}
\DefineVerbatimEnvironment{AsmCodeListing}{Verbatim}
{numbers=left, frame=lines,xleftmargin=\AsmMargin, labelposition=all, commentchar=^ }

\newcommand{\MarginNote}[1]{\marginpar{\sloppy \em \small #1}}
\thispagestyle{empty}
\vspace*{\fill}
\noindent This work is licensed under the Creative Commons 
Attribution-NonCommercial-ShareAlike 4.0 International License. To view
a copy of this license, visit
{\code http://creativecommons.org/licenses/by-nc-sa/4.0/}.


\index{subroutine|see{subprogram}}
\index{REPNZ|see{REPNE}}
\index{REPZ|see{REPE}}
\index{C++!member functions|see{methods}}
\index{text segment|see{code segment}}

\vfill
\frontmatter
\tableofcontents

% -*-LaTex-*-
%front matter of pcasm book

\chapter{Preface}

\section*{Purpose}

The purpose of this book is to give the reader a better understanding
of how computers really work at a lower level than in programming
languages like Pascal. By gaining a deeper understanding of how
computers work, the reader can often be much more productive
developing software in higher level languages such as C and
C++. Learning to program in assembly language is an excellent way to
achieve this goal. Other PC assembly language books still teach how to
program the 8086 processor that the original PC used in 1981!  The
8086 processor only supported \emph{real} mode. In this mode, any
program may address any memory or device in the computer. This mode is
not suitable for a secure, multitasking operating system.  This book
instead discusses how to program the 80386 and later processors in
\emph{protected} mode (the mode that Windows and Linux runs in).
This mode supports the features that modern operating systems expect,
such as virtual memory and memory protection.
There are several reasons to use protected mode:
\begin{enumerate}
\item It is easier to program in protected mode than in the 8086 real mode
      that other books use.
\item All modern PC operating systems run in protected mode.
\item There is free software available that runs in this mode.
\end{enumerate}
The lack of textbooks for protected mode PC assembly programming is the
main reason that the author wrote this book.

As alluded to above, this text makes use of Free/Open Source software: namely,
the NASM assembler and the DJGPP C/C++ compiler. Both of these are available
to download from the Internet. The text also discusses how to use NASM 
assembly code under the Linux operating system and with Borland's and
Microsoft's C/C++ compilers under Windows. Examples for all of these
platforms can be found on my web site: 
{\code http://pacman128.github.io/pcasm/}.
You \emph{must} download the example code if you wish to assemble
and run many of the examples in this tutorial.

Be aware that this text does not attempt to cover every aspect of assembly
programming. The author has tried to cover the most important topics that
\emph{all} programmers should be acquainted with.

\section*{Acknowledgements}

The author would like to thank the many programmers around the world
that have contributed to the Free/Open Source movement. All the
programs and even this book itself were produced using free
software. Specifically, the author would like to thank John~S.~Fine,
Simon~Tatham, Julian~Hall and others for developing the NASM assembler
that all the examples in this book are based on; DJ Delorie for
developing the DJGPP C/C++ compiler used; the numerous people who have
contributed to the GNU gcc compiler on which DJGPP is based on; Donald
Knuth and others for developing the \TeX\ and \LaTeXe\ typesetting
languages that were used to produce the book; Richard Stallman
(founder of the Free Software Foundation), Linus Torvalds (creator of
the Linux kernel) and others who produced the underlying software the
author used to produce this work.

Thanks to the following people for corrections:
\begin{itemize}
\item John S. Fine
\item Marcelo Henrique Pinto de Almeida
\item Sam Hopkins
\item Nick D'Imperio
\item Jeremiah Lawrence
\item Ed Beroset
\item Jerry Gembarowski
\item Ziqiang Peng
\item Eno Compton
\item Josh I Cates
\item Mik Mifflin
\item Luke Wallis
\item Gaku Ueda
\item Brian Heward
\item Chad Gorshing
\item F. Gotti
\item Bob Wilkinson
\item Markus Koegel
\item Louis Taber
\item Dave Kiddell
\item Eduardo Horowitz
\item S\'{e}bastien Le Ray
\item Nehal Mistry
\item Jianyue Wang
\item Jeremias Kleer
\item Marc Janicki
\item Trevor Hansen
\item Giacomo Bruschi
\item Leonardo Rodr\'{i}guez M\'{u}jica
\item Ulrich Bicheler
\item Wu Xing
\item Oleksandr Baranyuk
\end{itemize}


\section*{Resources on the Internet}
\begin{center}
\begin{tabular}{|ll|}
\hline
Author's page & {\code http://pacman128.github.io/} \\
NASM SourceForge page & {\code http://www.nasm.us/} \\
DJGPP  & {\code http://www.delorie.com/djgpp} \\
The Art of Assembly & {\code http://webster.cs.ucr.edu/} \\
USENET & {\code comp.lang.asm.x86 } \\
\hline
\end{tabular}
\end{center}


\section*{Feedback}

The author welcomes any feedback on this work.
\begin{center}
\begin{tabular}{ll}
\textbf{E-mail:} & {\code pacman128@gmail.com} \\
\textbf{WWW:}    & {\code http://pacman128.github.io/pcasm/} \\
\end{tabular}
\end{center}




\mainmatter
% -*-latex-*-
\chapter{Introdução}
\section{Sistemas de Numeração}

A memória em um computador consiste em números. A memória de um computador não 
armazena esses números em decimal (base 10). Por quê se simplifica bastante o 
hardware, computadores armazenam toda a informação no formato binário (base 2). 
Primeiramente, vamos revisar o sistema decimal.

\subsection{Decimal\index{decimal}}

Números em base 10 são compostos de 10 possíveis digitos (0-9). Cada dígito de 
um número é associado a uma potência de 10 de acordo com sua posição no número.
Por exemplo:
\begin{displaymath}
234 = 2 \times 10^2 + 3 \times 10^1 + 4 \times 10^0
\end{displaymath}

\subsection{Binário\index{binary|(}}

Números em base 2 são compostos de 2 possíveis dígitos (0 e 1). Cada dígito de 
um número é associado a uma potência de 2 de acordo com sua posição no número. 
(Um único dígito binário é dito um bit.) Por exemplo\footnote{O 2 subsescrito 
é usado para mostrar que um número está representado em binário e não decimal}:
\begin{eqnarray*}
11001_2 & = & 1 \times 2^4 + 1 \times 2^3 + 0 \times 2^2 + 0 \times 2^1 
              + 1 \times 2^0 \\
 & = & 16 + 8 + 1 \\
 & = & 25 
\end{eqnarray*}

Isso mostra como números binários podem ser convertidos para decimal. A 
Tabela~\ref{tab:dec-bin} mostra como são representados os primeiros números em 
\begin{table}[t]
\begin{center}
\begin{tabular}{||c|c||cc||c|c||}
\hline
Decimal & Binário & & & Decimal & Binário \\
\hline
0       & 0000   & & & 8       & 1000 \\
\hline
1       & 0001   & & & 9       & 1001 \\
\hline
2       & 0010   & & & 10      & 1010 \\
\hline
3       & 0011   & & & 11      & 1011 \\
\hline
4       & 0100   & & & 12      & 1100 \\
\hline
5       & 0101   & & & 13      & 1101 \\
\hline
6       & 0110   & & & 14      & 1110 \\
\hline
7       & 0111   & & & 15      & 1111 \\
\hline
\end{tabular}
\caption{Decimal de 0 a 15 em Binário\label{tab:dec-bin}}
\end{center}
\end{table}


\begin{figure}[h]
\begin{center}
\begin{tabular}{|rrrrrrrrp{.1cm}|p{.1cm}rrrrrrrr|}
\hline
& \multicolumn{7}{c}{Sem carry anterior} & & & \multicolumn{7}{c}{Com carry anterior} & \\
\hline
&  0 & &  0 & &  1 & &  1 & & &  0 & &  0 & &  1 & & 1  & \\
& +0 & & +1 & & +0 & & +1 & & & +0 & & +1 & & +0 & & +1 &  \\
\cline{2-2} \cline{4-4} \cline{6-6} \cline{8-8} \cline{11-11} \cline{13-13} \cline{15-15} \cline{17-17}
& 0  & & 1  & & 1  & & 0  & & & 1  & & 0  & & 0  & & 1 & \\
&    & &    & &    & & c  & & &    & & c  & & c  & & c & \\
\hline
\end{tabular}

\caption{Soma binária (c representa \emph{carry})\label{fig:bin-add}}
%TODO: Change this so that it is clear that single bits are being added,
%not 4-bit numbers. Ideas: Table or do sums horizontally.
\index{binary!addition}
\end{center}
\end{figure}

A Figura~\ref{fig:bin-add} mostra como dígitos binários individuais (isto é, 
bits) podem ser somados. Segue um exemplo:

\begin{tabular}{r}
 $11011_2$ \\
+$10001_2$ \\
\hline
$101100_2$ \\
\end{tabular}

Se considerarmos a seguinte divisão em decimal:
\[ 1234 \div 10 = 123\; r\; 4 \]
podemos assumir que essa divisão remove o dígito decimal mais à direita do 
número e desloca os demais dígitos decimais em uma posição para a direita. 
Dividindo por dois fazemos uma operação similar, mas com os dígitos binários do 
número. Considere a seguinte divisão binária:
\[ 1101_2 \div 10_2 = 110_2\; r\; 1 \]
Esse fato pode ser usado para converter-se um número decimal em sua 
representação binária equivalente como mostra a Figura~\ref{fig:dec-convert}. 
Esse método encontra o dígito mais à direita primeiro, esse dígito é chamado de 
\emph{dígito menos significativo} (lsb). O dígito mais à esquerda é chamado de 
\emph{dígito mais significativo} (msb). A unidade básica de memória consiste em 
8 bits e é chamada de \emph{byte}.
\index{binary|)}

\begin{figure}[t]
\centering
\fbox{\parbox{\textwidth}{
\begin{eqnarray*}
\mathrm{Decimal} & \mathrm{Binary} \\
25 \div 2 = 12\;r\;1 & 11001 \div 10 = 1100\;r\;1 \\
12 \div 2 = 6\;r\;0  & 1100 \div 10 = 110\;r\;0 \\
6 \div 2 = 3\;r\;0   & 110 \div 10 = 11\;r\;0 \\
3 \div 2 = 1\;r\;1   & 11 \div 10 = 1\;r\;1 \\
1 \div 2 = 0\;r\;1   & 1 \div 10 = 0\;r\;1 \\
\end{eqnarray*}

\centering
Thus $25_{10} = 11001_{2}$
}}
\caption{Decimal conversion \label{fig:dec-convert}}
\end{figure}

\subsection{Hexadecimal\index{hexadecimal|(}}

Hexadecimal numbers use base 16. Hexadecimal (or \emph{hex} for short) can be
used as a shorthand for binary numbers. Hex has 16 possible
digits. This creates a problem since there are no symbols to use for
these extra digits after 9. By convention, letters are used for these
extra digits. The 16 hex digits are 0-9 then A, B, C, D, E and F. The
digit A is equivalent to 10 in decimal, B is 11, etc. Each digit of a
hex number has a power of 16 associated with it. Example:
\begin{eqnarray*}
\rm
2BD_{16} & = & 2 \times 16^2 + 11 \times 16^1 + 13 \times 16^0 \\
         & = & 512 + 176 + 13 \\
         & = & 701 \\
\end{eqnarray*}
To convert from decimal to hex, use the same idea that was used for
binary conversion except divide by 16. See Figure~\ref{fig:hex-conv} for
an example.

\begin{figure}[t]
\centering
\fbox{\parbox{\textwidth}{
\begin{eqnarray*}
589 \div 16 & = & 36\;r\;13 \\
36 \div 16 & = & 2\;r\;4 \\
2 \div 16 & = & 0\;r\;2 \\
\end{eqnarray*}

\centering
Thus $589 = 24\mathrm{D}_{16}$
}}
\caption{\label{fig:hex-conv}}
\end{figure}

The reason that hex is useful is that there is a very simple way to
convert between hex and binary. Binary numbers get large and
cumbersome quickly. Hex provides a much more compact way to represent
binary.

To convert a hex number to binary, simply convert each hex digit to a
4-bit binary number. For example, $\mathrm{24D}_{16}$ is converted to
\mbox{$0010\;0100\; 1101_2$}. Note that the leading zeros of the
4-bits are important! If the leading zero for the middle digit of
$\mathrm{24D}_{16}$ is not used the result is wrong. Converting from
binary to hex is just as easy. One does the process in reverse. Convert
each 4-bit segments of the binary to hex. Start from the
right end, not the left end of the binary number. This ensures that
the process uses the correct 4-bit segments\footnote{If it is not
clear why the starting point makes a difference, try converting the
example starting at the left.}. Example:\newline

\begin{tabular}{cccccc}
$110$ & $0000$ & $0101$ & $1010$ & $0111$ & $1110_2$ \\
  $6$ & $0$    &   $5$  &   A  &  $7$   &    $\mathrm{E}_{16}$ \\
\end{tabular}\newline

A 4-bit number is called a \emph{nibble} \index{nibble}. Thus each hex
digit corresponds to a nibble. Two nibbles make a byte and so a byte
can be represented by a 2-digit hex number. A byte's value ranges from
0 to 11111111 in binary, 0 to FF in hex and 0 to 255 in decimal.
\index{hexadecimal|)}

\section{Computer Organization}

\subsection{Memory\index{memory|(}}

The basic unit of memory is a byte. \index{byte} \MarginNote{Memory is
measured in units of kilobytes (~$2^{10} = 1,024$ bytes), megabytes
(~$2^{20} = 1,048,576$ bytes) and gigabytes (~$2^{30} = 1,073,741,824$
bytes).}A computer with 32 megabytes of memory can hold roughly 32
million bytes of information. Each byte in memory is labeled by a
unique number known as its address as Figure~\ref{fig:memory} shows.

\begin{figure}[ht]
\begin{center}
\begin{tabular}{rcccccccc}
Address & 0 & 1 & 2 & 3 & 4 & 5 & 6 & 7 \\
\cline{2-9}
Memory & \multicolumn{1}{|c}{2A}  & \multicolumn{1}{|c}{45}  
       & \multicolumn{1}{|c}{B8} & \multicolumn{1}{|c}{20} 
       & \multicolumn{1}{|c}{8F} & \multicolumn{1}{|c}{CD} 
       & \multicolumn{1}{|c}{12} & \multicolumn{1}{|c|}{2E} \\
\cline{2-9}
\end{tabular}
\caption{ Memory Addresses \label{fig:memory} }
\end{center}
\end{figure}

\begin{table}
\begin{center}
\begin{tabular}{|l|l|}
\hline
word & 2 bytes \\ \hline
double word & 4 bytes \\ \hline
quad word & 8 bytes \\ \hline
paragraph & 16 bytes \\ \hline
\end{tabular}
\caption{ Units of Memory \label{tab:mem_units} }
\end{center}
\end{table}

Often memory is used in larger chunks than single bytes. On
the PC architecture, names have been given to these larger sections of
memory as Table~\ref{tab:mem_units} shows.

All data in memory is numeric. Characters are stored by using a
\emph{character code} that maps numbers to characters. One of the 
most common character codes is known as \emph{ASCII} (American
Standard Code for Information Interchange). A new, more complete code
that is supplanting ASCII is Unicode. One key difference between the
two codes is that ASCII uses one byte to encode a character, but
Unicode uses multiple bytes. There are several different forms of Unicode.
On x86 C/C++ compilers, Unicode is represented in code using the 
{\code wchar\_t} type and the UTF-16 encoding which uses 16 bits (or a 
\emph{word}) per character. For example, ASCII maps the byte $41_{16}$ 
($65_{10}$) to the character capital \emph{A}; UTF-16 maps it to the 
word $0041_{16}$. Since ASCII uses a byte, it is limited to only 256 
different characters\footnote{In fact, ASCII only uses the lower 7-bits 
and so only has 128 different values to use.}. Unicode extends the ASCII 
values and allows many more characters to be represented. This is important 
for representing characters for all the languages of the world.
\index{memory|)}

\subsection{The CPU\index{CPU|(}}

The Central Processing Unit (CPU) is the physical device that performs
instructions. The instructions that CPUs perform are generally very
simple. Instructions may require the data they act on to be in special
storage locations in the CPU itself called
\emph{registers}. \index{register} The CPU can access data in registers
much faster than data in memory. However, the number of registers in a
CPU is limited, so the programmer must take care to keep only
currently used data in registers.

The instructions a type of CPU executes make up the CPU's
\emph{machine language}. \index{machine language} Machine programs
have a much more basic structure than higher-level languages. Machine
language instructions are encoded as raw numbers, not in friendly text
formats. A CPU must be able to decode an instruction's purpose very
quickly to run efficiently. Machine language is designed with this
goal in mind, not to be easily deciphered by humans. Programs written
in other languages must be converted to the native machine language of
the CPU to run on the computer. A \emph{compiler} \index{compiler} is
a program that translates programs written in a programming language
into the machine language of a particular computer architecture. In
general, every type of CPU has its own unique machine language. This
is one reason why programs written for a Mac can not run on an
IBM-type PC.

Computers use a \emph{clock} \index{clock} to synchronize the
execution of the \MarginNote{\emph{GHz} stands for gigahertz or one
billion cycles per second.  A 1.5 GHz CPU has 1.5 billion clock pulses
per second.} instructions.  The clock pulses at a fixed frequency
(known as the \emph{clock speed}). When you buy a 1.5 GHz computer,
1.5 GHz is the frequency of this clock\footnote{Actually, clock pulses
are used in many different components of a computer. The other
components often use different clock speeds than the CPU.}. The clock
does not keep track of minutes and seconds. It simply beats at a
constant rate. The electronics of the CPU uses the beats to perform
their operations correctly, like how the beats of a metronome help one
play music at the correct rhythm.  The number of beats (or as they are
usually called \emph{cycles}) an instruction requires depends on the
CPU generation and model. The number of cycles depends on the
instructions before it and other factors as well.


\subsection{The 80x86 family of CPUs\index{CPU!80x86}}

IBM-type PC's contain a CPU from Intel's 80x86 family (or a clone of one).
The CPU's in this family all have some common features including a base
machine language. However, the more recent members greatly enhance the
features.
\begin{description}

\item[8088,8086:] These CPU's from the programming standpoint are
identical. They were the CPU's used in the earliest PC's. They provide
several 16-bit registers: AX, BX, CX, DX, SI, DI, BP, SP, CS, DS, SS,
ES, IP, FLAGS. They only support up to one megabyte of memory and only
operate in \emph{real mode}.  In this mode, a program may access any
memory address, even the memory of other programs! This makes
debugging and security very difficult! Also, program memory has to be
divided into \emph{segments}. Each segment can not be larger than
64K.

\item[80286:] This CPU was used in AT class PC's. It adds some new
instructions to the base machine language of the 8088/86.  However,
its main new feature is \emph{16-bit protected mode}.  In this mode,
it can access up to 16 megabytes and protect programs from accessing
each other's memory. However, programs are still divided into
segments that could not be bigger than 64K.

\item[80386:] This CPU greatly enhanced the 80286. First, it extends many
of the registers to hold 32-bits (EAX, EBX, ECX, EDX, ESI, EDI, EBP, ESP,
EIP) and adds two new 16-bit registers FS and GS. It also adds a new
\emph{32-bit protected mode}. In this mode, it can access up to 4 gigabytes.
Programs are again divided into segments, but now each segment can also be
up to 4 gigabytes in size!

\item[80486/Pentium/Pentium Pro:] These members of the 80x86 family add
very few new features. They mainly speed up the execution of the
instructions.

\item[Pentium MMX:] This processor adds the MMX (MultiMedia eXtensions)
instructions to the Pentium. These instructions can speed up common graphics
operations.

\item[Pentium II:] This is the Pentium Pro processor with the MMX instructions
added. (The Pentium III is essentially just a faster Pentium II.)

\end{description}
\index{CPU|)}

\subsection{8086 16-bit Registers\index{register|(}}

The original 8086 CPU provided four 16-bit general purpose registers:
AX, BX, CX and DX. Each of these registers could be decomposed into
two 8-bit registers. For example, the AX register could be decomposed
into the AH and AL registers as Figure~\ref{fig:AX_reg} shows. The AH
register contains the upper (or high) 8 bits of AX and AL contains the
lower 8 bits of AX. Often AH and AL are used as independent one byte
registers; however, it is important to realize that they are not
independent of AX. Changing AX's value will change AH and AL and
{\em vice versa}\/. The general purpose registers are used in many of
the data movement and arithmetic instructions.

\begin{figure}
\begin{center}
\begin{tabular}{cc}
\multicolumn{2}{c}{AX} \\
\hline
\multicolumn{1}{||c|}{AH} & \multicolumn{1}{c||}{AL} \\
\hline
\end{tabular}
\caption{The AX register \label{fig:AX_reg} }
\end{center}
\end{figure}

There are two 16-bit index registers\index{register!index}: SI and
DI. They are often used as pointers, but can be used for many of the
same purposes as the general registers. However, they can not be
decomposed into 8-bit registers.

The 16-bit BP and SP registers are used to point to data in the
machine language stack and are called the Base Pointer\index{register!base pointer} 
and Stack Pointer\index{register!stack pointer}, respectively. These will be discussed later. 

The 16-bit CS, DS, SS and ES registers are \emph{segment
registers}. \index{register!segment} They denote what memory is used
for different parts of a program. CS stands for Code Segment, DS for
Data Segment, SS for Stack Segment and ES for Extra Segment. ES is
used as a temporary segment register. The details of these registers
are in Sections~\ref{real_mode} and \ref{16prot_mode}.

The Instruction Pointer (IP) \index{register!IP} register is used with
the CS register to keep track of the address of the next instruction
to be executed by the CPU. Normally, as an instruction is executed, IP
is advanced to point to the next instruction in memory.

The FLAGS \index{register!FLAGS} register stores important information
about the results of a previous instruction. These results are stored
as individual bits in the register. For example, the Z bit is 1 if the
result of the previous instruction was zero or 0 if not zero. Not all
instructions modify the bits in FLAGS, consult the table in the
appendix to see how individual instructions affect the FLAGS register.

\subsection{80386 32-bit registers\index{register!32-bit}}

The 80386 and later processors have extended registers. For example, the
16-bit AX register is extended to be 32-bits. To be backward compatible, AX
still refers to the 16-bit register and EAX is used to refer to the extended
32-bit register. AX is the lower 16-bits of EAX just as AL is the lower 8-bits
of AX (and EAX). There is no way to access the upper 16-bits of EAX directly.
The other extended registers are EBX, ECX, EDX, ESI and EDI. 

Many of the other registers are extended as well. BP becomes
EBP\index{register!base pointer}; SP becomes ESP\index{register!stack
pointer}; FLAGS becomes EFLAGS\index{register!EFLAGS} and IP becomes
EIP\index{register!EIP}. However, unlike the index and general purpose
registers, in 32-bit protected mode (discussed below) only the
extended versions of these registers are used.

The segment registers are still 16-bit in the 80386. There are also
two new segment registers: FS and GS\index{register!segment}. Their
names do not stand for anything. They are extra temporary segment
registers (like ES).

One of definitions of the term \emph{word} \index{word} refers to the
size of the data registers of the CPU. For the 80x86 family, the term
is now a little confusing. In Table~\ref{tab:mem_units}, one sees that
\emph{word} is defined to be 2 bytes (or 16 bits). It was given this
meaning when the 8086 was first released. When the 80386 was
developed, it was decided to leave the definition of \emph{word}
unchanged, even though the register size changed.
\index{register|)}

\subsection{Real Mode \label{real_mode} \index{real mode|(}}

In \MarginNote{So where did the infamous DOS 640K limit come from? The BIOS 
required some of the 1M for its code and for hardware devices like the video 
screen.} real mode, memory is limited to only one megabyte ($2^{20}$ bytes). 
Valid
address range from (in hex) 00000 to FFFFF.\@  % \@ means end of sentence
These addresses require a 20-bit number. Obviously, a 20-bit number will not
fit into any of the 8086's 16-bit registers. Intel solved this problem, by
using two 16-bit values to determine an address. The first 16-bit value is called
the \emph{selector}. Selector values must be stored in segment registers. The
second 16-bit value is called the \emph{offset}. The physical address 
referenced by a 32-bit \emph{selector:offset} pair is computed by the formula
\[ 16 * {\rm selector} + {\rm offset} \]
Multiplying by 16 in hex is easy, just add a 0 to the right of the number. For
example, the physical addresses referenced by 047C:0048 is given by:
\begin{center}
\begin{tabular}{r}
047C0 \\
+0048 \\
\hline
04808 \\
\end{tabular}
\end{center}
In effect, the selector value is a paragraph number
(see Table~\ref{tab:mem_units}).

Real segmented addresses have disadvantages:
\begin{itemize}
\item A single selector value can only reference 64K of memory (the
upper limit of the 16-bit offset). What if a program has more than 64K
of code? A single value in CS can not be used for the entire execution
of the program.  The program must be split up into sections (called
\emph{segments}\index{memory!segments}) less than 64K in size. When
execution moves from one segment to another, the value of CS must be
changed. Similar problems occur with large amounts of data and the DS
register. This can be very awkward!

\item Each byte in memory does not have a unique segmented address. The 
physical address 04808 can be referenced by 047C:0048, 047D:0038, 047E:0028
or 047B:0058.\@ This can complicate the comparison of segmented addresses.

\end{itemize}
\index{real mode|)}

\subsection{16-bit Protected Mode \label{16prot_mode} \index{protected mode!16-bit|(}}

In the 80286's 16-bit protected mode, selector values are interpreted
completely differently than in real mode. In real mode, a selector
value is a paragraph number of physical memory. In protected mode, a
selector value is an \emph{index} into a \emph{descriptor table}. In
both modes, programs are divided into
segments\index{memory:segments}. In real mode, these segments are at
fixed positions in physical memory and the selector value denotes the
paragraph number of the beginning of the segment. In protected mode,
the segments are not at fixed positions in physical memory. In fact,
they do not have to be in memory at all!

Protected mode uses a technique called \emph{virtual memory}
\index{memory!virtual}. The basic idea of a virtual memory system is
to only keep the data and code in memory that programs are currently
using. Other data and code are stored temporarily on disk until they
are needed again.  In 16-bit protected mode, segments are moved
between memory and disk as needed. When a segment is returned to
memory from disk, it is very likely that it will be put into a
different area of memory that it was in before being moved to
disk. All of this is done transparently by the operating system. The
program does not have to be written differently for virtual memory to
work.

In protected mode, each segment is assigned an entry in a descriptor table.
This entry has all the information that the system needs to know about the
segment. This information includes: is it currently in memory; if in memory,
where is it; access permissions ({\em e.g.\/}, read-only). The index of the
entry of the segment is the selector value that is stored in segment registers.

One \MarginNote{One well-known PC columnist called the 286 CPU ``brain
dead.''} big disadvantage of 16-bit protected mode is that offsets
are still 16-bit quantities. As a consequence of this, segment sizes
are still limited to at most 64K. This makes the use of large arrays
problematic!
\index{protected mode!16-bit|)}

\subsection{32-bit Protected Mode\index{protected mode!32-bit|(}}

The 80386 introduced 32-bit protected mode. There are two major differences
between 386 32-bit and 286 16-bit protected modes:
\begin{enumerate}
\item

Offsets are expanded to be 32-bits. This allows an offset to range up
to 4 billion. Thus, segments can have sizes up to 4 gigabytes.

\item

Segments\index{memory!segments} can be divided into smaller 4K-sized
units called \emph{pages}\index{memory!pages}. The virtual
memory\index{memory!virtual} system works with pages now instead of
segments. This means that only parts of segment may be in memory at
any one time. In 286 16-bit mode, either the entire segment is in
memory or none of it is. This is not practical with the larger
segments that 32-bit mode allows.

\end{enumerate}

\index{protected mode!32-bit|)}

In Windows 3.x, \emph{standard mode} referred to 286 16-bit protected mode and
\emph{enhanced mode} referred to 32-bit mode. Windows 9X, Windows NT/2000/XP, OS/2
and Linux all run in paged 32-bit protected mode.

\subsection{Interrupts\index{interrupt}}

Sometimes the ordinary flow of a program must be interrupted to process events
that require prompt response. The hardware of a computer provides a mechanism
called \emph{interrupts} to handle these events. For example, when a mouse is
moved, the mouse hardware interrupts the current program to handle the mouse
movement (to move the mouse cursor, {\em etc.\/}) Interrupts cause control to
be passed to an \emph{interrupt handler}. Interrupt handlers are routines that
process the interrupt. Each type of interrupt is assigned an integer number.
At the beginning of physical memory, a table of \emph{interrupt vectors}
resides that contain the segmented addresses of the interrupt handlers. The
number of interrupt is essentially an index into this table.

External interrupts are raised from outside the CPU. (The mouse is an
example of this type.) Many I/O devices raise interrupts ({\em
e.g.\/}, keyboard, timer, disk drives, CD-ROM and sound
cards). Internal interrupts are raised from within the CPU, either
from an error or the interrupt instruction. Error interrupts are also
called \emph{traps}. Interrupts generated from the interrupt
instruction are called \emph{software interrupts}. DOS uses these types of
interrupts to implement its API (Application Programming Interface). More
modern operating systems (such as Windows and UNIX) use a C based interface.
\footnote{However, they may use a lower level interface at the kernel level.}

Many interrupt handlers return control back to the interrupted program
when they finish. They restore all the registers to the same values
they had before the interrupt occurred. Thus, the interrupted program
runs as if nothing happened (except that it lost some CPU
cycles). Traps generally do not return. Often they abort the program.

\section{Assembly Language}

\subsection{Machine language\index{machine language}}

Every type of CPU understands its own machine language. Instructions
in machine language are numbers stored as bytes in memory. Each
instruction has its own unique numeric code called its \emph{operation
code} or \emph{opcode} \index{opcode} for short. The 80x86 processor's
instructions vary in size.  The opcode is always at the beginning of
the instruction. Many instructions also include data ({\em e.g.\/},
constants or addresses) used by the instruction.

Machine language is very difficult to program in directly. Deciphering the
meanings of the numerical-coded instructions is tedious for humans. For
example, the instruction that says to add the EAX and EBX registers together
and store the result back into EAX is encoded by the following hex codes:
\begin{quote}
   03 C3
\end{quote}
This is hardly obvious. Fortunately, a program called an
\emph{assembler} \index{assembler} can do this tedious work for the
programmer.

\subsection{Assembly language\index{assembly language|(}}

An assembly language program is stored as text (just as a higher level language
program). Each assembly instruction represents exactly one machine instruction.
For example, the addition instruction described above would be represented
in assembly language as:
\begin{CodeQuote}
   add eax, ebx
\end{CodeQuote}
Here the meaning of the instruction is \emph{much} clearer than in
machine code. The word {\code add} is a \emph{mnemonic}
\index{mnemonic} for the addition instruction.  The general form of an
assembly instruction is:
\begin{CodeQuote}
  {\em mnemonic operand(s)}
\end{CodeQuote}

An \emph{assembler} \index{assembler} is a program that reads a text
file with assembly instructions and converts the assembly into machine
code.  \emph{Compilers} \index{compiler} are programs that do similar
conversions for high-level programming languages. An assembler is much
simpler than a compiler. \MarginNote{It took several years for
computer scientists to figure out how to even write a compiler!} Every
assembly language statement directly represents a single machine
instruction. High-level language statements are \emph{much} more
complex and may require many machine instructions.

Another important difference between assembly and high-level languages is that
since every different type of CPU has its own machine language, it also has
its own assembly language. Porting assembly programs between different computer
architectures is \emph{much} more difficult than in a high-level language.

This book's examples uses the Netwide Assembler or NASM \index{NASM}
for short. It is freely available off the Internet (see the preface
for the URL). More common assemblers are Microsoft's Assembler (MASM)
\index{MASM} or Borland's Assembler (TASM). \index{TASM} There are
some differences in the assembly syntax for MASM/TASM and NASM.

\subsection{Instruction operands}

Machine code instructions have varying number and type of operands; however,
in general, each instruction itself will have a fixed number of operands (0
to 3). Operands can have the following types:
\begin{description}
\item[register:]
These operands refer directly to the contents of the CPU's registers.
\item[memory:]
These refer to data in memory. The address of the data may be a constant
hardcoded into the instruction or may be computed using values of registers.
Address are always offsets from the beginning of a segment.
\item[immediate:]
\index{immediate}
These are fixed values that are listed in the instruction itself. They are
stored in the instruction itself (in the code segment), not in the data
segment.
\item[implied:]
These operands are not explicitly shown. For example, the increment 
instruction adds one to a register or memory. The one is implied.
\end{description}
\index{assembly language|)}

\subsection{Basic instructions}

The most basic instruction is the {\code MOV} \index{MOV} instruction. It moves
data from one location to another (like the assignment operator in a
high-level language). It takes two operands:
\begin{CodeQuote}
  mov {\em dest, src}
\end{CodeQuote}
The data specified by {\em src} is copied to {\em dest\/}. One restriction
is that both operands may not be memory operands. This points out another
quirk of assembly. There are often somewhat arbitrary rules about how the
various instructions are used. The operands must also be the same size. The
value of AX can not be stored into BL.

Here is an example (semicolons start a comment\index{comment}):
\begin{AsmCodeListing}[frame=none, numbers=none]
      mov    eax, 3   ; store 3 into EAX register (3 is immediate operand) 
      mov    bx, ax   ; store the value of AX into the BX register 
\end{AsmCodeListing}

The {\code ADD} \index{ADD} instruction is used to add integers.
\begin{AsmCodeListing}[frame=none, numbers=none]
      add    eax, 4   ; eax = eax + 4
      add    al, ah   ; al = al + ah 
\end{AsmCodeListing}

The {\code SUB} \index{SUB} instruction subtracts integers.
\begin{AsmCodeListing}[frame=none, numbers=none]
      sub    bx, 10   ; bx = bx - 10
      sub    ebx, edi ; ebx = ebx - edi
\end{AsmCodeListing}

The {\code INC} \index{INC} and {\code DEC} \index{DEC} instructions
increment or decrement values by one. Since the one is an implicit
operand, the machine code for {\code INC} and {\code DEC} is smaller
than for the equivalent {\code ADD} and {\code SUB} instructions.
\begin{AsmCodeListing}[frame=none, numbers=none]
      inc    ecx      ; ecx++
      dec    dl       ; dl--
\end{AsmCodeListing}

\subsection{Directives\index{directive|(}}

A \emph{directive} is an artifact of the assembler not the CPU. They are
generally used to either instruct the assembler to do something or inform
the assembler of something. They are not translated into machine code. Common
uses of directives are:
\begin{list}{$\bullet$}{\setlength{\itemsep}{0pt}}
\item define constants
\item define memory to store data into
\item group memory into segments
\item conditionally include source code
\item include other files
\end{list}

NASM code passes through a preprocessor just like C. It has many of the
same preprocessor commands as C. However, NASM's preprocessor directives 
start with a \% instead of a \# as in C.

\subsubsection{The equ directive\index{directive!equ}}

The {\code equ} directive can be used to define a \emph{symbol}. Symbols are
named constants that can be used in the assembly program. The format is:
\begin{quote}
  \code {\em symbol} equ {\em value}
\end{quote}
Symbol values can \emph{not} be redefined later.

\subsubsection{The \%define directive\index{directive!\%define}}

This directive is similar to C's {\code \#define} directive. It is
most commonly used to define constant macros just as in C.
\begin{AsmCodeListing}[frame=none, numbers=none]
%define SIZE 100
      mov    eax, SIZE
\end{AsmCodeListing}
The above code defines a macro named {\code SIZE} and shows its use in
a {\code MOV} instruction. Macros are more flexible than symbols in
two ways. Macros can be redefined and can be more than simple constant
numbers.

\subsubsection{Data directives\index{directive!data|(}}

\begin{table}[t]
\centering
\begin{tabular}{||c|c||} \hline
{\bf Unit} & {\bf Letter} \\
\hline
byte & B \\
word & W \\
double word & D \\
quad word & Q \\
ten bytes & T \\
\hline
\end{tabular}
\caption{Letters for {\code RESX} and {\code DX} Directives 
         \label{tab:size-letters} }
\end{table}

Data directives are used in data segments to define room for
memory. There are two ways memory can be reserved. The first way only
defines room for data; the second way defines room and an initial
value. The first method uses one of the {\code RES{\em
X}}\index{directive!RES\emph{X}} directives. The {\em X} is replaced
with a letter that determines the size of the object (or objects) that
will be stored. Table~\ref{tab:size-letters} shows the possible
values.

The second method (that defines an initial value, too) uses one of the
{\code D{\em X}} directives\index{directive!D\emph{X}}. The {\em X}
letters are the same as those in the {\code RES{\em X}} directives.

It is very common to mark memory locations with
\emph{labels}. \index{label} Labels allow one to easily refer to
memory locations in code. Below are several examples:
\begin{AsmCodeListing}[frame=none, numbers=none]
L1    db     0        ; byte labeled L1 with initial value 0
L2    dw     1000     ; word labeled L2 with initial value 1000
L3    db     110101b  ; byte initialized to binary 110101 (53 in decimal)
L4    db     12h      ; byte initialized to hex 12 (18 in decimal)
L5    db     17o      ; byte initialized to octal 17 (15 in decimal)
L6    dd     1A92h    ; double word initialized to hex 1A92
L7    resb   1        ; 1 uninitialized byte
L8    db     "A"      ; byte initialized to ASCII code for A (65)
\end{AsmCodeListing}

Double quotes and single quotes are treated the same. Consecutive data
definitions are stored sequentially in memory. That is, the word L2 is
stored immediately after L1 in memory. Sequences of memory may also be
defined.
\begin{AsmCodeListing}[frame=none, numbers=none]
L9    db     0, 1, 2, 3              ; defines 4 bytes
L10   db     "w", "o", "r", 'd', 0   ; defines a C string = "word"
L11   db     'word', 0               ; same as L10
\end{AsmCodeListing}

The {\code DD}\index{directive!DD} directive can be used to define
both integer and single precision floating point\footnote{Single
precision floating point is equivalent to a {\code float} variable in
C.} constants. However, the {\code DQ}\index{directive!DQ} can only
be used to define double precision floating point constants.

For large sequences, NASM's {\code TIMES} \index{directive!TIMES}
directive is often useful. This directive repeats its operand a
specified number of times. For example,
\begin{AsmCodeListing}[frame=none, numbers=none]
L12   times 100 db 0                 ; equivalent to 100 (db 0)'s
L13   resw   100                     ; reserves room for 100 words
\end{AsmCodeListing}
\index{directive!data|)}
\index{directive|)}

\index{label|(}
Remember that labels  can be used to refer to data in code. There are
two ways that a label can be used. If a plain label is used, it is
interpreted as the address (or offset) of the data. If the label is
placed inside square brackets ({\code []}), it is interpreted as the data at
the address. In other words, one should think of a label as a \emph{pointer}
to the data and the square brackets dereferences the pointer just as
the asterisk does in C. (MASM/TASM follow a different convention.) 
In 32-bit mode, addresses are 32-bit. Here are some examples:
\begin{AsmCodeListing}[frame=none]
      mov    al, [L1]      ; copy byte at L1 into AL
      mov    eax, L1       ; EAX = address of byte at L1
      mov    [L1], ah      ; copy AH into byte at L1
      mov    eax, [L6]     ; copy double word at L6 into EAX
      add    eax, [L6]     ; EAX = EAX + double word at L6
      add    [L6], eax     ; double word at L6 += EAX
      mov    al, [L6]      ; copy first byte of double word at L6 into AL
\end{AsmCodeListing}
Line 7 of the examples shows an important property of NASM. The assembler does
\emph{not} keep track of the type of data that a label refers to. It is up
to the programmer to make sure that he (or she) uses a label correctly. Later
it will be common to store addresses of data in registers and use the register
like a pointer variable in C. Again, no checking is made that a pointer is
used correctly. In this way, assembly is much more error prone than even C.

Consider the following instruction:
\begin{AsmCodeListing}[frame=none, numbers=none]
      mov    [L6], 1             ; store a 1 at L6
\end{AsmCodeListing}
This statement produces an {\code operation size not specified} error. Why?
Because the assembler does not know whether to store the 1 as a byte, word
or double word. To fix this, add a size specifier:
\begin{AsmCodeListing}[frame=none, numbers=none]
      mov    dword [L6], 1       ; store a 1 at L6
\end{AsmCodeListing}
\index{DWORD}This tells the assembler to store an 1 at the double word that starts at
{\code L6}. Other size specifiers are: {\code BYTE}\index{BYTE}, {\code WORD}\index{WORD},
{\code QWORD}\index{QWORD} and {\code TWORD}\footnote{{\code TWORD} defines a ten byte
area of memory. The floating point coprocessor uses this data type.}\index{TWORD}.
\index{label|)}

\subsection{Input and Output \index{I/O|(}}

Input and output are very system dependent activities. It involves
interfacing with the system's hardware. High level languages, like C,
provide standard libraries of routines that provide a simple, uniform
programming interface for I/O.  Assembly languages provide no standard
libraries. They must either directly access hardware (which is a privileged
operation in protected mode) or use whatever low level routines that the
operating system provides.

\index{I/O!asm\_io library|(} 
It is very common for assembly routines to be interfaced with C. One
advantage of this is that the assembly code can use the standard C
library I/O routines.  However, one must know the rules for passing
information between routines that C uses. These rules are too
complicated to cover here. (They are covered later!) To simplify I/O,
the author has developed his own routines that hide the complex C
rules and provide a much more simple interface.  Table~\ref{tab:asmio}
describes the routines provided. All of the routines preserve the
value of all registers, except for the read routines. These routines
do modify the value of the EAX register. To use these routines, one
must include a file with information that the assembler needs to use
them.  To include a file in NASM, use the {\code \%include}
preprocessor directive. The following line includes the file needed by
the author's I/O routines\footnote{The {\code asm\_io.inc} (and the
{\code asm\_io} object file that {\code asm\_io.inc} requires) are in
the example code downloads on the web page for this tutorial, {\code
http://pacman128.github.io/pcasm/}}:
\begin{AsmCodeListing}[frame=none, numbers=none]
%include "asm_io.inc"
\end{AsmCodeListing}

\begin{table}[t]
\centering
\begin{tabular}{lp{3.5in}}
{\bf print\_int} & prints out to the screen the value of the integer stored 
                  in EAX \\
{\bf print\_char} & prints out to the screen the character whose
                    ASCII value stored in AL \\
{\bf print\_string} & prints out to the screen the contents of the string
                     at the {\em address} stored in EAX. The string must be
                     a C-type string ({\em i.e.} null terminated). \\
{\bf print\_nl} & prints out to the screen a new line character. \\
{\bf read\_int} & reads an integer from the keyboard and stores it into the
                 EAX register. \\
{\bf read\_char} & reads a single character from the keyboard and stores its
                  ASCII code into the EAX register. \\
\end{tabular}
\caption{Assembly I/O Routines \label{tab:asmio} \index{I/O!asm\_io library!print\_int}
\index{I/O!asm\_io library!print\_char} \index{I/O!asm\_io library!print\_string} 
\index{I/O!asm\_io library!print\_nl} \index{I/O!asm\_io library!read\_int}
\index{I/O!asm\_io library!read\_char}}
\end{table}

To use one of the print routines, one loads EAX with the correct value and
uses a {\code CALL} instruction to invoke it. The {\code CALL} instruction
is equivalent to a function call in a high level language. It jumps execution
to another section of code, but returns back to its origin after the routine
is over. The example program below shows several examples of calls to these
I/O routines.

\subsection{Debugging\index{debugging|(}}

The author's library also contains some useful routines for debugging 
programs. These debugging routines display information about the state of
the computer without modifying the state. These routines are really
\emph{macros} that preserve the current state of the CPU and then make a
subroutine call. The macros are defined in the {\code asm\_io.inc} file
discussed above. Macros are used like ordinary instructions. Operands of
macros are separated by commas.

There are four debugging routines named {\code dump\_regs}, {\code
dump\_mem}, {\code dump\_stack} and {\code dump\_math}; they display
the values of registers, memory, stack and the math coprocessor,
respectively.
\begin{description}

\item[dump\_regs]
\index{I/O!asm\_io library!dump\_regs} 
This macro prints out the values of the registers (in hexadecimal) of
the computer to {\code stdout} (\emph{i.e.} the screen). It also
displays the bits set in the FLAGS\footnote{Chapter~2 discusses this
register} register. For example, if the zero flag is 1, \emph{ZF} is
displayed. If it is 0, it is not displayed. It takes a single integer
argument that is printed out as well. This can be used to distinguish
the output of different {\code dump\_regs} commands.

\item[dump\_mem]
\index{I/O!asm\_io library!dump\_mem} 
This macro prints out the values of a region of memory (in
hexadecimal) and also as ASCII characters. It takes three comma delimited
arguments. The first is an integer that is used to label the
output (just as {\code dump\_regs} argument). The second argument is
the address to display. (This can be a label.) The last argument is
the number of 16-byte paragraphs to display after the address. The
memory displayed will start on the first paragraph boundary before the
requested address.

\item[dump\_stack]
\index{I/O!asm\_io library!dump\_stack} 
This macro prints out the values on the CPU stack. (The stack will be
covered in Chapter~4.) The stack is organized as double words and this
routine displays them this way. It takes three comma delimited
arguments. The first is an integer label (like {\code
dump\_regs}). The second is the number of double words to display
\emph{below} the address that the {\code EBP} register holds and the
third argument is the number of double words to display \emph{above}
the address in {\code EBP}.

\item[dump\_math]
\index{I/O!asm\_io library!dump\_math} 
This macro prints out the values of the registers of the math coprocessor.
It takes a single integer argument that is used to label the output just as
the argument of {\code dump\_regs} does.
\end{description}
\index{debugging|)}
\index{I/O!asm\_io library|)} 
\index{I/O|)}

\section{Creating a Program}

Today, it is unusual to create a stand alone program written
completely in assembly language. Assembly is usually used to key certain
critical routines. Why? It is \emph{much} easier to program in a higher level 
language than in assembly. Also, using assembly makes a program very hard to
port to other platforms. In fact, it is rare to use assembly at all.

So, why should anyone learn assembly at all?
\begin{enumerate}
\item Sometimes code written in assembly can be faster and smaller than
      compiler generated code.
\item Assembly allows access to direct hardware features of the system that
      might be difficult or impossible to use from a higher level language.
\item Learning to program in assembly helps one gain a deeper understanding of
      how computers work.
\item Learning to program in assembly helps one understand better how compilers
      and high level languages like C work.
\end{enumerate}
These last two points demonstrate that learning assembly can be useful even if
one never programs in it later. In fact, the author rarely programs in
assembly, but he uses the ideas he learned from it everyday.

\subsection{First program}

\begin{figure}[t]
\begin{lstlisting}[frame=tlrb]{}
int main()
{
  int ret_status;
  ret_status = asm_main();
  return ret_status;
}
\end{lstlisting}
\caption{{\code driver.c} code\label{fig:driverProg} \index{C driver}}
\end{figure}

The early programs in this text will all start from the simple C
driver program in Figure~\ref{fig:driverProg}. It simply calls
another function named {\code asm\_main}. This is really a routine
that will be written in assembly. There are several advantages in
using the C driver routine. First, this lets the C system set up the
program to run correctly in protected mode. All the segments and their
corresponding segment registers will be initialized by C. The assembly
code need not worry about any of this. Secondly, the C library will
also be available to be used by the assembly code. The author's I/O
routines take advantage of this. They use C's I/O functions ({\code
printf}, {\em etc.}).  The following shows a simple assembly program.

\begin{AsmCodeListing}[label=first.asm]
; file: first.asm
; First assembly program. This program asks for two integers as
; input and prints out their sum.
;
; To create executable using djgpp:
; nasm -f coff first.asm
; gcc -o first first.o driver.c asm_io.o

%include "asm_io.inc"
;
; initialized data is put in the .data segment
;
segment .data
;
; These labels refer to strings used for output
;
prompt1 db    "Enter a number: ", 0       ; don't forget null terminator
prompt2 db    "Enter another number: ", 0
outmsg1 db    "You entered ", 0
outmsg2 db    " and ", 0
outmsg3 db    ", the sum of these is ", 0

;
; uninitialized data is put in the .bss segment
;
segment .bss
;
; These labels refer to double words used to store the inputs
;
input1  resd 1
input2  resd 1

;
; code is put in the .text segment
;
segment .text
        global  _asm_main
_asm_main:
        enter   0,0               ; setup routine
        pusha

        mov     eax, prompt1      ; print out prompt
        call    print_string

        call    read_int          ; read integer
        mov     [input1], eax     ; store into input1

        mov     eax, prompt2      ; print out prompt
        call    print_string

        call    read_int          ; read integer
        mov     [input2], eax     ; store into input2

        mov     eax, [input1]     ; eax = dword at input1
        add     eax, [input2]     ; eax += dword at input2
        mov     ebx, eax          ; ebx = eax

        dump_regs 1                ; print out register values
        dump_mem  2, outmsg1, 1    ; print out memory
;
; next print out result message as series of steps
;
        mov     eax, outmsg1
        call    print_string      ; print out first message
        mov     eax, [input1]     
        call    print_int         ; print out input1
        mov     eax, outmsg2
        call    print_string      ; print out second message
        mov     eax, [input2]
        call    print_int         ; print out input2
        mov     eax, outmsg3
        call    print_string      ; print out third message
        mov     eax, ebx
        call    print_int         ; print out sum (ebx)
        call    print_nl          ; print new-line

        popa
        mov     eax, 0            ; return back to C
        leave                     
        ret
\end{AsmCodeListing}

Line~13 of the program defines a section of the program that specifies
memory to be stored in the data segment (whose name is {\code
.data})\index{data segment}. Only initialized data should be defined
in this segment. On lines~17 to 21, several strings are declared. They
will be printed with the C library and so must be terminated with a
\emph{null} character (ASCII code 0).  Remember there is a big
difference between {\code 0} and {\code '0'}.

Uninitialized data should be declared in the bss segment (named {\code
.bss} on line 26)\index{bss segment}. This segment gets its name from an early UNIX-based
assembler operator that meant ``block started by symbol.'' There is
also a stack segment too. It will be discussed later.

The code segment \index{code segment} is named {\code .text}
historically. It is where instructions are placed. Note that the code
label for the main routine (line~38) has an underscore prefix.  This
is part of the \emph{C calling convention}. \index{calling
convention!C} This convention specifies the rules C uses when compiling
code. It is very important to know this convention when interfacing C
and assembly. Later the entire convention will be presented; however,
for now, one only needs to know that all C symbols ({\em i.e.},
functions and global variables) have a underscore prefix appended to
them by the C compiler. (This rule is specifically for DOS/Windows,
the Linux C compiler does not prepend anything to C symbol names.)

The {\code global} {\index{directive!global} directive on line 37
tells the assembler to make the {\code \_asm\_main} label
global. Unlike in C, labels have \emph{internal scope} by
default. This means that only code in the same module can use the
label. The {\code global} directive gives the specified label (or
labels) \emph{external scope}. This type of label can be accessed by
any module in the program. The {\code asm\_io} module declares the
{\code print\_int}, {\em et.al.\/} labels to be global. This is why
one can use them in the {\code first.asm} module.

\subsection{Compiler dependencies}

The assembly code above is specific to the free GNU\footnote{GNU is a
project of the Free Software Foundation ({\code
http://www.fsf.org})}-based DJGPP \index{compiler!DJGPP} C/C++
compiler.\footnote{\code http://www.delorie.com/djgpp} This compiler
can be freely downloaded from the Internet. It requires a 386-based PC
or better and runs under DOS, Windows 95/98 or NT. This compiler uses
object files in the COFF (Common Object File Format) format. To
assemble to this format use the {\code -f~coff} switch with {\code
nasm} (as shown in the comments of the above code). The extension of
the resulting object file will be {\code o}.

The Linux C compiler is a GNU compiler also. \index{compiler!gcc} To
convert the code above to run under Linux, simply remove the
underscore prefixes in lines~37 and 38. Linux uses the ELF (Executable
and Linkable Format) format for object files. Use the {\code -f~elf}
switch for Linux. It also produces an object with an {\code o}
extension.\MarginNote{The compiler specific example files, available
from the author's web site, have already been modified to work with
the appropriate compiler.}

Borland C/C++ \index{compiler!Borland} is another popular compiler. It
uses the Microsoft OMF format for object files. Use the {\code -f~obj}
switch for Borland compilers. The extension of the object file will be
{\code obj}. The OMF format uses different {\code segment} directives
than the other object formats. The data segment (line~13) must be
changed to:
\begin{CodeQuote}
segment \_DATA public align=4 class=DATA use32
\end{CodeQuote}
The bss segment (line 26) must be changed to:
\begin{CodeQuote}
segment \_BSS public align=4 class=BSS use32
\end{CodeQuote}
The text segment (line 36) must be changed to:
\begin{CodeQuote}
segment \_TEXT public align=1 class=CODE use32
\end{CodeQuote}
In addition a new line should be added before line 36:
\begin{CodeQuote}
group DGROUP \_BSS \_DATA
\end{CodeQuote}

The Microsoft C/C++ \index{compiler!Microsoft} compiler can use either
the OMF format or the Win32 format for object files. (If given a OMF
format, it converts the information to Win32 format internally.) Win32
format allows segments to be defined just as for DJGPP and Linux. Use
the {\code -f~win32} switch to output in this mode. The extension of
the object file will be {\code obj}.

\subsection{Assembling the code}

The first step is to assemble the code. From the command line, type:
\begin{CodeQuote}
nasm -f {\em object-format} first.asm
\end{CodeQuote}
where {\em object-format} is either {\em coff\/}, {\em elf\/}, {\em obj} or
{\em win32} depending on what C compiler will be used. (Remember that the
source file must be changed for both Linux and Borland as well.)


\subsection{Compiling the C code}

Compile the {\code driver.c} file using a C compiler. For DJGPP, use:
\begin{CodeQuote}
gcc -c driver.c
\end{CodeQuote}
The {\code -c} switch means to just compile, do not attempt to link yet. This
same switch works on Linux, Borland and Microsoft compilers as well.

\subsection{Linking the object files \label{seq:linking} \index{linking|(}}

Linking is the process of combining the machine code and data in
object files and library files together to create an executable
file. As will be shown below, this process is complicated.

C code requires the standard C library and special \emph{startup code}
\index{startup code} to run.  It is \emph{much} easier to let the C
compiler call the linker with the correct parameters, than to try to
call the linker directly. For example, to link the code for the first
program using DJGPP, \index{compiler!DJGPP} use:
\begin{CodeQuote}
gcc -o first driver.o first.o asm\_io.o
\end{CodeQuote}
This creates an executable called {\code first.exe} (or just {\code first}
under Linux). 

With Borland, \index{compiler!Borland} one would use:
\begin{CodeQuote}
bcc32 first.obj driver.obj asm\_io.obj
\end{CodeQuote}
Borland uses the name of the first file listed to determine the executable
name. So in the above case, the program would be named {\code first.exe}.

It is possible to combine the compiling and linking step. For example,
\begin{CodeQuote}
gcc -o first {\em driver.c} first.o asm\_io.o
\end{CodeQuote}
Now {\code gcc} will compile {\code driver.c} and then link.
\index{linking|)}

\subsection{Understanding an assembly listing file \index{listing file|(}}

The {\code -l {\em listing-file}} switch can be used to tell {\code
nasm} to create a listing file of a given name. This file shows how
the code was assembled. Here is how lines~17 and 18 (in the data
segment) appear in the listing file. (The line numbers are in the
listing file; however notice that the line numbers in the source file
may not be the same as the line numbers in the listing file.)
\begin{Verbatim}[xleftmargin=\AsmMargin]
48 00000000 456E7465722061206E-     prompt1 db    "Enter a number: ", 0
49 00000009 756D6265723A2000
50 00000011 456E74657220616E6F-     prompt2 db    "Enter another number: ", 0
51 0000001A 74686572206E756D62-
52 00000023 65723A2000
 \end{Verbatim}
The first column in each line is the line number and the second is the
offset (in hex) of the data in the segment. The third column shows the
raw hex values that will be stored. In this case the hex data
correspond to ASCII codes. Finally, the text from the source file is
displayed on the line. The offsets listed in the second column are
very likely \emph{not} the true offsets that the data will be placed
at in the complete program.  Each module may define its own labels in
the data segment (and the other segments, too). In the link step (see
section~\ref{seq:linking}), all these data segment label definitions
are combined to form one data segment. The new final offsets are then
computed by the linker.

Here is a small section (lines~54 to 56 of the source file) of the
text segment in the listing file:
\begin{Verbatim}[xleftmargin=\AsmMargin]
94 0000002C A1[00000000]          mov     eax, [input1]
95 00000031 0305[04000000]        add     eax, [input2]
96 00000037 89C3                  mov     ebx, eax
\end{Verbatim}
The third column shows the machine code generated by the
assembly. Often the complete code for an instruction can not be
computed yet. For example, in line~94 the offset (or address) of
{\code input1} is not known until the code is linked. The assembler
can compute the op-code for the {\code mov} instruction (which from
the listing is A1), but it writes the offset in square brackets
because the exact value can not be computed yet. In this case, a
temporary offset of 0 is used because {\code input1} is at the
beginning of the part of the bss segment defined in this
file. Remember this does \emph{not} mean that it will be at the
beginning of the final bss segment of the program. When the code is linked,
the linker will insert the correct offset into the position. Other 
instructions, like line~96, do not reference any labels. Here the assembler
can compute the complete machine code.
\index{listing file|)}

\subsubsection{Big and Little Endian Representation \index{endianess|(}}
If one looks closely at line~95, something seems very strange about
the offset in the square brackets of the machine code. The {\code
input2} label is at offset 4 (as defined in this file); however, the
offset that appears in memory is not 00000004, but 04000000. Why? Different
processors store multibyte integers in different orders in
memory. There are two popular methods of storing integers: \emph{big
endian} and \emph{little endian}. \MarginNote{Endian is pronounced
like \emph{indian}.} Big endian is the method that seems the most
natural. The biggest (\emph{i.e.} most significant) byte is stored
first, then the next biggest, \emph{etc.} For example, the dword
00000004 would be stored as the four bytes 00~00~00~04.  IBM
mainframes, most RISC processors and Motorola processors all use this
big endian method. However, Intel-based processors use the little
endian method! Here the least significant byte is stored first. So,
00000004 is stored in memory as 04~00~00~00. This format is hardwired
into the CPU and can not be changed. Normally, the programmer does not
need to worry about which format is used. However, there are
circumstances where it is important.
\begin{enumerate}
\item When binary data is transfered between different computers (either from
      files or through a network).
\item When binary data is written out to memory as a multibyte integer and
      then read back as individual bytes or \emph{vice versa}.
\end{enumerate}

Endianness does not apply to the order of array elements. The first
element of an array is always at the lowest address. This applies to
strings (which are just character arrays). Endianness still
applies to the individual elements of the arrays. 
\index{endianess|)}

\begin{figure}[t]
\begin{AsmCodeListing}[label=skel.asm]
%include "asm_io.inc"
segment .data
;
; initialized data is put in the data segment here
;

segment .bss
;
; uninitialized data is put in the bss segment
;

segment .text
        global  _asm_main
_asm_main:
        enter   0,0               ; setup routine
        pusha

;
; code is put in the text segment. Do not modify the code before
; or after this comment.
;

        popa
        mov     eax, 0            ; return back to C
        leave                     
        ret
\end{AsmCodeListing}
\caption{Skeleton Program \label{fig:skel}}
\end{figure}

\section{Skeleton File \index{skeleton file}}

Figure~\ref{fig:skel} shows a skeleton file that can be used as a starting
point for writing assembly programs.









\chapter{Basic Assembly Language}

\section{Working with Integers \index{integer|(}}

\subsection{Integer representation \index{integer!representation|(}}

\index{integer!unsigned|(}
Integers come in two flavors: unsigned and signed. Unsigned integers (which
are non-negative) are represented in a very straightforward binary manner.
The number 200 as an one byte unsigned integer would be represented as by
11001000 (or C8 in hex).
\index{integer!unsigned|)}

\index{integer!signed|(}
Signed integers (which may be positive or negative) are represented in a more
complicated ways. For example, consider $-56$. $+56$ as a byte would be
represented by 00111000. On paper, one could represent $-56$ as $-111000$,
but how would this be represented in a byte in the computer's memory.
How would the minus sign be stored?

There are three general techniques that have been used to represent
signed integers in computer memory. All of these methods use the most
significant bit of the integer as a \emph{sign
bit}. \index{integer!sign bit} This bit is 0 if the number is positive
and 1 if negative.

\subsubsection{Signed magnitude \index{integer!representation!signed magnitude}}

The first method is the simplest and is called \emph{signed magnitude}. It
represents the integer as two parts. The first part is the sign bit and
the second is the magnitude of the integer. So 56 would be represented as
the byte $\underline{0}0111000$ (the sign bit is underlined) and $-56$
would be $\underline{1}0111000$. The largest byte value would be
$\underline{0}1111111$ or $+127$ and the smallest byte value would be
$\underline{1}1111111$ or $-127$. To negate a value, the sign bit is reversed.
This method is straightforward, but it does have its drawbacks. First,
there are two possible values of zero, $+0$ ($\underline{0}0000000$) and
$-0$ ($\underline{1}0000000$). Since zero is neither positive nor negative,
both of these representations should act the same. This complicates the logic
of arithmetic for the CPU. Secondly, general arithmetic is also complicated.
If 10 is added to $-56$, this must be recast as 10 subtracted by 56. Again,
this complicates the logic of the CPU.

\subsubsection{One's complement \index{integer!representation!one's complement}}
The second method is known as \emph{one's complement} representation. The
one's complement of a number is found by reversing each bit in the number.
(Another way to look at it is that the new bit value is $1 - \mathrm{old bit value}$.) 
For example, the one's complement of 
$\underline{0}0111000$ ($+56$) is $\underline{1}1000111$. In one's complement
notation, computing the one's complement is equivalent to negation. Thus,
$\underline{1}1000111$ is the representation for $-56$. Note that the sign
bit was automatically changed by one's complement and that as one would
expect taking the one's complement twice yields the original number. As for
the first method, there are two representations of zero: 
$\underline{0}0000000$ ($+0$) and $\underline{1}1111111$ ($-0$). Arithmetic
with one's complement numbers is complicated.

There is a handy trick to finding the one's complement of a number in
hexadecimal without converting it to binary. The trick is to subtract the
hex digit from F (or 15 in decimal). This method assumes that the number of
bits in the number is a multiple of 4. Here is an example: $+56$ is
represented by 38 in hex. To find the one's complement, subtract each
digit from F to get C7 in hex. This agrees with the result above.

\subsubsection{Two's complement \index{integer!representation!two's complement|(}
               \index{two's complement|(}}

The first two methods described were used on early computers. Modern computers
use a third method called \emph{two's complement} representation. The two's
complement of a number is found by the following two steps:
\begin{enumerate}
\item Find the one's complement of the number
\item Add one to the result of step 1
\end{enumerate}
Here's an example using $\underline{0}0111000$ (56). First the one's complement
is computed: $\underline{1}1000111$. Then one is added:
\[
\begin{array}{rr}
 & \underline{1}1000111 \\
+&                    1 \\ \hline
 & \underline{1}1001000
\end{array}
\]

In two complement's notation, computing the two's complement is equivalent
to negating a number. Thus, $\underline{1}1001000$ is the two's complement
representation of $-56$. Two negations should reproduce the original number.
Surprising two's complement does meet this requirement. Take the two's
complement of $\underline{1}1001000$ by adding one to the one's complement.
\[
\begin{array}{rr}
 & \underline{0}0110111 \\
+&                    1 \\ \hline
 & \underline{0}0111000
\end{array}
\]

When performing the addition in the two's complement operation, the addition
of the leftmost bit may produce a carry. This carry is \emph{not} used. 
Remember that all data on the computer is of some fixed size (in terms of
number of bits). Adding two bytes always produces a byte as a result (just
as adding two words produces a word, {\em etc\/}.) This property is important 
for two's complement notation. For example, consider zero as a one byte
two's complement number ($\underline{0}0000000$). Computing its two complement
produces the sum:
\[
\begin{array}{rr}
 & \underline{1}1111111 \\
+&                    1 \\ \hline
c& \underline{0}0000000
\end{array}
\]
where $c$ represents a carry. (Later it will be shown how to detect this carry,
but it is not stored in the result.) Thus, in two's complement notation there
is only one zero. This makes two's complement arithmetic simpler than the
previous methods.

\begin{table}
\centering
\begin{tabular}{||c|c||}
\hline
Number & Hex Representation \\
\hline
0 & 00 \\
1 & 01 \\
127 & 7F \\
-128 & 80 \\
-127 & 81 \\
-2 & FE \\
-1 & FF \\
\hline
\end{tabular}
\caption{Two's Complement Representation \label{tab:twocomp}}
\end{table}

Using two's complement notation, a signed byte can be used to
represent the numbers $-128$ to $+127$. Table~\ref{tab:twocomp} shows some
selected values. If 16 bits are used, the signed numbers $-32,768$ to 
$+32,767$ can be represented. $+32,767$ is represented by 7FFF, 
$-32,768$ by 8000, -128 as FF80 and -1 as FFFF. 32 bit two's complement 
numbers range from $-2$ billion to $+2$ billion approximately. 


The CPU has no idea what a particular byte (or word or double word) is
supposed to represent. Assembly does not have the idea of types that a
high level language has. How data is interpreted depends on what instruction
is used on the data. Whether the hex value FF is considered to represent
a signed $-1$ or a unsigned $+255$ depends on the programmer. The C language
defines signed and unsigned integer types. This allows a C compiler to
determine the correct instructions to use with the data.

\index{two's complement|)}
\index{integer!representation!two's complement|)}
\index{integer!signed|)}

\subsection{Sign extension \index{integer!sign extension|(}}

In assembly, all data has a specified size. It is not uncommon to need to
change the size of data to use it with other data. Decreasing size is the
easiest.

\subsubsection{Decreasing size of data}

To decrease the size of data, simply remove the more significant bits of the
data. Here's a trivial example:
\begin{AsmCodeListing}[numbers=none,frame=none]
      mov    ax, 0034h      ; ax = 52 (stored in 16 bits)
      mov    cl, al         ; cl = lower 8-bits of ax
\end{AsmCodeListing}

Of course, if the number can not be represented correctly in the smaller
size, decreasing the size does not work. For example, if {\code AX}
were 0134h (or 308 in decimal) then the above code would still set
{\code CL} to 34h. This method works with both signed and unsigned
numbers.  Consider signed numbers, if {\code AX} was FFFFh ($-1$ as a
word), then {\code CL} would be FFh ($-1$ as a byte). However, note
that this is not correct if the value in {\code AX} was unsigned!

The rule for unsigned numbers is that all the bits being removed must
be 0 for the conversion to be correct. The rule for signed numbers is
that the bits being removed must be either all 1's or all 0's. In
addition, the first bit not being removed must have the same value as
the removed bits. This bit will be the new sign bit of the smaller value.
It is important that it be same as the original sign bit!

\subsubsection{Increasing size of data}

Increasing the size of data is more complicated than
decreasing. Consider the hex byte FF. If it is extended to a word,
what value should the word have?  It depends on how FF is
interpreted. If FF is a unsigned byte (255 in decimal), then the word
should be 00FF; however, if it is a signed byte ($-1$ in decimal),
then the word should be FFFF.

In general, to extend an unsigned number, one makes all the new bits
of the expanded number 0. Thus, FF becomes 00FF. However, to extend a
signed number, one must \emph{extend} the sign
bit. \index{integer!sign bit} This means that the new bits become
copies of the sign bit. Since the sign bit of FF is 1, the new bits
must also be all ones, to produce FFFF. If the signed number 5A (90 in
decimal) was extended, the result would be 005A.

There are several instructions that the 80386 provides for extension of
numbers. Remember that the computer does not know whether a number is signed
or unsigned. It is up to the programmer to use the correct instruction.

For unsigned numbers, one can simply put zeros in the upper bits using a
{\code MOV} instruction. For example, to extend the byte in AL to an unsigned
word in AX:
\begin{AsmCodeListing}[numbers=none,frame=none]
      mov    ah, 0   ; zero out upper 8-bits
\end{AsmCodeListing}
However, it is not possible to use a {\code MOV} instruction to
convert the unsigned word in AX to an unsigned double word in EAX. Why
not? There is no way to specify the upper 16 bits of EAX in a {\code
MOV}. The 80386 solves this problem by providing a new instruction
{\code MOVZX}. \index{MOVZX} This instruction has two operands. The destination
(first operand) must be a 16 or 32~bit register.  The source (second
operand) may be an 8 or 16~bit register or a byte or word of
memory. The other restriction is that the destination must be larger than
the source. (Most instructions require the source and destination to be
the same size.) Here are some examples:
\begin{AsmCodeListing}[numbers=none,frame=none]
      movzx  eax, ax      ; extends ax into eax
      movzx  eax, al      ; extends al into eax
      movzx  ax, al       ; extends al into ax
      movzx  ebx, ax      ; extends ax into ebx
\end{AsmCodeListing}

For signed numbers, there is no easy way to use the {\code MOV}
instruction for any case. The 8086 provided several instructions to
extend signed numbers.  The {\code CBW} \index{CBW} (Convert Byte to
Word) instruction sign extends the AL register into AX. The operands
are implicit. The {\code CWD} \index{CWD} (Convert Word to Double
word) instruction sign extends AX into DX:AX. The notation DX:AX means
to think of the DX and AX registers as one 32~bit register with the
upper 16 bits in DX and the lower bits in AX. (Remember that the 8086
did not have any 32~bit registers!) The 80386 added several new
instructions. The {\code CWDE} \index{CWDE} (Convert Word to Double
word Extended) instruction sign extends AX into EAX. The {\code CDQ}
\index{CDQ} (Convert Double word to Quad word) instruction sign
extends EAX into EDX:EAX\index{register!EDX:EAX} (64~bits!). Finally, the
{\code MOVSX} \index{MOVSX} instruction works like {\code MOVZX}
except it uses the rules for signed numbers.

\subsubsection{Application to C programming}

Extending \MarginNote{ANSI C does not define whether the {\code char}
type is signed or not, it is up to each individual compiler to decide
this. That is why the type is explicitly defined in
Figure~\ref{fig:charExt}.} of unsigned and signed integers also occurs
in C. Variables in C may be declared as either signed or unsigned
({\code int} is signed). Consider the code in
Figure~\ref{fig:charExt}.  In line~3, the variable {\code a} is
extended using the rules for unsigned values (using {\code MOVZX}), but in 
line~4, the signed rules are used for {\code b} (using {\code MOVSX}).

\begin{figure}[t]
\begin{lstlisting}[frame=tlrb]{}
unsigned char uchar = 0xFF;
signed char   schar = 0xFF;
int a = (int) uchar;     /* a = 255 (0x000000FF) */
int b = (int) schar;     /* b = -1  (0xFFFFFFFF) */
\end{lstlisting}
\caption{}
\label{fig:charExt}
\end{figure}

There is a common C programming bug that directly relates to this subject.
Consider the code in Figure~\ref{fig:IObug}. The prototype of 
{\code fgetc()}{\samepage is:
\begin{CodeQuote}
int fgetc( FILE * );
\end{CodeQuote}
One might question }why does the function return back an {\code int}
since it reads characters? The reason is that it normally does return
back an {\code char} (extended to an {\code int} value using zero
extension). However, there is one value that it may return that is not
a character, {\code EOF}. This is a macro that is usually defined as
$-1$. Thus, {\code fgetc()} either returns back a {\code char}
extended to an {\code int} value (which looks like {\code 000000{\em
xx}} in hex) or {\code EOF} (which looks like {\code FFFFFFFF} in
hex).

\begin{figure}[t]
\begin{lstlisting}[stepnumber=0,frame=tlrb]{}
char ch;
while( (ch = fgetc(fp)) != EOF ) {
  /* do something with ch */
}
\end{lstlisting}
\caption{}
\label{fig:IObug}
\end{figure}

The basic problem with the program in Figure~\ref{fig:IObug} is that
{\code fgetc()} returns an {\code int}, but this value is stored in a
{\code char}. C will truncate the higher order bits to fit the {\code
int} value into the {\code char}. The only problem is that the numbers
(in hex) {\code 000000FF} and {\code FFFFFFFF} both will be truncated
to the byte {\code FF}. Thus, the while loop can not distinguish
between reading the byte {\code FF} from the file and end of file.

Exactly what the code does in this case, depends on whether {\code char} is
signed or unsigned. Why? Because in line~2, {\code ch} is compared with 
{\code EOF}. Since {\code EOF} is an {\code int} value\footnote{It is a
common misconception that files have an EOF character at their end. This is
\emph{not} true!}, {\code ch} will be
extended to an {\code int} so that two values being compared are of the
same size\footnote{The reason for this requirement will be shown later.}.
As Figure~\ref{fig:charExt} showed, where the variable is signed or 
unsigned is very important.

If {\code char} is unsigned, {\code FF} is extended to be {\code
000000FF}. This is compared to {\code EOF} ({\code FFFFFFFF}) and
found to be not equal. Thus, the loop never ends!

If {\code char} is signed, {\code FF} is extended to {\code
FFFFFFFF}. This does compare as equal and the loop ends. However,
since the byte {\code FF} may have been read from the file, the loop
could be ending prematurely.

The solution to this problem is to define the {\code ch} variable as an
{\code int}, not a {\code char}. When this is done, no truncating or
extension is done in line~2. Inside the loop, it is safe to truncate the
value since {\code ch} \emph{must} actually be a simple byte there.

\index{integer!sign extension|)}
\index{integer!representation|)}

\subsection{Two's complement arithmetic \index{two's complement!arithmetic|(}}

As was seen earlier, the {\code add} instruction performs addition and
the {\code sub} instruction performs subtraction. Two of the bits in
the FLAGS register that these instructions set are the \emph{overflow}
and \emph{carry flag}. The overflow flag is set if the true result of
the operation is too big to fit into the destination for signed
arithmetic. The carry flag is set if there is a carry in the msb of an
addition or a borrow in the msb of a subtraction. Thus, it can be used
to detect overflow for unsigned arithmetic. The uses of the carry flag
for signed arithmetic will be seen shortly. One of the great
advantages of 2's~complement is that the rules for addition and
subtraction are exactly the same as for unsigned integers. Thus,
{\code add} and {\code sub} may be used on signed or unsigned
integers.
\[
\begin{array}{rrcrr}
 & 002\mathrm{C} & & & 44\\
+& \mathrm{FFFF} & &+&(-1)\\ \cline{1-2} \cline{4-5}
 & 002\mathrm{B} & & & 43
\end{array}
\]
There is a carry generated, but it is not part of the answer.

\index{integer!multiplication|(}
\index{MUL|(}
\index{IMUL|(}
There are two different multiply and divide instructions. First, to
multiply use either the {\code MUL} or {\code IMUL} instruction. The
{\code MUL} instruction is used to multiply unsigned numbers and
{\code IMUL} is used to multiply signed integers. Why are two
different instructions needed? The rules for multiplication are
different for unsigned and 2's complement signed numbers. How so?
Consider the multiplication of the byte FF with itself yielding a word
result. Using unsigned multiplication this is 255 times 255 or 65025
(or FE01 in hex). Using signed multiplication this is $-1$ times $-1$ or
1 (or 0001 in hex).

There are several forms of the multiplication instructions. The oldest
form looks like:
\begin{AsmCodeListing}[numbers=none,frame=none]
      mul   source
\end{AsmCodeListing}
The \emph{source} is either a register or a memory reference. It can not be an 
immediate value. Exactly what multiplication is performed depends on the
size of the source operand. If the operand is byte sized, it is multiplied by
the byte in the AL register and the result is stored in the 16 bits of AX.
If the source is 16-bit, it is multiplied by the word in AX and the 32-bit
result is stored in DX:AX. If the source is 32-bit, it is multiplied by EAX
and the 64-bit result is stored into EDX:EAX\index{register!EDX:EAX}.
\index{MUL|)}

\begin{table}[t]
\centering
\begin{tabular}{|c|c|c|l|}
\hline
{ \bf dest} & { \bf source1 } & {\bf source2} & \multicolumn{1}{c|}{\bf Action} \\ \hline
            & reg/mem8        &               & AX = AL*source1 \\
            & reg/mem16       &               & DX:AX = AX*source1 \\
            & reg/mem32       &               & EDX:EAX = EAX*source1 \\
reg16       & reg/mem16       &               & dest *= source1 \\
reg32       & reg/mem32       &               & dest *= source1 \\
reg16       & immed8          &               & dest *= immed8 \\
reg32       & immed8          &               & dest *= immed8 \\
reg16       & immed16         &               & dest *= immed16 \\
reg32       & immed32         &               & dest *= immed32 \\
reg16       & reg/mem16       & immed8        & dest = source1*source2 \\
reg32       & reg/mem32       & immed8        & dest = source1*source2 \\
reg16       & reg/mem16       & immed16       & dest = source1*source2 \\
reg32       & reg/mem32       & immed32       & dest = source1*source2 \\
\hline
\end{tabular}
\caption{{\code imul} Instructions \label{tab:imul}}
\end{table}

The {\code IMUL} instruction has the same formats as {\code MUL}, but also
adds some other instruction formats. There are two and three operand formats:
\begin{AsmCodeListing}[numbers=none,frame=none]
      imul   dest, source1
      imul   dest, source1, source2
\end{AsmCodeListing}
Table~\ref{tab:imul} shows the possible combinations.
\index{IMUL|)}
\index{integer!multiplication|)}

\index{integer!division|(}
\index{DIV}
The two division operators are {\code DIV} and {\code IDIV}. They perform
unsigned and signed integer division respectively. The general format is:
\begin{AsmCodeListing}[numbers=none,frame=none]
      div   source
\end{AsmCodeListing}
If the source is 8-bit, then AX is divided by the operand. The
quotient is stored in AL and the remainder in AH. If the source is
16-bit, then DX:AX is divided by the operand. The quotient is stored
into AX and remainder into DX. If the source is 32-bit, then
EDX:EAX\index{register!EDX:EAX} is divided by the operand and the quotient is
stored into EAX and the remainder into EDX. The {\code IDIV}
\index{IDIV} instruction works the same way. There are no special
{\code IDIV} instructions like the special {\code IMUL} ones. If the
quotient is too big to fit into its register or the divisor is zero,
the program is interrupted and terminates. A very common error is to
forget to initialize DX or EDX before division.
\index{integer!division|)}

The {\code NEG} \index{NEG} instruction negates its single operand by computing
its two's complement. Its operand may be any 8-bit, 16-bit, or 32-bit
register or memory location.

\subsection{Example program}
\index{math.asm|(}
\begin{AsmCodeListing}[label=math.asm]
%include "asm_io.inc"
segment .data         ; Output strings
prompt          db    "Enter a number: ", 0
square_msg      db    "Square of input is ", 0
cube_msg        db    "Cube of input is ", 0
cube25_msg      db    "Cube of input times 25 is ", 0
quot_msg        db    "Quotient of cube/100 is ", 0
rem_msg         db    "Remainder of cube/100 is ", 0
neg_msg         db    "The negation of the remainder is ", 0

segment .bss
input   resd 1

segment .text
        global  _asm_main
_asm_main:
        enter   0,0               ; setup routine
	pusha

        mov     eax, prompt
        call    print_string

        call    read_int
        mov     [input], eax

        imul    eax               ; edx:eax = eax * eax
        mov     ebx, eax          ; save answer in ebx
        mov     eax, square_msg
        call    print_string
        mov     eax, ebx
        call    print_int
        call    print_nl

        mov     ebx, eax
        imul    ebx, [input]      ; ebx *= [input]
        mov     eax, cube_msg
        call    print_string
        mov     eax, ebx
        call    print_int
        call    print_nl

        imul    ecx, ebx, 25      ; ecx = ebx*25
        mov     eax, cube25_msg
        call    print_string
        mov     eax, ecx
        call    print_int
        call    print_nl

        mov     eax, ebx
        cdq                       ; initialize edx by sign extension
        mov     ecx, 100          ; can't divide by immediate value
        idiv    ecx               ; edx:eax / ecx
        mov     ecx, eax          ; save quotient into ecx
        mov     eax, quot_msg
        call    print_string
        mov     eax, ecx
        call    print_int
        call    print_nl
        mov     eax, rem_msg
        call    print_string
        mov     eax, edx
        call    print_int
        call    print_nl
        
        neg     edx               ; negate the remainder
        mov     eax, neg_msg
        call    print_string
        mov     eax, edx
        call    print_int
        call    print_nl

        popa
        mov     eax, 0            ; return back to C
        leave                     
        ret
\end{AsmCodeListing}
\index{math.asm|)}

\subsection{Extended precision arithmetic \label{sec:ExtPrecArith} \index{integer!extended precision|(}}}

Assembly language also provides instructions that allow one to perform
addition and subtraction of numbers larger than double words. These
instructions use the carry flag. As stated above, both the {\code ADD}
\index{ADD} and {\code SUB} \index{SUB} instructions modify the carry
flag if a carry or borrow are generated, respectively. This
information stored in the carry flag can be used to add or subtract
large numbers by breaking up the operation into smaller double word
(or smaller) pieces.

The {\code ADC} \index{ADC} and {\code SBB} \index{SBB} instructions
use this information in the carry flag. The {\code ADC} instruction
performs the following operation:
\begin{center}
{\code \emph{operand1} = \emph{operand1} + carry flag + \emph{operand2} }
\end{center}
The {\code SBB} instruction performs:
\begin{center}
{\code \emph{operand1} = \emph{operand1} - carry flag - \emph{operand2} }
\end{center}
How are these used? Consider the sum of 64-bit integers in
EDX:EAX\index{register!EDX:EAX} and EBX:ECX. The following code would store the
sum in EDX:EAX:
\begin{AsmCodeListing}[frame=none]
      add    eax, ecx       ; add lower 32-bits
      adc    edx, ebx       ; add upper 32-bits and carry from previous sum
\end{AsmCodeListing}
Subtraction is very similar. The following code subtracts EBX:ECX from 
EDX:EAX:
\begin{AsmCodeListing}[frame=none]
      sub    eax, ecx       ; subtract lower 32-bits
      sbb    edx, ebx       ; subtract upper 32-bits and borrow
\end{AsmCodeListing}

For \emph{really} large numbers, a loop could be used (see 
Section~\ref{sec:control}). For a sum loop, it would be convenient to use
{\code ADC} instruction for every iteration (instead of all but the first
iteration). This can be done by using the {\code CLC} \index{CLC} (CLear Carry)
instruction right before the loop starts to initialize the carry flag to 0.
If the carry flag is 0, there is no difference between the {\code ADD} and
{\code ADC} instructions. The same idea can be used for subtraction, too.
\index{integer!extended precision|)}
\index{two's complement!arithmetic|)}

\section{Control Structures}
\label{sec:control}
High level languages provide high level control structures (\emph{e.g.}, the
\emph{if} and \emph{while} statements) that control the thread of execution.
Assembly language does not provide such complex control structures. It instead
uses the infamous \emph{goto} and used inappropriately can result in
spaghetti code! However, it \emph{is} possible to write structured assembly
language programs. The basic procedure is to design the program logic using
the familiar high level control structures and translate the design into
the appropriate assembly language (much like a compiler would do).

\subsection{Comparisons \index{integer!comparisons|(} \index{CMP|(}}
%TODO: Make a table of all the FLAG bits

\index{register!FLAGS|(}
Control structures decide what to do based on comparisons of data. In
assembly, the result of a comparison is stored in the FLAGS register
to be used later. The 80x86 provides the {\code CMP} instruction to
perform comparisons.  The FLAGS register is set based on the
difference of the two operands of the {\code CMP} instruction. The
operands are subtracted and the FLAGS are set based on the result, but
the result is \emph{not} stored anywhere. If you need the result use
the SUB instead of the {\code CMP} instruction.

\index{integer!unsigned|(}
For unsigned integers, there are two flags (bits in the FLAGS
register) that are important: the zero (ZF) \index{register!FLAGS!ZF} and carry (CF) 
\index{register!FLAGS!CF} flags. The
zero flag is set (1) if the resulting difference would be zero. The
carry flag is used as a borrow flag for subtraction. Consider a
comparison like:
\begin{AsmCodeListing}[frame=none, numbers=none]
      cmp    vleft, vright
\end{AsmCodeListing}
The difference of {\code vleft~-~vright} is computed and the flags are
set accordingly. If the difference of the of {\code CMP} is zero, {\code 
vleft~=~vright}, then ZF is set (\emph{i.e.} 1) and the CF is unset
(\emph{i.e.} 0). If {\code vleft~>~vright}, then ZF is unset and CF
is unset (no borrow). If {\code vleft~<~vright}, then ZF is unset and
CF is set (borrow).
\index{integer!unsigned|)}

\index{integer!signed|(} 
For signed integers, there are three flags
that are important: the zero \index{register!FLAGS!ZF} (ZF) flag, the
overflow \index{register!FLAGS!OF}(OF) flag and the sign
\index{register!FLAGS!SF} (SF) flag. \MarginNote{Why does SF~=~OF if
{\code vleft~>~vright}? If there is no overflow, then the difference
will have the correct value and must be non-negative. Thus,
SF~=~OF~=~0. However, if there is an overflow, the difference will not
have the correct value (and in fact will be negative). Thus,
SF~=~OF~=~1.}The overflow flag is set if the result of an operation
overflows (or underflows). The sign flag is set if the result of an
operation is negative. If {\code vleft~=~vright}, the ZF is set (just
as for unsigned integers). If {\code vleft~>~vright}, ZF is unset and
SF~=~OF.  If {\code vleft~<~vright}, ZF is unset and SF~$\neq$~OF.
\index{integer!signed|)}

Do not forget that other instructions can also change the FLAGS
register, not just {\code CMP}.
\index{CMP|)}
\index{integer!comparisons|)}
\index{register!FLAGS|)}
\index{integer|)}

\subsection{Branch instructions}

Branch instructions can transfer execution to arbitrary points of a program.
In other words, they act like a \emph{goto}. There are two types of branches:
unconditional and conditional. An unconditional branch is just like a goto,
it always makes the branch. A conditional branch may or may not make the
branch depending on the flags in the FLAGS register. If a conditional branch
does not make the branch, control passes to the next instruction.

\index{JMP|(}
The {\code JMP} (short for \emph{jump}) instruction makes
unconditional branches. Its single argument is usually a \emph{code
label} to the instruction to branch to. The assembler or linker will
replace the label with correct address of the instruction. This is
another one of the tedious operations that the assembler does to make
the programmer's life easier. It is important to realize that the
statement immediately after the {\code JMP} instruction will never be
executed unless another instruction branches to it!

There are several variations of the jump instruction:
\begin{description}

\item[SHORT] This jump is very limited in range. It can only move up or
down 128 bytes in memory. The advantage of this type is that it uses less
memory than the others. It uses a single signed byte to store the 
\emph{displacement} of the jump. The displacement is how many bytes to move 
ahead or behind. (The displacement is added to EIP). To specify a
short jump, use the {\code SHORT} keyword immediately before the label
in the {\code JMP} instruction.

\item[NEAR] This jump is the default type for both unconditional and 
conditional branches, it can be used to jump to any location in a
segment. Actually, the 80386 supports two types of near jumps.  One
uses two bytes for the displacement. This allows one to move up or
down roughly 32,000 bytes. The other type uses four bytes for the
displacement, which of course allows one to move to any location in
the code segment. The four byte type is the default in 386 protected
mode. The two byte type can be specified by putting the {\code WORD}
keyword before the label in the {\code JMP} instruction.

\item[FAR] This jump allows control to move to another code segment. This is
a very rare thing to do in 386 protected mode.
\end{description}

Valid code labels follow the same rules as data labels. Code labels
are defined by placing them in the code segment in front of the statement
they label. A colon is placed at the end of the label at its point of
definition. The colon is \emph{not} part of the name.
\index{JMP|)}

\index{conditional branch|(}
\begin{table}[t]
\center
\begin{tabular}{|ll|}
\hline
JZ  & branches only if ZF is set \\
JNZ & branches only if ZF is unset \\
JO  & branches only if OF is set \\
JNO & branches only if OF is unset \\
JS  & branches only if SF is set \\
JNS & branches only if SF is unset \\
JC  & branches only if CF is set \\
JNC & branches only if CF is unset \\
JP  & branches only if PF is set \\
JNP & branches only if PF is unset \\
\hline
\end{tabular}
\caption{Simple Conditional Branches \label{tab:SimpBran} \index{JZ} \index{JNZ}
        \index{JO} \index{JNO} \index{JS} \index{JNS} \index{JC} \index{JNC}
        \index{JP} \index{JNP}}
\end{table}

There are many different conditional branch instructions. They also
take a code label as their single operand. The simplest ones just look
at a single flag in the FLAGS register to determine whether to branch
or not.  See Table~\ref{tab:SimpBran} for a list of these
instructions. (PF is the \emph{parity flag} \index{register!FLAGS!PF}
which indicates the odd or evenness of the number of bits set in the
lower 8-bits of the result.)

The following pseudo-code:
\begin{Verbatim}
if ( EAX == 0 )
  EBX = 1;
else
  EBX = 2;
\end{Verbatim}
could be written in assembly as:
\begin{AsmCodeListing}[frame=none]
      cmp    eax, 0            ; set flags (ZF set if eax - 0 = 0)
      jz     thenblock         ; if ZF is set branch to thenblock
      mov    ebx, 2            ; ELSE part of IF
      jmp    next              ; jump over THEN part of IF
thenblock:
      mov    ebx, 1            ; THEN part of IF
next:
\end{AsmCodeListing}

Other comparisons are not so easy using the conditional branches in 
Table~\ref{tab:SimpBran}. To illustrate, consider the following pseudo-code:
\begin{Verbatim}
if ( EAX >= 5 )
  EBX = 1;
else
  EBX = 2;
\end{Verbatim}
If EAX is greater than or equal to five, the ZF may be set or unset and SF
will equal OF. Here is assembly code that tests for these conditions 
(assuming that EAX is signed):
\begin{AsmCodeListing}[frame=none]
      cmp    eax, 5
      js     signon            ; goto signon if SF = 1
      jo     elseblock         ; goto elseblock if OF = 1 and SF = 0
      jmp    thenblock         ; goto thenblock if SF = 0 and OF = 0
signon:
      jo     thenblock         ; goto thenblock if SF = 1 and OF = 1
elseblock:
      mov    ebx, 2
      jmp    next
thenblock:
      mov    ebx, 1
next:
\end{AsmCodeListing}

\begin{table}
\center
\begin{tabular}{|ll|ll|}
\hline
\multicolumn{2}{|c|}{\textbf{Signed}} & \multicolumn{2}{c|}{\textbf{Unsigned}} \\
\hline
JE & branches if {\code vleft = vright} & JE & branches if {\code vleft = vright} \\
JNE & branches if {\code vleft $\neq$ vright} & JNE & branches if {\code vleft $\neq$ vright} \\
JL, JNGE & branches if {\code vleft < vright} & JB, JNAE & branches if {\code vleft < vright} \\
JLE, JNG & branches if {\code vleft $\leq$ vright} & JBE, JNA & branches if {\code vleft $\leq$ vright} \\
JG, JNLE & branches if {\code vleft > vright} & JA, JNBE & branches if {\code vleft > vright} \\
JGE, JNL & branches if {\code vleft $\geq$ vright} & JAE, JNB & branches if {\code vleft $\geq$ vright} \\
\hline
\end{tabular}
\caption{Signed and Unsigned Comparison Instructions \label{tab:CompBran} \index{JE} \index{JNE}
         \index{JL} \index{JNGE} \index{JLE} \index{JNG} \index{JG} \index{JNLE} \index{JGE}
         \index{JNL}}
\end{table}

The above code is very awkward. Fortunately, the 80x86 provides additional
branch instructions to make these type of tests \emph{much} easier. There
are signed and unsigned versions of each. Table~\ref{tab:CompBran} shows
these instructions. The equal and not equal branches (JE and JNE) are the
same for both signed and unsigned integers. (In fact, JE and JNE are really
identical to JZ and JNZ, respectively.) Each of the other branch 
instructions have two synonyms. For example, look at JL (jump less than) and
JNGE (jump not greater than or equal to). These are the same instruction
because:
\[ x < y \Longrightarrow \mathbf{not}( x \geq y ) \]
The unsigned branches use A for \emph{above} and B for \emph{below} instead of
L and G.

Using these new branch instructions, the pseudo-code above can be translated
to assembly much easier.
\begin{AsmCodeListing}[frame=none]
      cmp    eax, 5
      jge    thenblock
      mov    ebx, 2
      jmp    next
thenblock:
      mov    ebx, 1
next:
\end{AsmCodeListing}
\index{conditional branch|)}

\subsection{The loop instructions}

The 80x86 provides several instructions designed to implement 
\emph{for}-like loops. Each of these instructions takes a code label
as its single operand.
\begin{description}
\item[LOOP] 
\index{LOOP}
Decrements ECX, if ECX $\neq$ 0, branches to label
\item[LOOPE, LOOPZ]
\index{LOOPE} \index{LOOPZ}
Decrements ECX (FLAGS register is not modified), if
                    ECX $\neq$ 0 and ZF = 1, branches
\item[LOOPNE, LOOPNZ]
\index{LOOPNE} \index{LOOPNZ}
Decrements ECX (FLAGS unchanged), if ECX $\neq$ 0
                      and ZF = 0, branches
\end{description}

The last two loop instructions are useful for sequential search loops. The
following pseudo-code:
\begin{lstlisting}[stepnumber=0]{}
sum = 0;
for( i=10; i >0; i-- )
  sum += i;
\end{lstlisting}
\noindent could be translated into assembly as:
\begin{AsmCodeListing}[frame=none]
      mov    eax, 0          ; eax is sum
      mov    ecx, 10         ; ecx is i
loop_start:
      add    eax, ecx
      loop   loop_start
\end{AsmCodeListing}

\section{Translating Standard Control Structures}

This section looks at how the standard control structures of high level
languages can be implemented in assembly language.

\subsection{If statements \index{if statment|(}}
The following pseudo-code:
\begin{lstlisting}[stepnumber=0]{}
if ( condition )
  then_block;
else
  else_block;
\end{lstlisting}
\noindent could be implemented as:
\begin{AsmCodeListing}[frame=none]
      ; code to set FLAGS
      jxx    else_block    ; select xx so that branches if condition false
      ; code for then block
      jmp    endif
else_block:
      ; code for else block
endif:
\end{AsmCodeListing}

If there is no else, then the {\code else\_block} branch can be replaced by
a branch to {\code endif}.
\begin{AsmCodeListing}[frame=none]
      ; code to set FLAGS
      jxx    endif          ; select xx so that branches if condition false
      ; code for then block
endif:
\end{AsmCodeListing}
\index{if statment|)}

\subsection{While loops \index{while loop|(}}
The \emph{while} loop is a top tested loop:
\begin{lstlisting}[stepnumber=0]{}
while( condition ) {
  body of loop;
}
\end{lstlisting}
\noindent This could be translated into:
\begin{AsmCodeListing}[frame=none]
while:
      ; code to set FLAGS based on condition
      jxx    endwhile       ; select xx so that branches if false
      ; body of loop
      jmp    while
endwhile:
\end{AsmCodeListing}
\index{while loop|)}

\subsection{Do while loops \index{do while loop|(}}
The \emph{do while} loop is a bottom tested loop:
\begin{lstlisting}[stepnumber=0]{}
do {
  body of loop;
} while( condition );
\end{lstlisting}
\noindent This could be translated into:
\begin{AsmCodeListing}[frame=none]
do:
      ; body of loop
      ; code to set FLAGS based on condition
      jxx    do          ; select xx so that branches if true
\end{AsmCodeListing}
\index{do while loop|)}


\begin{figure}[t]
\begin{lstlisting}[frame=tlrb]{}
  unsigned guess;   /* current guess for prime      */
  unsigned factor;  /* possible factor of guess     */
  unsigned limit;   /* find primes up to this value */

  printf("Find primes up to: ");
  scanf("%u", &limit);
  printf("2\n");    /* treat first two primes as  */
  printf("3\n");    /* special case               */
  guess = 5;        /* initial guess */
  while ( guess <= limit ) {
    /* look for a factor of guess */
    factor = 3;
    while ( factor*factor < guess &&
            guess % factor != 0 )
     factor += 2;
    if ( guess % factor != 0 )
      printf("%d\n", guess);
    guess += 2;    /* only look at odd numbers */
  }
\end{lstlisting}
\caption{}\label{fig:primec}
\end{figure}

\section{Example: Finding Prime Numbers}
This section looks at a program that finds prime numbers. Recall that
prime numbers are evenly divisible by only 1 and themselves. There is
no formula for doing this. The basic method this program uses is to
find the factors of all odd numbers\footnote{2 is the only even prime
number.} below a given limit. If no factor can be found for an odd
number, it is prime.  Figure~\ref{fig:primec} shows the basic
algorithm written in C.

Here's the assembly version:
\index{prime.asm|(}
\begin{AsmCodeListing}[label=prime.asm]
%include "asm_io.inc"
segment .data
Message         db      "Find primes up to: ", 0

segment .bss
Limit           resd    1               ; find primes up to this limit
Guess           resd    1               ; the current guess for prime

segment .text
        global  _asm_main
_asm_main:
        enter   0,0               ; setup routine
        pusha

        mov     eax, Message
        call    print_string
        call    read_int             ; scanf("%u", & limit );
        mov     [Limit], eax

        mov     eax, 2               ; printf("2\n");
        call    print_int
        call    print_nl
        mov     eax, 3               ; printf("3\n");
        call    print_int
        call    print_nl

        mov     dword [Guess], 5     ; Guess = 5;
while_limit:                         ; while ( Guess <= Limit )
        mov     eax,[Guess]
        cmp     eax, [Limit]
        jnbe    end_while_limit      ; use jnbe since numbers are unsigned

        mov     ebx, 3               ; ebx is factor = 3;
while_factor:
        mov     eax,ebx
        mul     eax                  ; edx:eax = eax*eax
        jo      end_while_factor     ; if answer won't fit in eax alone
        cmp     eax, [Guess]
        jnb     end_while_factor     ; if !(factor*factor < guess)
        mov     eax,[Guess]
        mov     edx,0
        div     ebx                  ; edx = edx:eax % ebx
        cmp     edx, 0
        je      end_while_factor     ; if !(guess % factor != 0)

        add     ebx,2                ; factor += 2;
        jmp     while_factor
end_while_factor:
        je      end_if               ; if !(guess % factor != 0)
        mov     eax,[Guess]          ; printf("%u\n")
        call    print_int
        call    print_nl
end_if:
        add     dword [Guess], 2     ; guess += 2
        jmp     while_limit
end_while_limit:

        popa
        mov     eax, 0            ; return back to C
        leave                     
        ret
\end{AsmCodeListing}
\index{prime.asm|)}

% -*-latex-*-
\chapter{Bit Operations}
\section{Shift Operations\index{bit operations!shifts|(}}

Assembly language allows the programmer to manipulate the individual bits
of data. One common bit operation is called a \emph{shift}. A shift operation
moves the position of the bits of some data. Shifts can be either toward the
left (\emph{i.e.} toward the most significant bits) or toward the right
(the least significant bits).

\subsection{Logical shifts\index{bit operations!shifts!logical shifts|(}}

A logical shift is the simplest type of shift. It shifts in a very 
straightforward manner. Figure~\ref{fig:logshifts} shows an example of a
shifted single byte number.

\begin{figure}[h]
\centering
\begin{tabular}{l|c|c|c|c|c|c|c|c|}
\cline{2-9}
Original      & 1 & 1 & 1 & 0 & 1 & 0 & 1 & 0 \\
\cline{2-9}
Left shifted  & 1 & 1 & 0 & 1 & 0 & 1 & 0 & 0 \\
\cline{2-9}
Right shifted & 0 & 1 & 1 & 1 & 0 & 1 & 0 & 1 \\
\cline{2-9}
\end{tabular}
\caption{Logical shifts \label{fig:logshifts}}
\end{figure}

Note that new, incoming bits are always zero. The {\code SHL}
\index{SHL} and {\code SHR} \index{SHR} instructions are used to
perform logical left and right shifts respectively.  These
instructions allow one to shift by any number of positions. The number
of positions to shift can either be a constant or can be stored in the
{\code CL} register. The last bit shifted out of the data is stored in
the carry flag. Here are some code examples:
\begin{AsmCodeListing}[frame=none]
      mov    ax, 0C123H
      shl    ax, 1           ; shift 1 bit to left,   ax = 8246H, CF = 1
      shr    ax, 1           ; shift 1 bit to right,  ax = 4123H, CF = 0
      shr    ax, 1           ; shift 1 bit to right,  ax = 2091H, CF = 1
      mov    ax, 0C123H
      shl    ax, 2           ; shift 2 bits to left,  ax = 048CH, CF = 1
      mov    cl, 3
      shr    ax, cl          ; shift 3 bits to right, ax = 0091H, CF = 1
\end{AsmCodeListing}

\subsection{Use of shifts}

Fast multiplication and division are the most common uses of a shift
operations. Recall that in the decimal system, multiplication and
division by a power of ten are simple, just shift digits. The same is
true for powers of two in binary. For example, to double the binary
number $1011_2$ (or 11 in decimal), shift once to the left to get
$10110_2$ (or 22). The quotient of a division by a power of two is the
result of a right shift. To divide by just 2, use a single right
shift; to divide by 4 ($2^2$), shift right 2 places; to divide by 8
($2^3$), shift 3 places to the right, \emph{etc.} Shift instructions
are very basic and are \emph{much} faster than the corresponding
{\code MUL} \index{MUL} and {\code DIV} \index{DIV} instructions!

Actually, logical shifts can be used to multiply and divide unsigned
values. They do not work in general for signed values. Consider the
2-byte value FFFF (signed $-1$). If it is logically right shifted
once, the result is 7FFF which is $+32,767$! Another type of shift can
be used for signed values.  
\index{bit operations!shifts!logical shifts|)}

\subsection{Arithmetic shifts\index{bit operations!shifts!arithmetic shifts|(}}

These shifts are designed to allow signed numbers to be quickly multiplied
and divided by powers of 2. They insure that the sign bit is treated 
correctly.
\begin{description}
\item[SAL] \index{SAL} Shift Arithmetic Left - This instruction is just a synonym for
           {\code SHL}. It is assembled into the exactly the same machine
           code as {\code SHL}. As long as the sign bit is not changed by
           the shift, the result will be correct.
\item[SAR] \index{SAR} Shift Arithmetic Right - This is a new instruction that does
           not shift the sign bit (\emph{i.e.} the msb) of its operand. The
           other bits are shifted as normal except that the new bits that 
           enter from the left are copies of the sign bit (that is, if the 
           sign bit is 1, the new bits are also 1). Thus, if a byte is shifted
           with this instruction, only the lower 7 bits are shifted. As for
           the other shifts, the last bit shifted out is stored in the
           carry flag.
\end{description}

\begin{AsmCodeListing}[frame=none]
      mov    ax, 0C123H
      sal    ax, 1           ; ax = 8246H, CF = 1
      sal    ax, 1           ; ax = 048CH, CF = 1
      sar    ax, 2           ; ax = 0123H, CF = 0
\end{AsmCodeListing}
\index{bit operations!shifts!arithmetic shifts|)}

\subsection{Rotate shifts\index{bit operations!shifts!rotates|(}}

The rotate shift instructions work like logical shifts except that
bits lost off one end of the data are shifted in on the other
side. Thus, the data is treated as if it is a circular structure. The
two simplest rotate instructions are {\code ROL} \index{ROL} and
{\code ROR} \index{ROR} which make left and right rotations,
respectively. Just as for the other shifts, these shifts leave the a
copy of the last bit shifted around in the carry flag.
\begin{AsmCodeListing}[frame=none]
      mov    ax, 0C123H
      rol    ax, 1           ; ax = 8247H, CF = 1
      rol    ax, 1           ; ax = 048FH, CF = 1
      rol    ax, 1           ; ax = 091EH, CF = 0
      ror    ax, 2           ; ax = 8247H, CF = 1
      ror    ax, 1           ; ax = C123H, CF = 1
\end{AsmCodeListing}

There are two additional rotate instructions that shift the bits in
the data and the carry flag named {\code RCL} \index{RCL} and {\code
RCR}. \index{RCR} For example, if the {\code AX} register is rotated
with these instructions, the 17-bits made up of {\code AX} and the
carry flag are rotated.
\begin{AsmCodeListing}[frame=none]
      mov    ax, 0C123H
      clc                    ; clear the carry flag (CF = 0)
      rcl    ax, 1           ; ax = 8246H, CF = 1
      rcl    ax, 1           ; ax = 048DH, CF = 1
      rcl    ax, 1           ; ax = 091BH, CF = 0
      rcr    ax, 2           ; ax = 8246H, CF = 1
      rcr    ax, 1           ; ax = C123H, CF = 0
\end{AsmCodeListing}
\index{bit operations!shifts!rotates|)}

\subsection{Simple application\label{sect:AddBitsExample}}

Here is a code snippet that counts the number of bits that are ``on''
(\emph{i.e.}~1) in the EAX register.
%TODO: show how the ADC instruction could be used to remove the jnc
\begin{AsmCodeListing}
      mov    bl, 0           ; bl will contain the count of ON bits
      mov    ecx, 32         ; ecx is the loop counter
count_loop:
      shl    eax, 1          ; shift bit into carry flag
      jnc    skip_inc        ; if CF == 0, goto skip_inc
      inc    bl
skip_inc:
      loop   count_loop
\end{AsmCodeListing}
The above code destroys the original value of {\code EAX} ({\code EAX} is zero
at the end of the loop). If one wished to retain the value of {\code EAX},
line~4 could be replaced with {\code rol  eax, 1}.
\index{bit operations!shifts|)}

\section{Boolean Bitwise Operations}

There are four common boolean operators: \emph{AND}, \emph{OR}, \emph{XOR} and
\emph{NOT}. A \emph{truth table} shows the result of each operation for each
possible value of its operands.

\subsection{The \emph{AND} operation\index{bit operations!AND}}

\begin{table}[t]
\centering
\begin{tabular}{|c|c|c|}
\hline
\emph{X} & \emph{Y} & \emph{X} AND \emph{Y} \\
\hline \hline
0 & 0 & 0 \\
0 & 1 & 0 \\
1 & 0 & 0 \\
1 & 1 & 1 \\
\hline
\end{tabular}
\caption{The AND operation \label{tab:and} \index{AND}}
\end{table}

The result of the \emph{AND} of two bits is only 1 if both bits are 1, else
the result is 0 as the truth table in Table~\ref{tab:and} shows.

\begin{figure}[t]
\centering
\begin{tabular}{rcccccccc}
    & 1 & 0 & 1 & 0 & 1 & 0 & 1 & 0 \\
AND & 1 & 1 & 0 & 0 & 1 & 0 & 0 & 1 \\
\hline
    & 1 & 0 & 0 & 0 & 1 & 0 & 0 & 0
\end{tabular}
\caption{ANDing a byte \label{fig:and}}
\end{figure}

Processors support these operations as instructions that act 
independently on all the bits of data in parallel. For example, if the contents
of {\code AL} and {\code BL} are \emph{AND}ed together, the basic \emph{AND}
operation is applied to each of the 8 pairs of corresponding bits in the
two registers as Figure~\ref{fig:and} shows. Below is a code example:
\begin{AsmCodeListing}[frame=none]
      mov    ax, 0C123H
      and    ax, 82F6H          ; ax = 8022H
\end{AsmCodeListing}

\subsection{The \emph{OR} operation\index{bit operations!OR}}

\begin{table}[t]
\centering
\begin{tabular}{|c|c|c|}
\hline
\emph{X} & \emph{Y} & \emph{X} OR \emph{Y} \\
\hline \hline
0 & 0 & 0 \\
0 & 1 & 1 \\
1 & 0 & 1 \\
1 & 1 & 1 \\
\hline
\end{tabular}
\caption{The OR operation \label{tab:or} \index{OR}}
\end{table}


The inclusive \emph{OR} of 2 bits is 0 only if both bits are 0, else
the result is 1 as the truth table in Table~\ref{tab:or} shows. Below
is a code example:

\begin{AsmCodeListing}[frame=none]
      mov    ax, 0C123H
      or     ax, 0E831H          ; ax = E933H
\end{AsmCodeListing}

\subsection{The \emph{XOR} operation\index{bit operations!XOR}}

\begin{table}
\centering
\begin{tabular}{|c|c|c|}
\hline
\emph{X} & \emph{Y} & \emph{X} XOR \emph{Y} \\
\hline \hline
0 & 0 & 0 \\
0 & 1 & 1 \\
1 & 0 & 1 \\
1 & 1 & 0 \\
\hline
\end{tabular}
\caption{The XOR operation \label{tab:xor}\index{XOR}}
\end{table}


The exclusive \emph{OR} of 2 bits is 0 if and only if both bits
are equal, else the result is 1 as the truth table in
Table~\ref{tab:xor} shows. Below is a code example:

\begin{AsmCodeListing}[frame=none]
      mov    ax, 0C123H
      xor    ax, 0E831H          ; ax = 2912H
\end{AsmCodeListing}

\subsection{The \emph{NOT} operation\index{bit operations!NOT}}

\begin{table}[t]
\centering
\begin{tabular}{|c|c|}
\hline
\emph{X} & NOT \emph{X} \\
\hline \hline
0 & 1 \\
1 & 0 \\
\hline
\end{tabular}
\caption{The NOT operation \label{tab:not}\index{NOT}}
\end{table}

The \emph{NOT} operation is a \emph{unary} operation (\emph{i.e.} it
acts on one operand, not two like \emph{binary} operations such as
\emph{AND}).  The \emph{NOT} of a bit is the opposite value of the bit
as the truth table in Table~\ref{tab:not} shows. Below is a code
example:

\begin{AsmCodeListing}[frame=none]
      mov    ax, 0C123H
      not    ax                 ; ax = 3EDCH
\end{AsmCodeListing}

Note that the \emph{NOT} finds the one's complement. Unlike the other
bitwise operations, the {\code NOT} instruction does not change any of
the bits in the {\code FLAGS} register.

\subsection{The {\code TEST} instruction\index{TEST}}

The {\code TEST} instruction performs an \emph{AND} operation, but
does not store the result. It only sets the {\code FLAGS} register
based on what the result would be (much like how the {\code CMP}
instruction performs a subtraction but only sets {\code FLAGS}). For
example, if the result would be zero, {\code ZF} would be set.

\begin{table}
\begin{tabular}{lp{3in}}
Turn on bit \emph{i} & \emph{OR} the number with $2^i$ (which is
                              the binary number with just bit \emph{i} on) \\
Turn off bit \emph{i} & \emph{AND} the number with the binary number with
                              only bit \emph{i} off. This operand is often
                  	      called a \emph{mask} \\
Complement bit \emph{i} & \emph{XOR} the number with $2^i$
\end{tabular}
\caption{Uses of boolean operations \label{tab:bool}}
\end{table}

\subsection{Uses of bit operations\index{bit operations!assembly|(}}

Bit operations are very useful for manipulating individual bits of data
without modifying the other bits. Table~\ref{tab:bool} shows three common
uses of these operations. Below is some example code, implementing these
ideas.
\begin{AsmCodeListing}[frame=none]
      mov    ax, 0C123H
      or     ax, 8           ; turn on bit 3,   ax = C12BH
      and    ax, 0FFDFH      ; turn off bit 5,  ax = C10BH
      xor    ax, 8000H       ; invert bit 15,   ax = 410BH
      or     ax, 0F00H       ; turn on nibble,  ax = 4F0BH
      and    ax, 0FFF0H      ; turn off nibble, ax = 4F00H
      xor    ax, 0F00FH      ; invert nibbles,  ax = BF0FH
      xor    ax, 0FFFFH      ; 1's complement,  ax = 40F0H
\end{AsmCodeListing}

The \emph{AND} operation can also be used to find the remainder of a
division by a power of two. To find the remainder of a division by
$2^i$, \emph{AND} the number with a mask equal to $2^i - 1$. This mask will
contain ones from bit 0 up to bit $i-1$. It is just these bits that contain
the remainder. The result of the \emph{AND} will keep these bits and
zero out the others. Next is a snippet of code that finds the quotient and
remainder of the division of 100 by 16.
\begin{AsmCodeListing}[frame=none]
      mov    eax, 100        ; 100 = 64H
      mov    ebx, 0000000FH  ; mask = 16 - 1 = 15 or F
      and    ebx, eax        ; ebx = remainder = 4
      shr    eax, 4          ; eax = quotient of eax/2^4 = 6
\end{AsmCodeListing}
Using the {\code CL} register it is possible to modify arbitrary bits of data.
Next is an example that sets (turns on) an arbitrary bit in {\code EAX}. The
number of the bit to set is stored in {\code BH}.
\begin{AsmCodeListing}[frame=none]
      mov    cl, bh          ; first build the number to OR with
      mov    ebx, 1
      shl    ebx, cl         ; shift left cl times
      or     eax, ebx        ; turn on bit
\end{AsmCodeListing}
Turning a bit off is just a little harder.
\begin{AsmCodeListing}[frame=none]
      mov    cl, bh          ; first build the number to AND with
      mov    ebx, 1
      shl    ebx, cl         ; shift left cl times
      not    ebx             ; invert bits
      and    eax, ebx        ; turn off bit
\end{AsmCodeListing}
Code to complement an arbitrary bit is left as an exercise for the reader.

It is not uncommon to see the following puzzling instruction in a 80x86
program:
\begin{AsmCodeListing}[frame=none,numbers=none]
      xor    eax, eax         ; eax = 0
\end{AsmCodeListing}
A number \emph{XOR}'ed with itself always results in zero. This instruction
is used because its machine code is smaller than the corresponding 
{\code MOV} instruction.
\index{bit operations!assembly|)}

\begin{figure}[t]
\begin{AsmCodeListing}
      mov    bl, 0           ; bl will contain the count of ON bits
      mov    ecx, 32         ; ecx is the loop counter
count_loop:
      shl    eax, 1          ; shift bit into carry flag
      adc    bl, 0           ; add just the carry flag to bl
      loop   count_loop
\end{AsmCodeListing}
\caption{Counting bits with {\code ADC}\label{fig:countBitsAdc}}
\end{figure}

\section{Avoiding Conditional Branches}
\index{branch prediction|(} 

Modern processors use very sophisticated techniques to execute code as
quickly as possible. One common technique is known as
\emph{speculative execution}\index{speculative execution}. This
technique uses the parallel processing capabilities of the CPU to
execute multiple instructions at once. Conditional branches present a
problem with this idea. The processor, in general, does not know
whether the branch will be taken or not. If it is taken, a different
set of instructions will be executed than if it is not
taken. Processors try to predict whether the branch will be taken. If
the prediciton is wrong, the processor has wasted its time executing
the wrong code.

\index{branch prediction|)}

One way to avoid this problem is to avoid using conditional branches
when possible. The sample code in \ref{sect:AddBitsExample} provides a
simple example of where one could do this. In the previous example, the
``on'' bits of the EAX register are counted. It uses a branch to skip
the {\code INC} instruction. Figure~\ref{fig:countBitsAdc} shows how
the branch can be removed by using the {\code ADC}\index{ADC}
instruction to add the carry flag directly.

The {\code SET\emph{xx}}\index{SET\emph{xx}} instructions provide a
way to remove branches in certain cases. These instructions set the
value of a byte register or memory location to zero or one based on
the state of the FLAGS register.  The characters after {\code SET} are
the same characters used for conditional branches. If the
corresponding condition of the {\code SET\emph{xx}} is true, the result stored
is a one, if false a zero is stored. For example,
\begin{AsmCodeListing}[frame=none,numbers=none]
      setz   al        ; AL = 1 if Z flag is set, else 0
\end{AsmCodeListing}
Using these instructions, one can develop some clever techniques that
calculate values without branches.

For example, consider the problem of finding the maximum of two values.
The standard approach to solving this problem would be to use a {\code
CMP} and use a conditional branch to act on which value was larger. The
example program below shows how the maximum can be found without any 
branches.

\begin{AsmCodeListing}
; file: max.asm
%include "asm_io.inc"
segment .data

message1 db "Enter a number: ",0
message2 db "Enter another number: ", 0
message3 db "The larger number is: ", 0

segment .bss

input1  resd    1        ; first number entered

segment .text
        global  _asm_main
_asm_main:
        enter   0,0               ; setup routine
        pusha

        mov     eax, message1     ; print out first message
        call    print_string
        call    read_int          ; input first number
        mov     [input1], eax

        mov     eax, message2     ; print out second message
        call    print_string
        call    read_int          ; input second number (in eax)

        xor     ebx, ebx          ; ebx = 0
        cmp     eax, [input1]     ; compare second and first number
        setg    bl                ; ebx = (input2 > input1) ?          1 : 0
        neg     ebx               ; ebx = (input2 > input1) ? 0xFFFFFFFF : 0
        mov     ecx, ebx          ; ecx = (input2 > input1) ? 0xFFFFFFFF : 0
        and     ecx, eax          ; ecx = (input2 > input1) ?     input2 : 0
        not     ebx               ; ebx = (input2 > input1) ?          0 : 0xFFFFFFFF
        and     ebx, [input1]     ; ebx = (input2 > input1) ?          0 : input1
        or      ecx, ebx          ; ecx = (input2 > input1) ?     input2 : input1

        mov     eax, message3     ; print out result
        call    print_string
        mov     eax, ecx
        call    print_int
        call    print_nl

        popa
        mov     eax, 0            ; return back to C
        leave                     
        ret
\end{AsmCodeListing}

The trick is to create a bit mask that can be used to select the
correct value for the maximum. The {\code SETG}\index{SETG}
instruction in line~30 sets BL to 1 if the second input is the maximum
or 0 otherwise. This is not quite the bit mask desired. To create the
required bit mask, line~31 uses the {\code NEG}\index{NEG} instruction
on the entire EBX register. (Note that EBX was zeroed out earlier.)
If EBX is 0, this does nothing; however, if EBX is 1, the result is
the two's complement representation of -1 or 0xFFFFFFFF. This is just
the bit mask required. The remaining code uses this bit mask to select
the correct input as the maximum.

An alternative trick is to use the {\code DEC} statement. In the above
code, if the {\code NEG} is replaced with a {\code DEC}, again the result
will either be 0 or 0xFFFFFFFF. However, the values are reversed than
when using the {\code NEG} instruction.


\section{Manipulating bits in C\index{bit operations!C|(}}

\subsection{The bitwise operators of C}

Unlike some high-level languages, C does provide operators for bitwise
operations. The \emph{AND} operation is represented by the binary
{\code \&} operator\footnote{This operator is different from the
binary {\code \&\&} and unary {\code \&} operators!}. The \emph{OR}
operation is represented by the binary {\code |} operator. The
\emph{XOR} operation is represented by the binary {\code \verb|^| 
}operator. And the \emph{NOT} operation is represented by the unary
{\code \verb|~| }operator.

The shift operations are performed by C's {\code <<} and {\code >>}
binary operators. The {\code <<} operator performs left shifts and the 
{\code >>} operator performs right shifts. These operators take two
operands. The left operand is the value to shift and the right operand is
the number of bits to shift by. If the value to shift is an unsigned type,
a logical shift is made. If the value is a signed type (like {\code int}),
then an arithmetic shift is used. Below is some example C code using these
operators:
\begin{lstlisting}{}
short int s;          /* assume that short int is 16-bit */
short unsigned u;
s = -1;               /* s = 0xFFFF (2's complement) */
u = 100;              /* u = 0x0064 */
u = u | 0x0100;       /* u = 0x0164 */
s = s & 0xFFF0;       /* s = 0xFFF0 */
s = s ^ u;            /* s = 0xFE94 */
u = u << 3;           /* u = 0x0B20 (logical shift) */
s = s >> 2;           /* s = 0xFFA5 (arithmetic shift) */
\end{lstlisting}

\subsection{Using bitwise operators in C}

The bitwise operators are used in C for the same purposes as they are used
in assembly language. They allow one to manipulate individual bits of data
and can be used for fast multiplication and division. In fact, a smart C
compiler will use a shift for a multiplication like, {\code x *= 2}, 
automatically.
\begin{table}
\centering
\begin{tabular}{|c|l|}
\hline
Macro & \multicolumn{1}{c|}{Meaning} \\
\hline \hline
{\code S\_IRUSR} & user can read \\
{\code S\_IWUSR} & user can write \\
{\code S\_IXUSR} & user can execute \\
\hline
{\code S\_IRGRP} & group can read \\
{\code S\_IWGRP} & group can write \\
{\code S\_IXGRP} & group can execute \\
\hline
{\code S\_IROTH} & others can read \\
{\code S\_IWOTH} & others can write \\
{\code S\_IXOTH} & others can execute \\
\hline
\end{tabular}
\caption{POSIX File Permission Macros \label{tab:posix}}
\end{table}

Many operating system API\footnote{Application Programming
Interface}'s (such as \emph{POSIX}\footnote{stands for Portable
Operating System Interface for Computer Environments. A standard
developed by the IEEE based on UNIX.} and Win32) contain
functions which use operands that have data encoded as bits. For
example, POSIX systems maintain file permissions for three different
types of users: \emph{user} (a better name would be \emph{owner}),
\emph{group} and \emph{others}. Each type of user can be granted permission 
to read, write and/or execute a file. To change the permissions of a file
requires the C programmer to manipulate individual bits. POSIX defines
several macros to help (see Table~\ref{tab:posix}). The {\code chmod}
function can be used to set the permissions of file. This function
takes two parameters, a string with the name of the file to act on and
an integer\footnote{Actually a parameter of type {\code mode\_t} which
is a typedef to an integral type.} with the appropriate bits set for
the desired permissions. For example, the code below sets the
permissions to allow the owner of the file to read and write to it,
users in the group to read the file and others have no access.
\begin{lstlisting}[stepnumber=0]{}
chmod("foo", S_IRUSR | S_IWUSR | S_IRGRP );
\end{lstlisting}

The POSIX {\code stat} function can be used to find out the current 
permission bits for the file. Used with the {\code chmod} function, it
is possible to modify some of the permissions without changing others.
Here is an example that removes write access to others and adds read
access to the owner of the file. The other permissions are not altered.
\begin{lstlisting}{}
struct stat file_stats;    /* struct used by stat() */
stat("foo", &file_stats);  /* read file info. 
                              file_stats.st_mode holds permission bits */
chmod("foo", (file_stats.st_mode & ~S_IWOTH) | S_IRUSR);
\end{lstlisting}
\index{bit operations!C|)}

\section{Big and Little Endian Representations\index{endianess|(}}

Chapter~1 introduced the concept of big and little endian
representations of multibyte data. However, the author has found
that this subject confuses many people. This section covers the
topic in more detail. 

The reader will recall that endianness refers to the order that the
individual bytes (\emph{not} bits) of a multibyte data element is
stored in memory. Big endian is the most straightforward method. It
stores the most significant byte first, then the next significant byte
and so on. In other words the \emph{big} bits are stored first. Little
endian stores the bytes in the opposite order (least significant first).
The x86 family of processors use little endian representation.

As an example, consider the double word representing $12345678_{16}$. In
big endian representation, the bytes would be stored as 12~34~56~78. In
little endian represenation, the bytes would be stored as 78~56~34~12.

The reader is probably asking himself right now, why any sane chip
designer would use little endian representation? Were the engineers at
Intel sadists for inflicting this confusing representations on
multitudes of programmers? It would seem that the CPU has to do extra
work to store the bytes backward in memory like this (and to unreverse
them when read back in to memory). The answer is that the CPU does not
do any extra work to write and read memory using little endian format.
One has to realize that the CPU is composed of many electronic
circuits that simply work on bit values. The bits (and bytes) are not
in any necessary order in the CPU.

Consider the 2-byte {\code AX} register. It can be decomposed into the
single byte registers: {\code AH} and {\code AL}. There are circuits
in the CPU that maintain the values of {\code AH} and {\code
AL}. Circuits are not in any order in a CPU. That is, the circuits for
{\code AH} are not before or after the circuits for {\code AL}. A
{\code mov} instruction that copies the value of {\code AX} to memory
copies the value of {\code AL} then {\code AH}. This is not any harder
for the CPU to do than storing {\code AH} first.

\begin{figure}[t]
\begin{lstlisting}[stepnumber=0,frame=tblr]{}
  unsigned short word = 0x1234;   /* assumes sizeof(short) == 2 */
  unsigned char * p = (unsigned char *) &word;

  if ( p[0] == 0x12 )
    printf("Big Endian Machine\n");
  else
    printf("Little Endian Machine\n");
\end{lstlisting}
\caption{How to Determine Endianness \label{fig:determineEndian}}
\end{figure}

The same argument applies to the individual bits in a byte. They are
not really in any order in the circuits of the CPU (or memory for that
matter). However, since individual bits can not be addressed in the
CPU or memory, there is no way to know (or care about) what order they
seem to be kept internally by the CPU.

The C code in Figure~\ref{fig:determineEndian} shows how the
endianness of a CPU can be determined.  The \lstinline|p| pointer
treats the \lstinline|word| variable as a two element character
array. Thus, \lstinline|p[0]| evaluates to the first byte of
\lstinline|word| in memory which depends on the endianness of the CPU.

\subsection{When to Care About Little and Big Endian}

For typical programming, the endianness of the CPU is not
significant. The most common time that it is important is when binary
data is transferred between different computer systems. This is
usually either using some type of physical data media (such as a disk)
or a network. \MarginNote{With the advent of multibyte character sets,
like UNICODE\index{UNICODE}, endianness is important for even text data. UNICODE
supports either endianness and has a mechanism for specifying which
endianness is being used to represent the data.} Since ASCII data
is single byte, endianness is not an issue for it.

All internal TCP/IP headers store integers in big endian format
(called \emph{network byte order}). TCP/IP \index{TCP/IP}libraries provide C
functions for dealing with endianness issues in a portable way.  For
example, the \lstinline|htonl()| function converts a double word (or
long integer) from \emph{host} to \emph{network} format. The
\lstinline|ntohl()| function performs the opposite
transformation.\footnote{Actually, reversing the endianness of an
integer simply reverses the bytes; thus, converting from big to little
or little to big is the same operation. So both of these functions do
the same thing.} For a big endian system, the two functions just
return their input unchanged. This allows one to write network
programs that will compile and run correctly on any system
irrespective of its endianness. For more information, about endianness
and network programming see W. Richard Steven's excellent book, 
\emph{UNIX Network Programming}.

\begin{figure}[t]
\begin{lstlisting}[frame=tlrb]{}
unsigned invert_endian( unsigned x )
{
  unsigned invert;
  const unsigned char * xp = (const unsigned char *) &x;
  unsigned char * ip = (unsigned char *) & invert;

  ip[0] = xp[3];   /* reverse the individual bytes */
  ip[1] = xp[2];
  ip[2] = xp[1];
  ip[3] = xp[0];

  return invert;   /* return the bytes reversed */
}
\end{lstlisting}
\caption{invert\_endian Function \label{fig:invertEndian}\index{endianess!invert\_endian}}
\end{figure}

Figure~\ref{fig:invertEndian} shows a C function that inverts the
endianness of a double word. The 486 processor introduced a new
machine instruction named {\code BSWAP} \index{BSWAP} that reverses
the bytes of any 32-bit register. For example,
\begin{AsmCodeListing}[frame=none,numbers=none]
      bswap   edx          ; swap bytes of edx
\end{AsmCodeListing}
The instruction can not be used on 16-bit registers. However, the
{\code XCHG} \index{XCHG} instruction can be used to swap the bytes of
the 16-bit registers that can be decomposed into 8-bit registers. For
example:
\begin{AsmCodeListing}[frame=none,numbers=none]
      xchg    ah,al        ; swap bytes of ax
\end{AsmCodeListing}
\index{endianess|)}

\section{Counting Bits\index{counting bits|(}}

Earlier a straightforward technique was given for counting the number of bits
that are ``on'' in a double word. This section looks at other less direct
methods of doing this as an exercise using the bit operations discussed in
this chapter.


\begin{figure}[t]
\begin{lstlisting}[frame=tblr]{}
int count_bits( unsigned int data )
{
  int cnt = 0;

  while( data != 0 ) {
    data = data & (data - 1);
    cnt++;
  }
  return cnt;
}
\end{lstlisting}
\caption{Bit Counting: Method One \label{fig:meth1}}
\end{figure}

\subsection{Method one\index{counting bits!method one|(}}

The first method is very simple, but not obvious. Figure~\ref{fig:meth1} shows the code.

How does this method work? In every iteration of the loop, one bit is turned
off in {\code data}. When all the bits are off (\emph{i.e.} when {\code data}
is zero), the loop stops. The number of iterations required to make 
{\code data} zero is equal to the number of bits in the original value of
{\code data}.

Line~6 is where a bit of {\code data} is turned off. How does this work?
Consider the general form of the binary representation of {\code data} and
the rightmost 1 in this representation. By definition, every bit after this
1 must be zero. Now, what will be the binary representation of {\code data
- 1}? The bits to the left of the rightmost 1 will be the same as for
{\code data}, but at the point of the rightmost 1 the bits will be the 
complement of the original bits of {\code data}. For example:\\
\begin{tabular}{lcl}
{\code data}     & = & xxxxx10000 \\
{\code data - 1} & = & xxxxx01111
\end{tabular}\\
where the x's are the same for both numbers. When {\code data} is
\emph{AND}'ed with {\code data - 1}, the result will zero the rightmost
1 in {\code data} and leave all the other bits unchanged.

\begin{figure}[t]
\begin{lstlisting}[frame=tlrb]{}
static unsigned char byte_bit_count[256];  /* lookup table */

void initialize_count_bits()
{
  int cnt, i, data;

  for( i = 0; i < 256; i++ ) {
    cnt = 0;
    data = i;
    while( data != 0 ) {	/* method one */
      data = data & (data - 1);
      cnt++;
    }
    byte_bit_count[i] = cnt;
  }
}

int count_bits( unsigned int data )
{
  const unsigned char * byte = ( unsigned char *) & data;

  return byte_bit_count[byte[0]] + byte_bit_count[byte[1]] +
         byte_bit_count[byte[2]] + byte_bit_count[byte[3]];
}
\end{lstlisting}
\caption{Method Two \label{fig:meth2}}
\end{figure}
\index{counting bits!method one|)}

\subsection{Method two\index{counting bits!method two|(}}

A lookup table can also be used to count the bits of an arbitrary double
word. The straightforward approach would be to precompute the number of bits
for each double word and store this in an array. However, there are two
related problems with this approach. There are roughly \emph{4 billion}
double word values! This means that the array will be very big and that
initializing it will also be very time consuming. (In fact, unless one is 
going to actually use the array more than 4 billion times, more time will
be taken to initialize the array than it would require to just compute the
bit counts using method one!)

A more realistic method would precompute the bit counts for all possible
byte values and store these into an array. Then the double word can be
split up into four byte values. The bit counts of these four byte values
are looked up from the array and sumed to find the bit count of the 
original double word. Figure~\ref{fig:meth2} shows the to code implement
this approach.

The {\code initialize\_count\_bits} function must be called before the
first call to the {\code count\_bits} function. This function initializes
the global {\code byte\_bit\_count} array. The {\code count\_bits} function
looks at the {\code data} variable not as a double word, but as an array
of four bytes. The {\code dword} pointer acts as a pointer to this
four byte array. Thus, {\code dword[0]} is one of the bytes in {\code
data} (either the least significant or the most significant byte depending 
on if the hardware is little or big endian, respectively.) Of course, one
could use a construction like:
\begin{lstlisting}[stepnumber=0]{}
(data >> 24) & 0x000000FF
\end{lstlisting}
\noindent to find the most significant byte value and similar ones for the 
other bytes; however, these constructions will be slower than an array
reference.

One last point, a {\code for} loop could easily be used to compute the
sum on lines~22 and 23. But, a {\code for} loop would include the
overhead of initializing a loop index, comparing the index after each
iteration and incrementing the index. Computing the sum as the
explicit sum of four values will be faster. In fact, a smart compiler
would convert the {\code for} loop version to the explicit sum. This
process of reducing or eliminating loop iterations is a compiler
optimization technique known as \emph{loop unrolling}.
\index{counting bits!method two|)}

\subsection{Method three\index{counting bits!method three|(}}

\begin{figure}[t]
\begin{lstlisting}[frame=tlrb]{}
int count_bits(unsigned int x )
{
  static unsigned int mask[] = { 0x55555555,
                                 0x33333333,
                                 0x0F0F0F0F,
                                 0x00FF00FF,
                                 0x0000FFFF };
  int i;
  int shift;   /* number of positions to shift to right */

  for( i=0, shift=1; i < 5; i++, shift *= 2 )
    x = (x & mask[i]) + ( (x >> shift) & mask[i] );
  return x;
}
\end{lstlisting}
\caption{Method 3 \label{fig:method3}}
\end{figure}

There is yet another clever method of counting the bits that are on in
data. This method literally adds the one's and zero's of the data together.
This sum must equal the number of one's in the data. For example, consider
counting the one's in a byte stored in a variable named {\code data}. The
first step is to perform the following operation:
\begin{lstlisting}[stepnumber=0]{}
data = (data & 0x55) + ((data >> 1) & 0x55);
\end{lstlisting}
What does this do? The hex constant {\code 0x55} is $01010101$ in
binary. In the first operand of the addition, {\code data} is
\emph{AND}'ed with this, bits at the odd bit positions are pulled
out. The second operand {\code ((data >> 1) \& 0x55)} first moves all
the bits at the even positions to an odd position and uses the same
mask to pull out these same bits. Now, the first operand contains the
odd bits and the second operand the even bits of {\code data}. When
these two operands are added together, the even and odd bits of {\code
data} are added together.  For example, if {\code data} is
$10110011_2$, then:\\
\begin{tabular}{rcr|l|l|l|l|}
\cline{4-7}
{\code data \&} $01010101_2$          &    &   & 00 & 01 & 00 & 01 \\
+ {\code (data >> 1) \&} $01010101_2$ & or & + & 01 & 01 & 00 & 01 \\
\cline{1-1} \cline{3-7}
                                      &    &   & 01 & 10 & 00 & 10 \\
\cline{4-7}
\end{tabular}

The addition on the right shows the actual bits added together. The bits of
the byte are divided into four 2-bit fields to show that actually there are
four independent additions being performed. Since the most these sums can be is
two, there is no possibility that the sum will overflow its field and corrupt
one of the other field's sums.

Of course, the total number of bits have not been computed yet. However, the
same technique that was used above can be used to compute the total in a
series of similar steps. The next step would be:
\begin{lstlisting}[stepnumber=0]{}
data = (data & 0x33) + ((data >> 2) & 0x33);
\end{lstlisting}
Continuing the above example (remember that {\code data} now is
$01100010_2$):\\
\begin{tabular}{rcr|l|l|}
\cline{4-5}
{\code data \&} $00110011_2$          &    &   & 0010 & 0010 \\
+ {\code (data >> 2) \&} $00110011_2$ & or & + & 0001 & 0000 \\
\cline{1-1} \cline{3-5}
                                      &    &   & 0011 & 0010 \\
\cline{4-5}
\end{tabular}\\
Now there are two 4-bit fields to that are independently added. 

The next step is to add these two bit sums together to form the final
result:
\begin{lstlisting}[stepnumber=0]{}
data = (data & 0x0F) + ((data >> 4) & 0x0F);
\end{lstlisting} 

Using the example above (with {\code data} equal to $00110010_2$):\\
\begin{tabular}{rcrl}
{\code data \&} $00001111_2$          &    &   & 00000010 \\
+ {\code (data >> 4) \&} $00001111_2$ & or & + & 00000011 \\
\cline{1-1} \cline{3-4}
                                      &    &   & 00000101 \\
\end{tabular}\\
Now {\code data} is 5 which is the correct result. Figure~\ref{fig:method3}
shows an implementation of this method that counts the bits in a double word.
It uses a {\code for} loop to compute the sum. It would be faster to 
unroll the loop; however, the loop makes it clearer how the method
generalizes to different sizes of data.
\index{counting bits!method three|)}
\index{counting bits|)}

%-*- latex -*-
\chapter{Subprograms}

This chapter looks at using subprograms to make modular programs and to
interface with high level languages (like C). Functions and procedures are
high level language examples of subprograms.

The code that calls a subprogram and the subprogram itself must agree
on how data will be passed between them. These rules on how data will
be passed are called \emph{calling conventions}. \index{calling
convention} A large part of this chapter will deal with the standard C
calling conventions that can be used to interface assembly subprograms
with C programs. This (and other conventions) often pass the addresses
of data (\emph{i.e.} pointers) to allow the subprogram to access the
data in memory.

\section{Indirect Addressing\index{indirect addressing|(}}

Indirect addressing allows registers to act like pointer variables. To
indicate that a register is to be used indirectly as a pointer, it is
enclosed in square brackets ({\code []}). For example:
\begin{AsmCodeListing}[frame=none]
      mov    ax, [Data]     ; normal direct memory addressing of a word
      mov    ebx, Data      ; ebx = & Data
      mov    ax, [ebx]      ; ax = *ebx
\end{AsmCodeListing}
Because AX holds a word, line~3 reads a word starting at the address stored 
in EBX. If AX was replaced with AL, only a single byte would be read. It is
important to realize that registers do not have types like variables do in
C. What EBX is assumed to point to is completely determined by what
instructions are used. Furthermore, even the fact that EBX is a pointer is
completely determined by the what instructions are used. If EBX is used
incorrectly, often there will be no assembler error; however, the program
will not work correctly. This is one of the many reasons that assembly
programming is more error prone than high level programming.

All the 32-bit general purpose (EAX, EBX, ECX, EDX) and index (ESI, EDI)
registers can be used for indirect addressing. In general, the 16-bit 
and 8-bit registers can not be.
\index{indirect addressing|)}

\section{Simple Subprogram Example\index{subprogram|(}}

A subprogram is an independent unit of code that can be used from different
parts of a program. In other words, a subprogram is like a function in C. A
jump can be used to invoke the subprogram, but returning presents a problem.
If the subprogram is to be used by different parts of the program, it must
return back to the section of code that invoked it. Thus, the jump back from
the subprogram can not be hard coded to a label. The code below shows how this
could be done using the indirect form of the {\code JMP} instruction. This 
form of the instruction uses the value of a register to determine where to
jump to (thus, the register acts much like a \emph{function pointer} in C.)
Here is the first program from chapter~1 rewritten to use a subprogram.
\begin{AsmCodeListing}[label=sub1.asm]
; file: sub1.asm
; Subprogram example program
%include "asm_io.inc"

segment .data
prompt1 db    "Enter a number: ", 0       ; don't forget null terminator
prompt2 db    "Enter another number: ", 0
outmsg1 db    "You entered ", 0
outmsg2 db    " and ", 0
outmsg3 db    ", the sum of these is ", 0

segment .bss
input1  resd 1
input2  resd 1

segment .text
        global  _asm_main
_asm_main:
        enter   0,0               ; setup routine
        pusha

        mov     eax, prompt1      ; print out prompt
        call    print_string

        mov     ebx, input1       ; store address of input1 into ebx
        mov     ecx, ret1         ; store return address into ecx
        jmp     short get_int     ; read integer
ret1:
        mov     eax, prompt2      ; print out prompt
        call    print_string

        mov     ebx, input2
        mov     ecx, $ + 7        ; ecx = this address + 7
        jmp     short get_int

        mov     eax, [input1]     ; eax = dword at input1
        add     eax, [input2]     ; eax += dword at input2
        mov     ebx, eax          ; ebx = eax

        mov     eax, outmsg1
        call    print_string      ; print out first message
        mov     eax, [input1]     
        call    print_int         ; print out input1
        mov     eax, outmsg2
        call    print_string      ; print out second message
        mov     eax, [input2]
        call    print_int         ; print out input2
        mov     eax, outmsg3
        call    print_string      ; print out third message
        mov     eax, ebx
        call    print_int         ; print out sum (ebx)
        call    print_nl          ; print new-line

        popa
        mov     eax, 0            ; return back to C
        leave                     
        ret
; subprogram get_int
; Parameters:
;   ebx - address of dword to store integer into
;   ecx - address of instruction to return to
; Notes:
;   value of eax is destroyed
get_int:
        call    read_int
        mov     [ebx], eax         ; store input into memory
        jmp     ecx                ; jump back to caller
\end{AsmCodeListing}

The {\code get\_int} subprogram uses a simple, register-based calling
convention. It expects the EBX register to hold the address of the
DWORD to store the number input into and the ECX register to hold the
code address of the instruction to jump back to. In lines~25 to 28,
the {\code ret1} label is used to compute this return address. In
lines~32 to 34, the {\code \$} operator is used to compute the return
address. The {\code \$} operator returns the current address for the
line it appears on. The expression {\code \$ + 7} computes the address
of the {\code MOV} instruction on line~36.

Both of these return address computations are awkward. The first method
requires a label to be defined for each subprogram call. The second method
does not require a label, but does require careful thought. If a near jump
was used instead of a short jump, the number to add to {\code \$} would not
be 7! Fortunately, there is a much simpler way to invoke subprograms. This
method uses the \emph{stack}.

\section{The Stack\index{stack|(}}

Many CPUs have built-in support for a stack. A stack is a Last-In First-Out
(\emph{LIFO}) list. The stack is an area of memory that is organized in this
fashion. The {\code PUSH} instruction adds data to the stack and the
{\code POP} instruction removes data. The data removed is always the last
data added (that is why it is called a last-in first-out list).

The SS segment register specifies the segment that contains the stack (usually
this is the same segment data is stored into). The ESP register contains the
address of the data that would be removed from the stack. This data is said
to be at the \emph{top} of the stack. Data can only be added in double word
units. That is, one can not push a single byte on the stack.

The {\code PUSH} instruction inserts a double word\footnote{Actually
words can be pushed too, but in 32-bit protected mode, it is better to
work with only double words on the stack.} on the stack by subtracting
4 from ESP and then stores the double word at {\code [ESP]}. The
{\code POP} instruction reads the double word at {\code [ESP]} and
then adds 4 to ESP. The code below demonstrates how these instructions
work and assumes that ESP is initially {\code 1000H}.
\begin{AsmCodeListing}[frame=none]
      push   dword 1    ; 1 stored at 0FFCh, ESP = 0FFCh
      push   dword 2    ; 2 stored at 0FF8h, ESP = 0FF8h
      push   dword 3    ; 3 stored at 0FF4h, ESP = 0FF4h
      pop    eax        ; EAX = 3, ESP = 0FF8h
      pop    ebx        ; EBX = 2, ESP = 0FFCh
      pop    ecx        ; ECX = 1, ESP = 1000h
\end{AsmCodeListing}

The stack can be used as a convenient place to store data temporarily. It is
also used for making subprogram calls, passing parameters and local
variables.

The 80x86 also provides a {\code PUSHA} instruction that pushes the values
of EAX, EBX, ECX, EDX, ESI, EDI and EBP registers (not in this order). The
{\code POPA} instruction can be used to pop them all back off.
\index{stack|)}

\section{The CALL and RET Instructions\index{subprogram!calling|(}}
\index{CALL|(}
\index{RET|(}
The 80x86 provides two instructions that use the stack to make calling
subprograms quick and easy. The CALL instruction makes an
unconditional jump to a subprogram and \emph{pushes} the address of
the next instruction on the stack. The RET instruction
\emph{pops off} an address and jumps to that address. When using these
instructions, it is very important that one manage the stack correctly
so that the right number is popped off by the RET instruction!

The previous program can be rewritten to use these new instructions by 
changing lines~25 to 34 to be:
\begin{AsmCodeListing}[numbers=none]
      mov    ebx, input1
      call   get_int

      mov    ebx, input2
      call   get_int
\end{AsmCodeListing}
and change the subprogram {\code get\_int} to:
\begin{AsmCodeListing}[numbers=none]
get_int:
      call   read_int
      mov    [ebx], eax
      ret
\end{AsmCodeListing}

There are several advantages to CALL and RET:
\begin{itemize}
\item It is simpler!
\item It allows subprograms calls to be nested easily. Notice that
{\code get\_int} calls {\code read\_int}. This call pushes another address
on the stack. At the end of {\code read\_int}'s code is a RET that pops
off the return address and jumps back to {\code get\_int}'s code. Then when
{\code get\_int}'s RET is executed, it pops off the return address that 
jumps back to {\code asm\_main}. This works correctly because of the LIFO
property of the stack.
\end{itemize}

Remember it is \emph{very} important to pop off all data that is pushed
on the stack. For example, consider the following:
\begin{AsmCodeListing}[frame=none]
get_int:
      call   read_int
      mov    [ebx], eax
      push   eax
      ret                  ; pops off EAX value, not return address!!
\end{AsmCodeListing}
This code would not return correctly!
\index{RET|)}
\index{CALL|)}

\section{Calling Conventions\index{calling convention|(}}

When a subprogram is invoked, the calling code and the subprogram (the
\emph{callee}) must agree on how to pass data between them. High-level
languages have standard ways to pass data known as \emph{calling 
conventions}. For high-level code to interface with assembly language, the
assembly language code must use the same conventions as the high-level
language. The calling conventions can differ from compiler to compiler or
may vary depending on how the code is compiled (\emph{e.g.} if
optimizations are on or not). One universal convention is that the code will
be invoked with a {\code CALL} instruction and return via a {\code RET}.

All PC C compilers support one calling convention that will be
described in the rest of this chapter in stages. These conventions
allow one to create subprograms that are \emph{reentrant}. A reentrant
subprogram may be called at any point of a program safely (even inside
the subprogram itself).

\subsection{Passing parameters on the stack\index{stack|(}\index{stack!parameters|(}}

Parameters to a subprogram may be passed on the stack. They are pushed onto
the stack before the {\code CALL} instruction. Just as in C, if the
parameter is to be changed by the subprogram, the \emph{address} of the 
data must be passed, not the \emph{value}. If the parameter's size is less
than a double word, it must be converted to a double word before being pushed.

The parameters on the stack are not popped off by the subprogram, instead
they are accessed from the stack itself. Why?
\begin{itemize}
\item Since they have to be pushed on the stack before the {\code CALL}
instruction, the return address would have to be popped off first (and
then pushed back on again).
\item Often the parameters will have to be used in several places in the
subprogram. Usually, they can not be kept in a register for the entire
subprogram and would have to be stored in memory. Leaving them on the
stack keeps a copy of the data in memory that can be accessed at any
point of the subprogram.
\end{itemize}

\begin{figure}
\centering
\begin{tabular}{l|c|}
\cline{2-2}
&  \\ \cline{2-2}
ESP + 4 & Parameter \\ \cline{2-2}
ESP     & Return address \\ \cline{2-2}
 & \\ \cline{2-2}
\end{tabular}
\caption{}
\label{fig:stack1}
\end{figure}
Consider \MarginNote{When using indirect addressing, the 80x86 processor 
accesses different segments depending on what registers are used in the
indirect addressing expression. ESP (and EBP) use the stack segment while
EAX, EBX, ECX and EDX use the data seg\-ment. However, this is usually 
unimportant for most protected mode programs, because for them the data 
and stack segments are the same.}
a subprogram that is passed a single parameter on the stack. When
the subprogram is invoked, the stack looks like Figure~\ref{fig:stack1}.
The parameter can be accessed using indirect addressing ({\code [ESP+4]}
\footnote{It is legal to add a constant to a register when using indirect
addressing. More complicated expressions are possible too. This topic is covered
in the next chapter}).
\begin{figure}
\centering
\begin{tabular}{l|c|}
\cline{2-2}
&  \\ \cline{2-2}
ESP + 8 & Parameter \\ \cline{2-2}
ESP + 4 & Return address \\ \cline{2-2}
ESP     & subprogram data \\ \cline{2-2}
\end{tabular}
\caption{}
\label{fig:stack2}
\end{figure}

\begin{figure}[t]
\begin{AsmCodeListing}[frame=single]
subprogram_label:
      push   ebp           ; save original EBP value on stack
      mov    ebp, esp      ; new EBP = ESP
; subprogram code
      pop    ebp           ; restore original EBP value
      ret
\end{AsmCodeListing}
\caption{General subprogram form \label{fig:subskel1}}
\end{figure}

If the stack is also used inside the subprogram to store data, the
number needed to be added to ESP will change. For example,
Figure~\ref{fig:stack2} shows what the stack looks like if a DWORD is
pushed the stack. Now the parameter is at {\code ESP + 8} not {\code
ESP + 4}. Thus, it can be very error prone to use ESP when referencing
parameters. To solve this problem, the 80386 supplies another register
to use: EBP. This register's only purpose is to reference data on the
stack. The C calling convention mandates that a subprogram first save
the value of EBP on the stack and then set EBP to be equal to ESP.
This allows ESP to change as data is pushed or popped off the stack
without modifying EBP. At the end of the subprogram, the original
value of EBP must be restored (this is why it is saved at the start of
the subprogram.)  Figure~\ref{fig:subskel1} shows the general form of
a subprogram that follows these conventions.

\begin{figure}[t]
\centering
\begin{tabular}{ll|c|}
\cline{3-3}
&  & \\ \cline{3-3}
ESP + 8 & EBP + 8 & Parameter \\ \cline{3-3}
ESP + 4 & EBP + 4 & Return address \\ \cline{3-3}
ESP     & EBP     & saved EBP \\ \cline{3-3}
\end{tabular}
\caption{}
\label{fig:stack3}
\end{figure}


Lines 2 and 3 in Figure~\ref{fig:subskel1} make up the general \emph{prologue}
of a subprogram. Lines 5 and 6 make up the \emph{epilogue}. 
Figure~\ref{fig:stack3} shows what the stack looks like immediately
after the prologue. Now the parameter can be access with {\code [EBP + 8]}
at any place in the subprogram without worrying about what else has
been pushed onto the stack by the subprogram.

After the subprogram is over, the parameters that were pushed on the
stack must be removed. The C calling convention \index{calling
convention!C} specifies that the caller code must do this. Other
conventions are different. For example, the Pascal calling convention
\index{calling convention!Pascal} specifies that the subprogram must
remove the parameters.  (There is another form of the RET \index{RET}
instruction that makes this easy to do.) Some C compilers support this
convention too. The {\code pascal} keyword is used in the prototype
and definition of the function to tell the compiler to use this
convention. In fact, the {\code stdcall} convention \index{calling
convention!stdcall} that the MS Windows API C functions use also works
this way.  What is the advantage of this way? It is a little more
efficient than the C convention. Why do all C functions not use this
convention, then? In general, C allows a function to have varying
number of arguments (\emph{e.g.}, the {\code printf} and {\code scanf}
functions). For these types of functions, the operation to remove the
parameters from the stack will vary from one call of the function to
the next. The C convention allows the instructions to perform this
operation to be easily varied from one call to the next. The Pascal
and stdcall convention makes this operation very difficult. Thus, the
Pascal convention (like the Pascal language) does not allow this type
of function. MS Windows can use this convention since none of its API
functions take varying numbers of arguments.

\begin{figure}[t]
\begin{AsmCodeListing}[frame=single]
      push   dword 1        ; pass 1 as parameter
      call   fun
      add    esp, 4         ; remove parameter from stack
\end{AsmCodeListing}
\caption{Sample subprogram call \label{fig:subcall}}
\end{figure}

Figure~\ref{fig:subcall} shows how a subprogram using the C calling
convention would be called. Line~3 removes the parameter from the
stack by directly manipulating the stack pointer. A {\code POP}
instruction could be used to do this also, but would require the
useless result to be stored in a register. Actually, for this
particular case, many compilers would use a {\code POP ECX}
instruction to remove the parameter. The compiler would use a {\code
POP} instead of an {\code ADD} because the {\code ADD} requires more
bytes for the instruction. However, the {\code POP} also changes ECX's
value! Next is another example program with two subprograms that use
the C calling conventions discussed above. Line~54 (and other lines)
shows that multiple data and text segments may be declared in a single
source file. They will be combined into single data and text segments
in the linking process. Splitting up the data and code into separate
segments allow the data that a subprogram uses to be defined close by
the code of the subprogram.
\index{stack!parameters|)}

\begin{AsmCodeListing}[label=sub3.asm]
%include "asm_io.inc"

segment .data
sum     dd   0

segment .bss
input   resd 1

;
; pseudo-code algorithm
; i = 1;
; sum = 0;
; while( get_int(i, &input), input != 0 ) {
;   sum += input;
;   i++;
; }
; print_sum(num);
segment .text
        global  _asm_main
_asm_main:
        enter   0,0               ; setup routine
        pusha

        mov     edx, 1            ; edx is 'i' in pseudo-code
while_loop:
        push    edx               ; save i on stack
        push    dword input       ; push address of input on stack
        call    get_int
        add     esp, 8            ; remove i and &input from stack

        mov     eax, [input]
        cmp     eax, 0
        je      end_while

        add     [sum], eax        ; sum += input

        inc     edx
        jmp     short while_loop

end_while:
        push    dword [sum]       ; push value of sum onto stack
        call    print_sum
        pop     ecx               ; remove [sum] from stack

        popa
        leave                     
        ret

; subprogram get_int
; Parameters (in order pushed on stack)
;   number of input (at [ebp + 12])
;   address of word to store input into (at [ebp + 8])
; Notes:
;   values of eax and ebx are destroyed
segment .data
prompt  db      ") Enter an integer number (0 to quit): ", 0

segment .text
get_int:
        push    ebp
        mov     ebp, esp

        mov     eax, [ebp + 12]
        call    print_int

        mov     eax, prompt
        call    print_string
        
        call    read_int
        mov     ebx, [ebp + 8]
        mov     [ebx], eax         ; store input into memory

        pop     ebp
        ret                        ; jump back to caller

; subprogram print_sum
; prints out the sum
; Parameter:
;   sum to print out (at [ebp+8])
; Note: destroys value of eax
;
segment .data
result  db      "The sum is ", 0

segment .text
print_sum:
        push    ebp
        mov     ebp, esp

        mov     eax, result
        call    print_string

        mov     eax, [ebp+8]
        call    print_int
        call    print_nl

        pop     ebp
        ret
\end{AsmCodeListing}


\subsection{Local variables on the stack\index{stack!local variables|(}}

The stack can be used as a convenient location for local variables. This is
exactly where C stores normal (or \emph{automatic} in C lingo) variables.
Using the stack for variables is important if one wishes subprograms to be
reentrant. A reentrant subprogram will work if it is invoked at any place,
including the subprogram itself. In other words, reentrant subprograms
can be invoked \emph{recursively}. Using the stack for variables also saves
memory. Data not stored on the stack is using memory from the beginning of
the program until the end of the program (C calls these types of variables
\emph{global} or \emph{static}). Data stored on the stack only use memory
when the subprogram they are defined for is active.

\begin{figure}[t]
\begin{AsmCodeListing}[frame=single]
subprogram_label:
      push   ebp                ; save original EBP value on stack
      mov    ebp, esp           ; new EBP = ESP
      sub    esp, LOCAL_BYTES   ; = # bytes needed by locals
; subprogram code
      mov    esp, ebp           ; deallocate locals
      pop    ebp                ; restore original EBP value
      ret
\end{AsmCodeListing}
\caption{General subprogram form with local variables\label{fig:subskel2}}
\end{figure}

\begin{figure}[t]
\begin{lstlisting}[frame=tlrb]{}
void calc_sum( int n, int * sump )
{
  int i, sum = 0;

  for( i=1; i <= n; i++ )
    sum += i;
  *sump = sum;
}
\end{lstlisting}
\caption{C version of sum \label{fig:Csum}}
\end{figure}

\begin{figure}[t]
\begin{AsmCodeListing}[frame=single]
cal_sum:
      push   ebp
      mov    ebp, esp
      sub    esp, 4               ; make room for local sum

      mov    dword [ebp - 4], 0   ; sum = 0
      mov    ebx, 1               ; ebx (i) = 1
for_loop:
      cmp    ebx, [ebp+8]         ; is i <= n?
      jnle   end_for

      add    [ebp-4], ebx         ; sum += i
      inc    ebx
      jmp    short for_loop

end_for:
      mov    ebx, [ebp+12]        ; ebx = sump
      mov    eax, [ebp-4]         ; eax = sum
      mov    [ebx], eax           ; *sump = sum;

      mov    esp, ebp
      pop    ebp
      ret
\end{AsmCodeListing}
\caption{Assembly version of sum\label{fig:Asmsum}}
\end{figure}

Local variables are stored right after the saved EBP value in the stack.
They are allocated by subtracting the number of bytes required from ESP
in the prologue of the subprogram. Figure~\ref{fig:subskel2} shows the 
new subprogram skeleton. The EBP register is used to access local variables.
Consider the C function in Figure~\ref{fig:Csum}. Figure~\ref{fig:Asmsum}
shows how the equivalent subprogram could be written in assembly.

\begin{figure}[t]
\centering
\begin{tabular}{ll|c|}
\cline{3-3}
ESP + 16 & EBP + 12 & {\code sump} \\ \cline{3-3}
ESP + 12 & EBP + 8  & {\code n} \\ \cline{3-3}
ESP + 8  & EBP + 4  & Return address \\ \cline{3-3}
ESP + 4  & EBP      & saved EBP \\ \cline{3-3}
ESP      & EBP - 4  & {\code sum} \\ \cline{3-3}
\end{tabular}
\caption{}
\label{fig:SumStack}
\end{figure}

Figure~\ref{fig:SumStack} shows what the stack looks like after the
prologue of the program in Figure~\ref{fig:Asmsum}. This section of
the stack that contains the parameters, return information and local
variable storage is called a \emph{stack frame}. Every invocation of
a C function creates a new stack frame on the stack.

\begin{figure}[t]
\begin{AsmCodeListing}[frame=single]
subprogram_label:
      enter  LOCAL_BYTES, 0     ; = # bytes needed by locals
; subprogram code
      leave
      ret
\end{AsmCodeListing}
\caption{General subprogram form with local variables using 
{\code ENTER} and {\code LEAVE}\label{fig:subskel3}}
\end{figure}

\MarginNote{Despite the fact that {\code ENTER} and {\code LEAVE} simplify
the prologue and epilogue they are not used very often. Why? Because
they are slower than the equivalent simpler instructions! This is an
example of when one can not assume that a one instruction sequence is
faster than a multiple instruction one.} 
The prologue and epilogue of a subprogram can be simplified by using
two special instructions that are designed specifically for this
purpose. The {\code ENTER} instruction performs the prologue code and the
{\code LEAVE} performs the epilogue. The {\code ENTER} instruction
takes two immediate operands. For the C calling convention, the second
operand is always 0. The first operand is the number of bytes needed by
local variables. The {\code LEAVE} instruction has no
operands. Figure~\ref{fig:subskel3} shows how these instructions are
used. Note that the program skeleton (Figure~\ref{fig:skel}) also uses
{\code ENTER} and {\code LEAVE}.
\index{stack!local variables|)}
\index{stack|)}
\index{calling convention|)}
\index{subprogram!calling|)}

\section{Multi-Module Programs\index{multi-module programs|(}}

A \emph{multi-module program} is one composed of more than one object
file.  All the programs presented here have been multi-module
programs. They consisted of the C driver object file and the assembly
object file (plus the C library object files). Recall that the linker
combines the object files into a single executable program. The linker
must match up references made to each label in one module (\emph{i.e.}
object file) to its definition in another module. In order for module
A to use a label defined in module B, the {\code extern} directive
must be used. After the {\code extern} \index{directive!extern}
directive comes a comma delimited list of labels. The directive tells
the assembler to treat these labels as \emph{external} to the
module. That is, these are labels that can be used in this module, but
are defined in another. The {\code asm\_io.inc} file defines the
{\code read\_int}, \emph{etc.} routines as external.

In assembly, labels can not be accessed externally by default. If a
label can be accessed from other modules than the one it is defined
in, it must be declared \emph{global} in its module. The {\code
global} \index{directive!global} directive does this. Line~13 of the
skeleton program listing in Figure~\ref{fig:skel} shows the {\code
\_asm\_main} label being defined as global. Without this declaration,
there would be a linker error. Why? Because the C code would not be
able to refer to the \emph{internal} {\code \_asm\_main} label.

Next is the code for the previous example, rewritten to use two modules. The
two subprograms ({\code get\_int} and {\code print\_sum}) are in a separate
source file than the {\code \_asm\_main} routine.

\begin{AsmCodeListing}[label=main4.asm,commandchars=\\\{\}]
%include "asm_io.inc"

segment .data
sum     dd   0

segment .bss
input   resd 1

segment .text
        global  _asm_main
\textit{        extern  get_int, print_sum}
_asm_main:
        enter   0,0               ; setup routine
        pusha

        mov     edx, 1            ; edx is 'i' in pseudo-code
while_loop:
        push    edx               ; save i on stack
        push    dword input       ; push address on input on stack
        call    get_int
        add     esp, 8            ; remove i and &input from stack

        mov     eax, [input]
        cmp     eax, 0
        je      end_while

        add     [sum], eax        ; sum += input

        inc     edx
        jmp     short while_loop

end_while:
        push    dword [sum]       ; push value of sum onto stack
        call    print_sum
        pop     ecx               ; remove [sum] from stack

        popa
        leave                     
        ret
\end{AsmCodeListing}

\begin{AsmCodeListing}[label=sub4.asm,commandchars=\\\{\}]
%include "asm_io.inc"

segment .data
prompt  db      ") Enter an integer number (0 to quit): ", 0

segment .text
\textit{        global  get_int, print_sum}
get_int:
        enter   0,0

        mov     eax, [ebp + 12]
        call    print_int

        mov     eax, prompt
        call    print_string
        
        call    read_int
        mov     ebx, [ebp + 8]
        mov     [ebx], eax         ; store input into memory

        leave
        ret                        ; jump back to caller

segment .data
result  db      "The sum is ", 0

segment .text
print_sum:
        enter   0,0

        mov     eax, result
        call    print_string

        mov     eax, [ebp+8]
        call    print_int
        call    print_nl

        leave
        ret
\end{AsmCodeListing}

The previous example only has global \index{directive!global} code
labels; however, global data labels work exactly the same way.
\index{multi-module programs|)}

\section{Interfacing Assembly with C\index{interfacing with C|(}\index{calling convention!C|(}}

Today, very few programs are written completely in assembly. Compilers are
very good at converting high level code into efficient machine code. Since
it is much easier to write code in a high level language, it is more popular.
In addition, high level code is \emph{much} more portable than assembly!

When assembly is used, it is often only used for small parts of the code.
This can be done in two ways: calling assembly subroutines from C or inline
assembly. Inline assembly allows the programmer to place assembly statements
directly into C code. This can be very convenient; however, there are 
disadvantages to inline assembly. The assembly code must be written
in the format the compiler uses. No compiler at the moment supports NASM's
format. Different compilers require different formats. Borland and Microsoft
require MASM format. DJGPP and Linux's gcc require GAS\footnote{GAS is the
assembler that all GNU compiler's use. It uses the AT\&T syntax which is
very different from the relatively similar syntaxes of MASM, TASM and NASM.}
format. The technique of calling an assembly subroutine is much more
standardized on the PC.

Assembly routines are usually used with C for the following reasons:
\begin{itemize}
\item Direct access is needed to hardware features of the computer that
      are difficult or impossible to access from C.
\item The routine must be as fast as possible and the programmer can
      hand optimize the code better than the compiler can.
\end{itemize}

The last reason is not as valid as it once was. Compiler technology has
improved over the years and compilers can often generate very efficient code
(especially if compiler optimizations are turned on). The disadvantages of
assembly routines are: reduced portability and readability.

Most of the C calling conventions have already been specified. However, there
are a few additional features that need to be described.

\subsection{Saving registers\index{calling convention!C!registers|(}}
First, 
\MarginNote{The {\code register} keyword can be used in a C variable
declaration to suggest to the compiler that it use a register for this
variable instead of a memory location. These are known as register
variables. Modern compilers do this automatically without requiring any
suggestions.}
C assumes that a subroutine maintains the values of the
following registers: EBX, ESI, EDI, EBP, CS, DS, SS, ES. This does not
mean that the subroutine can not change them internally. Instead, it
means that if it does change their values, it must restore their 
original values before the subroutine returns. The EBX, ESI and EDI values
must be unmodified because C uses these registers for \emph{register
variables}. Usually the stack is used to save the original values of these
registers.

\begin{figure}[t]
\begin{AsmCodeListing}[frame=single]
segment .data
x            dd     0
format       db     "x = %d\n", 0

segment .text
...
      push   dword [x]     ; push x's value
      push   dword format  ; push address of format string
      call   _printf       ; note underscore!
      add    esp, 8        ; remove parameters from stack
\end{AsmCodeListing}
\caption{Call to {\code printf} \label{fig:Cprintf}}
\end{figure}

\begin{figure}[t]
\centering
\begin{tabular}{l|c|}
\cline{2-2}
EBP + 12 & value of {\code x} \\ \cline{2-2}
EBP + 8  & address of format string \\ \cline{2-2}
EBP + 4  & Return address \\ \cline{2-2}
EBP      & saved EBP \\ \cline{2-2}
\end{tabular}
\caption{Stack inside {\code printf}\label{fig:CprintfStack}}
\end{figure}
\index{calling convention!C!registers|)}

\subsection{Labels of functions\index{calling convention!C!labels|(}}
Most C compilers prepend a single underscore({\code \_}) character at
the beginning of the names of functions and global/static
variables. For example, a function named {\code f} will be assigned
the label {\code \_f}. Thus, if this is to be an assembly routine, it
\emph{must} be labelled {\code \_f}, not {\code f}. The Linux gcc
compiler does \emph{not} prepend any character.  Under Linux ELF
executables, one simply would use the label {\code f} for the C
function {\code f}.  However, DJGPP's gcc does prepend an
underscore. Note that in the assembly skeleton program
(Figure~\ref{fig:skel}), the label for the main routine is {\code
\_asm\_main}.
\index{calling convention!C!labels|)}

\subsection{Passing parameters\index{calling convention!C!parameters|(}}
Under the C calling convention, the arguments of a function are pushed on
the stack in the \emph{reverse} order that they appear in the function
call.

Consider the following C statement: \verb|printf("x = %d\n",x);|
Figure~\ref{fig:Cprintf} shows how this would be compiled (shown in
the equivalent NASM format). Figure~\ref{fig:CprintfStack} shows what
the stack looks like after the prologue inside the {\code printf}
function. The {\code printf} function is one of the C library
functions that can take any number of arguments. The rules of the C
calling conventions were specifically written to allow these types of
functions. \MarginNote{It is not necessary to use assembly to process
an arbitrary number of arguments in C. The {\code stdarg.h} header
file defines macros that can be used to process them portably. See any
good C book for details.} Since the address of the format string is
pushed last, its location on the stack will \emph{always} be at
{\code EBP + 8} no matter how many parameters are passed to the
function. The {\code printf} code can then look at the format string
to determine how many parameters should have been passed and look for
them on the stack.

Of course, if a mistake is made, \verb|printf("x = %d\n")|, the
{\code printf} code will still print out the double word value at 
{\code [EBP + 12]}. However, this will not be {\code x}'s value!
\index{calling convention!C!parameters|)}

\subsection{Calculating addresses of local variables\index{stack!local variables|(}}

Finding the address of a label defined in the {\code data} or {\code
bss} segments is simple. Basically, the linker does this. However,
calculating the address of a local variable (or parameter) on the
stack is not as straightforward. However, this is a very common need
when calling subroutines. Consider the case of passing the address of
a variable (let's call it {\code x}) to a function (let's call it
{\code foo}).  If {\code x} is located at EBP $-$ 8 on the stack, one
cannot just use:
\begin{AsmCodeListing}[numbers=none,frame=none]
      mov    eax, ebp - 8
\end{AsmCodeListing}
Why? The value that {\code MOV} stores into EAX must be computed by
the assembler (that is, it must in the end be a constant). However,
there is an instruction that does the desired calculation. It is
called \index{LEA|(} {\code LEA}  (for \emph{Load Effective Address}). The following
would calculate the address of {\code x} and store it into EAX:
\begin{AsmCodeListing}[numbers=none,frame=none]
      lea    eax, [ebp - 8]
\end{AsmCodeListing}
Now EAX holds the address of {\code x} and could be pushed on the
stack when calling function {\code foo}. Do not be confused, it looks
like this instruction is reading the data at
[EBP\nolinebreak$-$\nolinebreak8]; however, this is \emph{not}
true. The {\code LEA} instruction \emph{never} reads memory! It only
computes the address that would be read by another instruction and
stores this address in its first register operand. Since it does not
actually read any memory, no memory size designation (\emph{e.g.}
{\code dword}) is needed or allowed.

\index{LEA|)}
\index{stack!local variables|)}

\subsection{Returning values\index{calling convention!C!return values|(}}

Non-void C functions return back a value. The C calling conventions
specify how this is done. Return values are passed via registers. All
integral types ({\code char}, {\code int}, {\code enum}, \emph{etc.})
are returned in the EAX register. If they are smaller than 32-bits,
they are extended to 32-bits when stored in EAX. (How they are
extended depends on if they are signed or unsigned types.) 64-bit values
are returned in the EDX:EAX\index{register!EDX:EAX} register pair. Pointer
values are also stored in EAX. Floating point values are stored in the
ST0 register of the math coprocessor. (This register is discussed in
the floating point chapter.)
\index{calling convention!C!return values|)}
\index{calling convention!C|)}

\subsection{Other calling conventions\index{calling convention|(}}

The rules above describe the standard C calling convention that is
supported by all 80x86 C compilers. Often compilers support other
calling conventions as well. When interfacing with assembly language
it is \emph{very} important to know what calling convention the
compiler is using when it calls your function. Usually, the default is
to use the standard calling convention; however, this is not always
the case\footnote{The Watcom C\index{compiler!Watcom} compiler is an
example of one that does \emph{not} use the standard convention by
default. See the example source code file for Watcom for details}.
Compilers that use multiple conventions often have command line
switches that can be used to change the default convention.  They also
provide extensions to the C syntax to explicitly assign calling
conventions to individual functions. However, these extensions are not
standardized and may vary from one compiler to another.

The GCC compiler allows different calling conventions. The convention
of a function can be explicitly declared by using the {\code
\_\_attribute\_\_} extension\index{compiler!gcc!\_\_attribute\_\_}. For example,
to declare a void function that uses the standard calling convention
\index{calling convention!C} named {\code f} that takes a single
{\code int} parameter, use the following syntax for its prototype:
\begin{lstlisting}[stepnumber=0]{}
void f( int ) __attribute__((cdecl));
\end{lstlisting}
GCC also supports the \emph{standard call} \index{calling
convention!stdcall} calling convention. The function above could be
declared to use this convention by replacing the {\code cdecl} with
{\code stdcall}. The difference in {\code stdcall} and {\code cdecl}
is that {\code stdcall} requires the subroutine to remove the
parameters from the stack (as the Pascal calling convention
does). Thus, the {\code stdcall} convention can only be used with
functions that take a fixed number of arguments (\emph{i.e.} ones not
like {\code printf} and {\code scanf}).

GCC also supports an additional attribute called {\code regparm}
\index{calling convention!register} that tells the compiler to use
registers to pass up to 3 integer arguments to a function instead of
using the stack. This is a common type of optimization that many
compilers support.

Borland and Microsoft use a common syntax to declare calling
conventions.  They add the {\code \_\_cdecl}\index{calling
convention!\_\_cdecl} and {\code \_\_stdcall}\index{calling
convention!\_\_stdcall} keywords to C. These keywords act as function
modifiers and appear immediately before the function name in a
prototype. For example, the function {\code f} above would be defined
as follows for Borland and Microsoft:
\begin{lstlisting}[stepnumber=0]{}
void __cdecl f( int );
\end{lstlisting}

There are advantages and disadvantages to each of the calling
conventions.  The main advantages of the {\code cdecl}\index{calling
convention!C} convention are that it is simple and very flexible. It
can be used for any type of C function and C compiler. Using other
conventions can limit the portability of the subroutine. Its main
disadvantage is that it can be slower than some of the others and use
more memory (since every invocation of the function requires code to
remove the parameters on the stack).

The advantage of the {\code stdcall}\index{calling
convention!standard call} convention is that it uses less memory than
{\code cdecl}. No stack cleanup is required after the {\code CALL}
instruction. Its main disadvantage is that it can not be used with
functions that have variable numbers of arguments.

The advantage of using a convention that uses registers to pass integer
parameters is speed. The main disadvantage is that the convention is more
complex. Some parameters may be in registers and others on the stack.

\index{calling convention|)}

\subsection{Examples}

Next is an example that shows how an assembly routine can be interfaced to
a C program. (Note that this program does not use the assembly skeleton
program (Figure~\ref{fig:skel}) or the driver.c module.)

\LabelLine{main5.c}
\begin{lstlisting}{}
#include <stdio.h>
/* prototype for assembly routine */
void calc_sum( int, int * ) __attribute__((cdecl));

int main( void )
{
  int n, sum;

  printf("Sum integers up to: ");
  scanf("%d", &n);
  calc_sum(n, &sum);
  printf("Sum is %d\n", sum);
  return 0;
}
\end{lstlisting}
\LabelLine{main5.c}

\begin{AsmCodeListing}[label=sub5.asm, commandchars=\\\%|]
; subroutine _calc_sum
; finds the sum of the integers 1 through n
; Parameters:
;   n    - what to sum up to (at [ebp + 8])
;   sump - pointer to int to store sum into (at [ebp + 12])
; pseudo C code:
; void calc_sum( int n, int * sump )
; {
;   int i, sum = 0;
;   for( i=1; i <= n; i++ )
;     sum += i;
;   *sump = sum;
; }

segment .text
        global  _calc_sum
;
; local variable:
;   sum at [ebp-4]
_calc_sum:
        enter   4,0               ; make room for sum on stack
        push    ebx               ; IMPORTANT! \label%line:pushebx|

        mov     dword [ebp-4],0   ; sum = 0
        dump_stack 1, 2, 4        ; print out stack from ebp-8 to ebp+16 \label%line:dumpstack|
        mov     ecx, 1            ; ecx is i in pseudocode
for_loop:
        cmp     ecx, [ebp+8]      ; cmp i and n
        jnle    end_for           ; if not i <= n, quit

        add     [ebp-4], ecx      ; sum += i
        inc     ecx
        jmp     short for_loop

end_for:
        mov     ebx, [ebp+12]     ; ebx = sump
        mov     eax, [ebp-4]      ; eax = sum
        mov     [ebx], eax

        pop     ebx               ; restore ebx
        leave
        ret
\end{AsmCodeListing}

\begin{figure}[t]
\begin{Verbatim}[frame=single]
Sum integers up to: 10
Stack Dump # 1
EBP = BFFFFB70 ESP = BFFFFB68
 +16  BFFFFB80  080499EC
 +12  BFFFFB7C  BFFFFB80
  +8  BFFFFB78  0000000A
  +4  BFFFFB74  08048501
  +0  BFFFFB70  BFFFFB88
  -4  BFFFFB6C  00000000
  -8  BFFFFB68  4010648C
Sum is 55
\end{Verbatim}
\caption{Sample run of sub5 program \label{fig:dumpstack}}
\end{figure}

Why is line~\ref{line:pushebx} of {\code sub5.asm} so important?
Because the C calling convention requires the value of EBX to be
unmodified by the function call. If this is not done, it is very
likely that the program will not work correctly.

Line~\ref{line:dumpstack} demonstrates how the {\code dump\_stack} macro
works. Recall that the first parameter is just a numeric label, and the
second and third parameters determine how many double words to display below
and above EBP respectively. Figure~\ref{fig:dumpstack} shows an example run
of the program. For this dump, one can see that the address of the dword
to store the sum is BFFFFB80 (at EBP~+~12); the number to sum up to is 0000000A
(at EBP~+~8); the return address for the routine is 08048501 (at EBP~+~4);
the saved EBP value is BFFFFB88 (at EBP); the value of the local variable is
0 at (EBP~-~4); and finally the saved EBX value is 4010648C (at EBP~-~8).

The {\code calc\_sum} function could be rewritten to return the sum as its
return value instead of using a pointer parameter. Since the sum is an
integral value, the sum should be left in the EAX register. Line~11 of the
{\code main5.c} file would be changed to:
\begin{lstlisting}[stepnumber=0]{}
  sum = calc_sum(n);
\end{lstlisting}
Also, the prototype of {\code calc\_sum} would need be altered. Below is
the modified assembly code:
\begin{AsmCodeListing}[label=sub6.asm]
; subroutine _calc_sum
; finds the sum of the integers 1 through n
; Parameters:
;   n    - what to sum up to (at [ebp + 8])
; Return value:
;   value of sum
; pseudo C code:
; int calc_sum( int n )
; {
;   int i, sum = 0;
;   for( i=1; i <= n; i++ )
;     sum += i;
;   return sum;
; }
segment .text
        global  _calc_sum
;
; local variable:
;   sum at [ebp-4]
_calc_sum:
        enter   4,0               ; make room for sum on stack

        mov     dword [ebp-4],0   ; sum = 0
        mov     ecx, 1            ; ecx is i in pseudocode
for_loop:
        cmp     ecx, [ebp+8]      ; cmp i and n
        jnle    end_for           ; if not i <= n, quit

        add     [ebp-4], ecx      ; sum += i
        inc     ecx
        jmp     short for_loop

end_for:
        mov     eax, [ebp-4]      ; eax = sum

        leave
        ret
\end{AsmCodeListing}

\subsection{Calling C functions from assembly}

\begin{figure}[t]
\begin{AsmCodeListing}[frame=single]
segment .data
format       db "%d", 0

segment .text
...
      lea    eax, [ebp-16]
      push   eax
      push   dword format
      call   _scanf
      add    esp, 8
...
\end{AsmCodeListing}
\caption{Calling {\code scanf} from assembly\label{fig:scanf}}
\end{figure}

One great advantage of interfacing C and assembly is that allows
assembly code to access the large C library and user-written functions.
For example, what if one wanted to call the {\code scanf} function to
read in an integer from the keyboard? Figure~\ref{fig:scanf} shows
code to do this. One very important point to remember is that {\code
scanf} follows the C calling standard to the letter. This means that it
preserves the values of the EBX, ESI and EDI registers; however, the
EAX, ECX and EDX registers may be modified! In fact, EAX will definitely
be changed, as it will contain the return value of the {\code scanf} call.
For other examples of using interfacing with C, look at the code in
{\code asm\_io.asm} which was used to create {\code asm\_io.obj}.
\index{interfacing with C|)}

\section{Reentrant and Recursive Subprograms\index{recursion|(}}

\index{subprogram!reentrant|(}
A reentrant subprogram must satisfy the following properties:
\begin{itemize}
\item It must not modify any code instructions. In a high level language
this would be difficult, but in assembly it is not hard for a program to
try to modify its own code. For example:
\begin{AsmCodeListing}[frame=none, numbers=none]
      mov    word [cs:$+7], 5      ; copy 5 into the word 7 bytes ahead
      add    ax, 2                 ; previous statement changes 2 to 5!
\end{AsmCodeListing}
This code would work in real mode, but in protected mode operating systems 
the code segment is marked as read only. When the first line above executes,
the program will be aborted on these systems. This type of programming is
bad for many reasons. It is confusing, hard to maintain and does not allow
code sharing (see below).

\item It must not modify global data (such as data in the {\code data} and
the {\code bss} segments). All variables are stored on the stack.

\end{itemize}

There are several advantages to writing reentrant code.
\begin{itemize}
\item A reentrant subprogram can be called recursively.
\item A reentrant program can be shared by multiple processes. On many
multi-tasking operating systems, if there are multiple instances of a
program running, only \emph{one} copy of the code is in memory. Shared
libraries and DLL's (\emph{Dynamic Link Libraries}) use this idea as well.
\item Reentrant subprograms work much better in \emph{multi-threaded}
\footnote{A multi-threaded program has multiple threads of execution. That
is, the program itself is multi-tasked.} pro\-grams. Windows 9x/NT and most
UNIX-like operating systems (Solaris, Linux, \emph{etc.}) support 
multi-threaded programs.
\end{itemize}
\index{subprogram!reentrant|)}

\subsection{Recursive subprograms}

These types of subprograms call themselves. The recursion can be either
\emph{direct} or \emph{indirect}. Direct recursion occurs when a subprogram,
say {\code foo}, calls itself inside {\code foo}'s body. Indirect recursion
occurs when a subprogram is not called by itself directly, but by another
subprogram it calls. For example, subprogram {\code foo} could call
{\code bar} and {\code bar} could call {\code foo}.

Recursive subprograms must have a \emph{termination condition}. When
this condition is true, no more recursive calls are made. If a
recursive routine does not have a termination condition or the condition
never becomes true, the recursion will never end (much like an infinite
loop).

\begin{figure}
\begin{AsmCodeListing}[frame=single]
; finds n!
segment .text
      global _fact
_fact:
      enter  0,0

      mov    eax, [ebp+8]    ; eax = n
      cmp    eax, 1
      jbe    term_cond       ; if n <= 1, terminate
      dec    eax
      push   eax
      call   _fact           ; eax = fact(n-1)
      pop    ecx             ; answer in eax
      mul    dword [ebp+8]   ; edx:eax = eax * [ebp+8]
      jmp    short end_fact
term_cond:
      mov    eax, 1
end_fact:
      leave
      ret
\end{AsmCodeListing}
\caption{Recursive factorial function\label{fig:factorial}}
\end{figure}

\begin{figure}
\centering
%\includegraphics{factStack.eps}
\input{factStack.latex}
\caption{Stack frames for factorial function\label{fig:factStack}}
\end{figure}

Figure~\ref{fig:factorial} shows a function that calculates factorials
recursively. It could be called from C with:
\begin{lstlisting}[stepnumber=0]{}
x = fact(3);         /* find 3! */
\end{lstlisting}
Figure~\ref{fig:factStack} shows what the stack looks like at its deepest
point for the above function call.

\begin{figure}[t]
\begin{lstlisting}[frame=tlrb]{}
void f( int x )
{
  int i;
  for( i=0; i < x; i++ ) {
    printf("%d\n", i);
    f(i);
  }
}
\end{lstlisting}
\caption{Another example (C version)\label{fig:rec2C}}
\end{figure}

\begin{figure}
\begin{AsmCodeListing}[frame=single]
%define i ebp-4
%define x ebp+8          ; useful macros
segment .data
format       db "%d", 10, 0     ; 10 = '\n'
segment .text
      global _f
      extern _printf
_f:
      enter  4,0           ; allocate room on stack for i

      mov    dword [i], 0  ; i = 0
lp:
      mov    eax, [i]      ; is i < x?
      cmp    eax, [x]
      jnl    quit

      push   eax           ; call printf
      push   format
      call   _printf
      add    esp, 8

      push   dword [i]     ; call f
      call   _f
      pop    eax

      inc    dword [i]     ; i++
      jmp    short lp
quit:
      leave
      ret
\end{AsmCodeListing}
\caption{Another example (assembly version)\label{fig:rec2Asm}}
\end{figure}

Figures~\ref{fig:rec2C} and \ref{fig:rec2Asm} show another more
complicated recursive example in C and assembly, respectively. What is
the output is for {\code f(3)}? Note that the {\code ENTER} instruction
creates a new {\code i} on the stack for each recursive call. Thus, each
recursive instance of {\code f} has its own independent variable {\code i}.
Defining {\code i} as a double word in the {\code data} segment would not
work the same. 
\index{recursion|)}

\subsection{Review of C variable storage types}

C provides several types of variable storage.
\begin{description}
\item[global] 
\index{storage types!global}
These variables are defined outside of any function and
are stored at fixed memory locations (in the {\code data} or {\code
bss} segments) and exist from the beginning of the program until the
end. By default, they can be accessed from any function in the program;
however, if they are declared as {\code static}, only the functions in
the same module can access them (\emph{i.e.} in assembly terms, the
label is internal, not external).

\item[static] 
\index{storage types!static}
These are \emph{local} variables of a function that are
declared {\code static}. (Unfortunately, C uses the keyword {\code
static} for two different purposes!) These variables are also stored
at fixed memory locations (in {\code data} or {\code bss}), but can
only be directly accessed in the functions they are defined in. 

\item[automatic] 
\index{storage types!automatic}
This is the default type for a C variable defined inside a
function. These variables are allocated on the stack when the function
they are defined in is invoked and are deallocated when the function
returns. Thus, they do not have fixed memory locations.

\item[register] 
\index{storage types!register}
This keyword asks the compiler to use a register for
the data in this variable. This is just a \emph{request}. The compiler
does \emph{not} have to honor it. If the address of the variable is
used anywhere in the program it will not be honored (since registers
do not have addresses). Also, only simple integral types can be
register values.  Structured types can not be; they would not fit in a
register! C compilers will often automatically make normal automatic
variables into register variables without any hint from the programmer.

\item[volatile] 
\index{storage types!volatile}
This keyword tells the compiler that the value of the
variable may change any moment. This means that the compiler can not
make any assumptions about when the variable is modified. Often a
compiler might store the value of a variable in a register temporarily
and use the register in place of the variable in a section of code. It
can not do these types of optimizations with {\code volatile}
variables. A common example of a volatile variable would be one could
be altered by two threads of a multi-threaded program. Consider the 
following code:
\begin{lstlisting}{}
x = 10;
y = 20;
z = x;
\end{lstlisting}
If {\code x} could be altered by another thread, it is possible that the
other thread changes {\code x} between lines~1 and 3 so that {\code z}
would not be 10. However, if the {\code x} was not declared volatile, the
compiler might assume that {\code x} is unchanged and set {\code z} to 10.

Another use of {\code volatile} is to keep the compiler from using a
register for a variable. 

\end{description}
\index{subprogram|)}

% -*-latex-*-
\chapter{Arrays}
\index{arrays|(}
\section{Introduction}

An \emph{array} is a contiguous block of list of data in memory. Each element
of the list must be the same type and use exactly the same number of bytes
of memory for storage. Because of these properties, arrays allow efficient
access of the data by its position (or index) in the array. The address
of any element can be computed by knowing three facts:
\begin{itemize}
\item The address of the first element of the array.
\item The number of bytes in each element
\item The index of the element
\end{itemize}

It is convenient to consider the index of the first element of the array
to be zero (just as in C). It is possible to use other values for the
first index, but it complicates the computations.

\subsection{Defining arrays\index{arrays!defining|(}}

\begin{figure}[t]
\begin{AsmCodeListing}[frame=single]
segment .data
; define array of 10 double words initialized to 1,2,..,10
a1           dd   1, 2, 3, 4, 5, 6, 7, 8, 9, 10
; define array of 10 words initialized to 0
a2           dw   0, 0, 0, 0, 0, 0, 0, 0, 0, 0
; same as before using TIMES
a3           times 10 dw 0
; define array of bytes with 200 0's and then 100 1's
a4           times 200 db 0
             times 100 db 1

segment .bss
; define an array of 10 uninitialized double words
a5           resd  10
; define an array of 100 uninitialized words
a6           resw  100
\end{AsmCodeListing}
\caption{Defining arrays\label{fig:DataArrays}}
\end{figure}

\subsubsection{Defining arrays in the {\code data} and {\code bss} segments
               \index{arrays!defining!static}}

To define an initialized array in the {\code data} segment, use the
normal {\code db}, {\code dw}, \emph{etc.}
\index{directive!D\emph{X}}directives. NASM also provides a useful directive
named {\code TIMES} \index{directive!TIMES} that can be used to repeat a statement many times
without having to duplicate the statements by hand.
Figure~\ref{fig:DataArrays} shows several examples of these.

To define an uninitialized array in the {\code bss} segment, use the
{\code resb}, {\code resw}, \emph{etc.} \index{directive!RES\emph{X}}
directives. Remember that these directives have an operand that
specifies how many units of memory to
reserve. Figure~\ref{fig:DataArrays} also shows examples of these
types of definitions.

\begin{figure}[t]
\centering
\begin{tabular}{l|c|ll|c|}
\cline{2-2} \cline{5-5}
EBP - 1  & char    & \hspace{2em} &           & \\
\cline{2-2}
         & unused  &              &           & \\
\cline{2-2}
EBP - 8  & dword 1 &              &           & \\
\cline{2-2}
EBP - 12 & dword 2 &              &           & word \\
\cline{2-2}
         &         &              &           & array \\
         &         &              &           & \\
         & word    &              &           & \\
         & array   &              & EBP - 100 & \\
\cline{5-5}
         &         &              & EBP - 104 & dword 1 \\
\cline{5-5}
         &         &              & EBP - 108 & dword 2 \\
\cline{5-5}
         &         &              & EBP - 109 & char \\
\cline{5-5}
EBP - 112 &        &              &           & unused \\
\cline{2-2} \cline{5-5}
\end{tabular}
\caption{Arrangements of the stack\label{fig:StackLayouts}}
\end{figure}

\subsubsection{Defining arrays as local variables on the stack\index{arrays!defining!local variable}}

There is no direct way to define a local array variable on the
stack. As before, one computes the total bytes required by \emph{all}
local variables, including arrays, and subtracts this from ESP (either
directly or using the {\code ENTER} instruction). For example, if a
function needed a character variable, two double word integers and a
50 element word array, one would need $1 + 2 \times 4 + 50 \times 2 =
109$ bytes. However, the number subtracted from ESP should be a
multiple of four (112 in this case) to keep ESP on a double word
boundary. One could arrange the variables inside this 109 bytes in
several ways. Figure~\ref{fig:StackLayouts} shows two possible ways. The
unused part of the first ordering is there to keep the double words on
double word boundaries to speed up memory accesses.
\index{arrays!defining|)}

\subsection{Accessing elements of arrays\index{arrays!accessing|(}}

There is no {\code [ ]} operator in assembly language as in C. To
access an element of an array, its address must be computed. Consider
the following two array definitions:
\begin{AsmCodeListing}[frame=none, numbers=none]
array1       db     5, 4, 3, 2, 1     ; array of bytes
array2       dw     5, 4, 3, 2, 1     ; array of words
\end{AsmCodeListing}
Here are some examples using these arrays:
\begin{AsmCodeListing}[frame=none]
      mov    al, [array1]             ; al = array1[0]
      mov    al, [array1 + 1]         ; al = array1[1]
      mov    [array1 + 3], al         ; array1[3] = al
      mov    ax, [array2]             ; ax = array2[0]
      mov    ax, [array2 + 2]         ; ax = array2[1] (NOT array2[2]!)
      mov    [array2 + 6], ax         ; array2[3] = ax
      mov    ax, [array2 + 1]         ; ax = ??
\end{AsmCodeListing}
In line~5, element 1 of the word array is referenced, not element 2. Why?
Words are two byte units, so to move to the next element of a word array,
one must move two bytes ahead, not one. Line~7 will read one byte from the
first element and one from the second. In C, the compiler looks at the type
of a pointer in determining how many bytes to move in an expression that
uses pointer arithmetic so that the programmer does not have to. However,
in assembly, it is up to the programmer to take the size of array elements
in account when moving from element to element.

\begin{figure}[t]
\begin{AsmCodeListing}[frame=single,commandchars=\\\{\}]
      mov    ebx, array1           ; ebx = address of array1
      mov    dx, 0                 ; dx will hold sum
      mov    ah, 0                 ; ?
      mov    ecx, 5
lp:
      mov    al, [ebx]             ; al = *ebx
      add    dx, ax                ; dx += ax (not al!) \label{line:SumArray1}
      inc    ebx                   ; bx++
      loop   lp
\end{AsmCodeListing}
\caption{Summing elements of an array (Version 1)\label{fig:SumArray1}}
\end{figure}

\begin{figure}[t]
\begin{AsmCodeListing}[frame=single,commandchars=\\\{\}]
      mov    ebx, array1           ; ebx = address of array1
      mov    dx, 0                 ; dx will hold sum
      mov    ecx, 5
lp:
\textit{      add    dl, [ebx]             ; dl += *ebx}
\textit{      jnc    next                  ; if no carry goto next}
\textit{      inc    dh                    ; inc dh}
\textit{next:}
      inc    ebx                   ; bx++
      loop   lp
\end{AsmCodeListing}
\caption{Summing elements of an array (Version 2)\label{fig:SumArray2}}
\end{figure}

\begin{figure}[t]
\begin{AsmCodeListing}[frame=single,commandchars=\\\{\}]
      mov    ebx, array1           ; ebx = address of array1
      mov    dx, 0                 ; dx will hold sum
      mov    ecx, 5
lp:
\textit{      add    dl, [ebx]             ; dl += *ebx}
\textit{      adc    dh, 0                 ; dh += carry flag + 0}
      inc    ebx                   ; bx++
      loop   lp
\end{AsmCodeListing}
\caption{Summing elements of an array (Version 3)\label{fig:SumArray3}}
\end{figure}

Figure~\ref{fig:SumArray1} shows a code snippet that adds all the
elements of {\code array1} in the previous example code. In
line~\ref{line:SumArray1}, AX is added to DX. Why not AL? First, the
two operands of the {\code ADD} instruction must be the same
size. Secondly, it would be easy to add up bytes and get a sum that
was too big to fit into a byte. By using DX, sums up to 65,535 are
allowed. However, it is important to realize that AH is being added
also.  This is why AH is set to zero\footnote{Setting AH to zero is
implicitly assuming that AL is an unsigned number. If it is signed,
the appropriate action would be to insert a {\code CBW} instruction
between lines~6 and 7} in line~3.

Figures~\ref{fig:SumArray2} and \ref{fig:SumArray3} show two alternative
ways to calculate the sum. The lines in italics replace lines~6 and 7
of Figure~\ref{fig:SumArray1}.

\subsection{More advanced indirect addressing\index{indirect addressing!arrays|(}}

Not surprisingly, indirect addressing is often used with arrays. The most
general form of an indirect memory reference is:
\begin{center}
{\code [ \emph{base reg} + \emph{factor}*\emph{index reg} + 
      \emph{constant}]}
\end{center}
where:
\begin{description}
\item[base reg] is one of the registers EAX, EBX, ECX, EDX, EBP, ESP, ESI
                or EDI.
\item[factor] is either 1, 2, 4 or 8. (If 1, factor is omitted.)
\item[index reg] is one of the registers EAX, EBX, ECX, EDX, EBP, ESI, EDI.
                 (Note that ESP is not in list.)
\item[constant] is a 32-bit constant. The constant can be a label (or
                a label expression).
\end{description}

\subsection{Example}
Here is an example that uses an array and passes it to a function. It uses the
{\code array1c.c} program (listed below) as a driver, not the 
{\code driver.c} program. \index{array1.asm|(}
\begin{AsmCodeListing}[label=array1.asm]
%define ARRAY_SIZE 100
%define NEW_LINE 10

segment .data
FirstMsg        db   "First 10 elements of array", 0
Prompt          db   "Enter index of element to display: ", 0
SecondMsg       db   "Element %d is %d", NEW_LINE, 0
ThirdMsg        db   "Elements 20 through 29 of array", 0
InputFormat     db   "%d", 0

segment .bss
array           resd ARRAY_SIZE

segment .text
        extern  _puts, _printf, _scanf, _dump_line
        global  _asm_main
_asm_main:
        enter   4,0		; local dword variable at EBP - 4
        push    ebx
        push    esi

; initialize array to 100, 99, 98, 97, ...

        mov     ecx, ARRAY_SIZE
        mov     ebx, array
init_loop:
        mov     [ebx], ecx
        add     ebx, 4
        loop    init_loop

        push    dword FirstMsg         ; print out FirstMsg
        call    _puts
        pop     ecx

        push    dword 10
        push    dword array
        call    _print_array           ; print first 10 elements of array
        add     esp, 8

; prompt user for element index
Prompt_loop:
        push    dword Prompt
        call    _printf
        pop     ecx

        lea     eax, [ebp-4]      ; eax = address of local dword
        push    eax
        push    dword InputFormat
        call    _scanf
        add     esp, 8
        cmp     eax, 1               ; eax = return value of scanf
        je      InputOK

        call    _dump_line  ; dump rest of line and start over
        jmp     Prompt_loop          ; if input invalid

InputOK:
        mov     esi, [ebp-4]
        push    dword [array + 4*esi]
        push    esi
        push    dword SecondMsg      ; print out value of element
        call    _printf
        add     esp, 12

        push    dword ThirdMsg       ; print out elements 20-29
        call    _puts
        pop     ecx

        push    dword 10
        push    dword array + 20*4     ; address of array[20]
        call    _print_array
        add     esp, 8

        pop     esi
        pop     ebx
        mov     eax, 0            ; return back to C
        leave                     
        ret

;
; routine _print_array
; C-callable routine that prints out elements of a double word array as
; signed integers.
; C prototype:
; void print_array( const int * a, int n);
; Parameters:
;   a - pointer to array to print out (at ebp+8 on stack)
;   n - number of integers to print out (at ebp+12 on stack)

segment .data
OutputFormat    db   "%-5d %5d", NEW_LINE, 0

segment .text
        global  _print_array
_print_array:
        enter   0,0
        push    esi
        push    ebx

        xor     esi, esi                  ; esi = 0
        mov     ecx, [ebp+12]             ; ecx = n
        mov     ebx, [ebp+8]              ; ebx = address of array
print_loop:
        push    ecx                       ; printf might change ecx!

        push    dword [ebx + 4*esi]       ; push array[esi]
        push    esi
        push    dword OutputFormat
        call    _printf
        add     esp, 12                   ; remove parameters (leave ecx!)

        inc     esi
        pop     ecx
        loop    print_loop

        pop     ebx
        pop     esi
        leave
        ret
\end{AsmCodeListing}

\LabelLine{array1c.c}
\begin{lstlisting}{}
#include <stdio.h>

int asm_main( void );
void dump_line( void );

int main()
{
  int ret_status;
  ret_status = asm_main();
  return ret_status;
}

/*
 * function dump_line
 * dumps all chars left in current line from input buffer
 */
void dump_line()
{
  int ch;

  while( (ch = getchar()) != EOF && ch != '\n')
    /* null body*/ ;
}
\end{lstlisting}
\LabelLine{array1c.c}
\index{array1.asm|)}
\index{indirect addressing!arrays|)}
\index{arrays!accessing|)}

\subsubsection{The {\code LEA} instruction revisited\index{LEA|(}}

The {\code LEA} instruction can be used for other purposes than just
calcuating addresses. A fairly
common one is for fast computations. Consider the following:
\begin{AsmCodeListing}[numbers=none,frame=none]
      lea    ebx, [4*eax + eax]
\end{AsmCodeListing}
This effectively stores the value of $5 \times \mathtt{EAX}$ into EBX.
Using {\code LEA} to do this is both easier and faster than using
{\code MUL}\index{MUL}. However, one must realize that the expression
inside the square brackets \emph{must} be a legal indirect address.
Thus, for example, this instruction can not be used to multiple by 6
quickly.
\index{LEA|)}


\subsection{Multidimensional Arrays\index{arrays!multidimensional|(}}

Multidimensional arrays are not really very different than the plain
one dimensional arrays already discussed. In fact, they are represented 
in memory as just that, a plain one dimensional array.

\subsubsection{Two Dimensional Arrays\index{arrays!multidimensional!two dimensional|(}}
Not surprisingly, the simplest multidimensional array is a two dimensional
one. A two dimensional array is often displayed as a grid of elements. Each
element is identified by a pair of indices. By convention, the first index
is identified with the row of the element and the second index the column.

Consider an array with three rows and two columns defined as: 
\begin{lstlisting}[stepnumber=0]{}
  int a[3][2];
\end{lstlisting}
The C compiler would reserve room for a 6 ($= 2 \times 3$) integer array and
map the elements as follows:

\parbox{\textwidth}{
\vspace{0.5em}
\centering
\begin{tabular}{||l|c|c|c|c|c|c||}
\hline
Index & 0 & 1 & 2 & 3 & 4 & 5 \\
\hline
Element & a[0][0] & a[0][1] & a[1][0] & a[1][1] & a[2][0] & a[2][1]  \\
\hline
\end{tabular}
\vspace{0.5em}
}
\noindent What the table attempts to show is that the element referenced as 
{\code a[0][0]} is stored at the beginning of the 6 element one
dimensional array. Element {\code a[0][1]} is stored in the next
position (index~1) and so on. Each row of the two dimensional array is
stored contiguously in memory. The last element of a row is followed
by the first element of the next row. This is known as the
\emph{rowwise} representation of the array and is how a C/C++ compiler would
represent the array.

\begin{figure}[t]
\begin{AsmCodeListing}[]
   mov    eax, [ebp - 44]          ; ebp - 44 is i's location
   sal    eax, 1                   ; multiple i by 2
   add    eax, [ebp - 48]          ; add j
   mov    eax, [ebp + 4*eax - 40]  ; ebp - 40 is the address of a[0][0]
   mov    [ebp - 52], eax          ; store result into x (at ebp - 52)
\end{AsmCodeListing}
\caption{ Assembly for \lstinline|x = a[i][j]| \label{fig:aij}}
\end{figure}

How does the compiler determine where {\code a[i][j]} appears in the rowwise
representation? A simple formula will compute the index from {\code i} and
{\code j}. The formula in this case is $2i + j$. It's not too hard to see how
this formula is derived. Each row is two elements long; so, the first element
of row $i$ is at position $2i$. Then the position of column $j$ is found by
adding $j$ to $2i$. This analysis also shows how the formula is generalized 
to an array with {\code N} columns: $N \times i + j$. Notice that the formula
does \emph{not} depend on the number of rows.

As an example, let us see how \emph{gcc} compiles the following code (using the
array {\code a} defined above):
\begin{lstlisting}[stepnumber=0]{}
  x = a[i][j];
\end{lstlisting}
Figure~\ref{fig:aij} shows the assembly this is translated into.
Thus, the compiler essentially converts the code to:
\begin{lstlisting}[stepnumber=0]{}
  x = *(&a[0][0] + 2*i + j);
\end{lstlisting}
and in fact, the programmer could write this way with the same result.

There is nothing magical about the choice of the rowwise representation of the
array. A columnwise representation would work just as well: 

\parbox{\textwidth}{
\vspace{0.5em}
\centering
\begin{tabular}{||l|c|c|c|c|c|c||}
\hline
Index & 0 & 1 & 2 & 3 & 4 & 5 \\
\hline
Element & a[0][0] & a[1][0] & a[2][0] & a[0][1] & a[1][1] & a[2][1]  \\
\hline
\end{tabular}
\vspace{0.5em}
}
\noindent In the columnwise representation, each column is stored contiguously. 
Element {\code [i][j]} is stored at position $i + 3j$. Other languages
(FORTRAN, for example) use the columnwise representation. This is
important when interfacing code with multiple languages.
\index{arrays!multidimensional!two dimensional|)}

\subsubsection{Dimensions Above Two}
For dimensions above two, the same basic idea is applied. Consider a three
dimensional array:
\begin{lstlisting}[stepnumber=0]{}
  int b[4][3][2];
\end{lstlisting}
This array would be stored like it was four two dimensional arrays each of size
{\code [3][2]} consecutively in memory. The table below shows how it starts out:

\parbox{\textwidth}{
\vspace{0.5em}
\centering
\begin{tabular}{||l|c|c|c|c|c|c||}
\hline
Index & 0 & 1 & 2 & 3 & 4 & 5  \\
\hline
Element & b[0][0][0] & b[0][0][1]  & b[0][1][0] & b[0][1][1] & b[0][2][0]
&  b[0][2][1]  \\
\hline
\hline
Index & 6 & 7 & 8 & 9 & 10 & 11 \\
\hline
Element & b[1][0][0] & b[1][0][1] & b[1][1][0] & b[1][1][1]  & b[1][2][0] 
& b[1][2][1] \\
\hline
\end{tabular}
\vspace{0.5em}
}
\noindent The formula for computing the position of {\code b[i][j][k]}
is $6i + 2j + k$. The 6 is determined by the size of the {\code
[3][2]} arrays. In general, for an array dimensioned as {\code
a[L][M][N]} the position of element {\code a[i][j][k]} will be $M\times N\times i 
+ N \times j + k$. Notice again that the first
dimension ({\code L}) does not appear in the formula.

For higher dimensions, the same process is generalized. For an $n$ dimensional
array of dimensions $D_1$ to $D_n$, the position of element denoted by the
indices $i_1$ to $i_n$ is given by the formula:
\begin{displaymath}
D_2 \times D_3 \cdots \times D_n \times i_1 + D_3 \times D_4 \cdots \times D_n 
\times i_2 + \cdots + D_n \times i_{n-1} + i_n
\end{displaymath}
or for the \"{u}ber math geek, it can be written more succinctly as:
\begin{displaymath}
\sum_{j=1}^{n} \: \left( \prod_{k=j+1}^{n} D_k \right) \: i_j
\end{displaymath}
\MarginNote{This is where you can tell the author was a physics major. (Or was the
reference to FORTRAN a giveaway?)}
The first dimension, $D_1$, does not appear in the formula.

For the columnwise representation, the general formula would be:
\begin{displaymath}
i_1 + D_1 \times i_2 + \cdots + D_1 \times D_2 \times \cdots \times D_{n-2} 
\times i_{n-1} + D_1 \times D_2 \times \cdots \times D_{n-1} \times i_n
\end{displaymath}
or in \"{u}ber math geek notation:
\begin{displaymath}
\sum_{j=1}^{n} \: \left( \prod_{k=1}^{j-1} D_k \right) \: i_j
\end{displaymath}
In this case, it is the last dimension, $D_n$, that does not appear in the
formula.

\subsubsection{Passing Multidimensional Arrays as Parameters in C\index{arrays!multidimensional!parameters|(}}

The rowwise representation of multidimensional arrays has a direct
effect in C programming. For one dimensional arrays, the size of the
array is not required to compute where any specific element is located
in memory. This is not true for multidimensional arrays.  To access
the elements of these arrays, the compiler must know all but the first
dimension. This becomes apparent when considering the prototype of a
function that takes a multidimensional array as a parameter. The
following will not compile:
\begin{lstlisting}[stepnumber=0]{}
  void f( int a[ ][ ] );  /* no dimension information */
\end{lstlisting}
However, the following does compile:
\begin{lstlisting}[stepnumber=0]{}
  void f( int a[ ][2] );
\end{lstlisting}
Any two dimensional array with two columns can be passed to this function.
The first dimension is not required\footnote{A size can be specified here,
but it is ignored by the compiler.}.

Do not be confused by a function with this prototype:
\begin{lstlisting}[stepnumber=0]{}
  void f( int * a[ ] );
\end{lstlisting}
This defines a single dimensional array of integer pointers (which incidently
can be used to create an array of arrays that acts much like a two dimensional
array).

For higher dimensional arrays, all but the first dimension of the array must
be specified for parameters. For example, a four dimensional array parameter
might be passed like:
\begin{lstlisting}[stepnumber=0]{}
  void f( int a[ ][4][3][2] );
\end{lstlisting}
\index{arrays!multidimensional!parameters|)}
\index{arrays!multidimensional|)}

\section{Array/String Instructions}
\index{string instructions|(} 

The 80x86 family of processors provide several instructions that are
designed to work with arrays. These instructions are called
\emph{string instructions}. They use the index registers (ESI and EDI)
to perform an operation and then to automatically increment or
decrement one or both of the index registers. The \emph{direction
flag} (DF) \index{register!FLAGS!DF} in the FLAGS register determines
where the index registers are incremented or decremented. There are
two instructions that modify the direction flag:
\begin{description}
\item[CLD] \index{CLD} clears the direction flag. In this state, the index registers
           are incremented.
\item[STD] \index{STD} sets the direction flag. In this state, the index registers are
           decremented.
\end{description}
A \emph{very} common mistake in 80x86 programming is to forget to explicitly
put the direction flag in the correct state. This often leads to code that
works most of the time (when the direction flag happens to be in the desired
state), but does not work \emph{all} the time.

\begin{figure}[t]
\centering
{\code
\begin{tabular}{|lp{1.5in}|lp{1.5in}|}
\hline
LODSB & AL = [DS:ESI]\newline ESI = ESI $\pm$ 1 & 
STOSB & [ES:EDI] = AL\newline EDI = EDI $\pm$ 1 \\
\hline
LODSW & AX = [DS:ESI]\newline ESI = ESI $\pm$ 2 & 
STOSW & [ES:EDI] = AX\newline EDI = EDI $\pm$ 2 \\
\hline
LODSD & EAX = [DS:ESI]\newline ESI = ESI $\pm$ 4 & 
STOSD & [ES:EDI] = EAX\newline EDI = EDI $\pm$ 4 \\
\hline
\end{tabular}
}
\caption{Reading and writing string instructions\label{fig:rwString}
         \index{LODSB} \index{STOSB} \index{LODSW} \index{LODSD} \index{STOSD}}
\end{figure}

\begin{figure}[t]
\begin{AsmCodeListing}[frame=single]
segment .data
array1  dd  1, 2, 3, 4, 5, 6, 7, 8, 9, 10

segment .bss
array2  resd 10

segment .text
      cld                   ; don't forget this!
      mov    esi, array1
      mov    edi, array2
      mov    ecx, 10
lp:
      lodsd
      stosd
      loop  lp
\end{AsmCodeListing}
\caption{Load and store example\label{fig:lodEx}}
\end{figure}

\subsection{Reading and writing memory}

The simplest string instructions either read or write memory or
both. They may read or write a byte, word or double word at a time.
Figure~\ref{fig:rwString} shows these instructions with a short
pseudo-code description of what they do. There are several points to
notice here. First, ESI is used for reading and EDI for writing. It is
easy to remember this if one remembers that SI stands for \emph{Source
Index} and DI stands for \emph{Destination
Index}. \index{register!ESI} \index{register!EDI} Next, notice that
the register that holds the data is fixed (either AL, AX or
EAX). Finally, note that the storing instructions use ES to detemine
the segment to write to, not DS. In protected mode programming this is
not usually a problem, since there is only one data segment and ES
should be automatically initialized to reference it (just as DS
is). However, in real mode programming, it is \emph{very} important
for the programmer to initialize ES to the correct segment
\index{register!segment} selector value\footnote{Another complication
is that one can not copy the value of the DS register into the ES
register directly using a single {\code MOV} instruction. Instead, the
value of DS must be copied to a general purpose register (like AX) and
then copied from that register to ES using two {\code MOV}
instructions.}. Figure~\ref{fig:lodEx} shows an example use of these
instructions that copies an array into another.

\begin{figure}[t]
\centering
{\code
\begin{tabular}{|lp{2.5in}|}
\hline
MOVSB & byte [ES:EDI] = byte [DS:ESI] \newline ESI = ESI $\pm$ 1 \newline
        EDI = EDI $\pm$ 1 \\
\hline
MOVSW & word [ES:EDI] = word [DS:ESI] \newline ESI = ESI $\pm$ 2 \newline
        EDI = EDI $\pm$ 2 \\
\hline
MOVSD & dword [ES:EDI] = dword [DS:ESI] \newline ESI = ESI $\pm$ 4 \newline
        EDI = EDI $\pm$ 4 \\
\hline
\end{tabular}
}
\caption{Memory move string instructions\label{fig:movString} \index{MOVSB}
         \index{MOVSW} \index{MOVSD}}
\end{figure}

\begin{figure}[t]
\begin{AsmCodeListing}[frame=single]
segment .bss
array  resd 10

segment .text
      cld                   ; don't forget this!
      mov    edi, array
      mov    ecx, 10
      xor    eax, eax
      rep stosd
\end{AsmCodeListing}
\caption{Zero array example\label{fig:zeroArrayEx}}
\end{figure}

The combination of a {\code LODSx} and {\code STOSx} instruction (as in
lines~13 and 14 of Figure~\ref{fig:lodEx}) is very common. In fact, this
combination can be performed by a single {\code MOVSx} string instruction.
Figure~\ref{fig:movString} describes the operations that these 
instructions perform. Lines~13 and 14 of Figure~\ref{fig:lodEx} could be
replaced with a single {\code MOVSD} instruction with the same effect. The
only difference would be that the EAX register would not be used at all
in the loop.

\subsection{The {\code REP} instruction prefix\index{REP|(}}

The 80x86 family provides a special instruction prefix\footnote{A
instruction prefix is not an instruction, it is a special byte that is
placed before a string instruction that modifies the instructions
behavior. Other prefixes are also used to override segment defaults of
memory accesses} called {\code REP} that can be used with the above string
instructions. This prefix tells the CPU to repeat the next string instruction
a specified number of times. The ECX register is used to count the iterations
(just as for the {\code LOOP} instruction). Using the {\code REP} prefix, 
the loop in Figure~\ref{fig:lodEx} (lines~12 to 15) could be replaced with
a single line:
\begin{AsmCodeListing}[frame=none, numbers=none]
      rep movsd
\end{AsmCodeListing}
Figure~\ref{fig:zeroArrayEx} shows another example that zeroes out the
contents of an array.
\index{REP|)}

\begin{figure}[t]
\centering
{\code
\begin{tabular}{|lp{3.5in}|}
\hline
CMPSB & compares byte [DS:ESI] and byte [ES:EDI] \newline ESI = ESI $\pm$ 1 
        \newline EDI = EDI $\pm$ 1 \\
\hline
CMPSW & compares word [DS:ESI] and word [ES:EDI] \newline ESI = ESI $\pm$ 2 
        \newline EDI = EDI $\pm$ 2 \\
\hline
CMPSD & compares dword [DS:ESI] and dword [ES:EDI] \newline ESI = ESI $\pm$ 4 
        \newline EDI = EDI $\pm$ 4 \\
\hline
SCASB & compares AL and [ES:EDI] \newline EDI $\pm$ 1 \\
\hline
SCASW & compares AX and [ES:EDI] \newline EDI $\pm$ 2 \\
\hline
SCASD & compares EAX and [ES:EDI] \newline EDI $\pm$ 4 \\
\hline
\end{tabular}
}
\caption{Comparison string instructions\label{fig:cmpString}
         \index{CMPSB} \index{CMPSW} \index{CMPSD} \index{SCASB}
         \index{SCASW} \index{SCASD}}
\end{figure}

\begin{figure}[t]
\begin{AsmCodeListing}[frame=single,commandchars=\\\{\}]
segment .bss
array        resd 100

segment .text
      cld
      mov    edi, array    ; pointer to start of array
      mov    ecx, 100      ; number of elements
      mov    eax, 12       ; number to scan for
lp:
      scasd    \label{line:scasdSrchStrEx}
      je     found
      loop   lp
 ; code to perform if not found
      jmp    onward
found:
      sub    edi, 4         ; edi now points to 12 in array\label{line:subSrchStrEx}
 ; code to perform if found
onward:
\end{AsmCodeListing}
\caption{Search example\label{fig:srchStrEx}}
\end{figure}

\subsection{Comparison string instructions}

Figure~\ref{fig:cmpString} shows several new string instructions that
can be used to compare memory with other memory or a register. They
are useful for comparing or searching arrays. They set the FLAGS
register just like the {\code CMP} instruction. The {\code CMPSx}
\index{CMPSB} \index{CMPSW} \index{CMPSD} instructions compare
corresponding memory locations and the {\code SCASx} \index{SCASB}
\index{SCASW} \index{SCASD} scan memory locations for a specific
value.

Figure~\ref{fig:srchStrEx} shows a short code snippet that searches
for the number 12 in a double word array. The {\code SCASD} instruction on
line~\ref{line:scasdSrchStrEx} always adds 4 to EDI, even if the value
searched for is found. Thus, if one wishes to find the address of the 12
found in the array, it is necessary to subtract 4 from EDI (as 
line~\ref{line:subSrchStrEx} does).

\begin{figure}[t]
\centering
\begin{tabular}{|l|p{4in}|}
\hline
{\code REPE}, {\code REPZ} & repeats instruction while Z flag is set or
                             at most ECX times \\
\hline
{\code REPNE}, {\code REPNZ} & repeats instruction while Z flag is cleared or
                             at most ECX times \\
\hline
\end{tabular}
\caption{{\code REPx} instruction prefixes\label{fig:repx} \index{REPE} 
          \index{REPNE}}
\end{figure}

\begin{figure}
\begin{AsmCodeListing}[frame=single,commandchars=\\\{\}]
segment .text
      cld
      mov    esi, block1        ; address of first block
      mov    edi, block2        ; address of second block
      mov    ecx, size          ; size of blocks in bytes
      repe   cmpsb              ; repeat while Z flag is set
      je     equal              ; if Z set, blocks equal \label{line:cmpBlocksEx}
   ; code to perform if blocks are not equal
      jmp    onward
equal:
   ; code to perform if equal
onward:
\end{AsmCodeListing}
\caption{Comparing memory blocks\label{fig:cmpBlocksEx}}
\end{figure}

\subsection{The {\code REPx} instruction prefixes}

There are several other {\code REP}-like instruction prefixes that can be
used with the comparison string instructions. Figure~\ref{fig:repx} shows
the two new prefixes and describes their operation. {\code REPE} \index{REPE} and
{\code REPZ} are just synonyms for the same prefix (as are {\code REPNE} \index{REPNE}
and {\code REPNZ}). If the repeated comparison string instruction stops
because of the result of the comparison, the index register or registers
are still incremented and ECX decremented; however, the FLAGS register
still holds the state that terminated the repetition. 
\MarginNote{Why can one not just look to see if ECX is zero after the
repeated comparison?} Thus, it is possible
to use the Z flag to determine if the repeated comparisons stopped because
of a comparison or ECX becoming zero.

Figure~\ref{fig:cmpBlocksEx} shows an example code snippet that determines
if two blocks of memory are equal. The {\code JE} on 
line~\ref{line:cmpBlocksEx} of the example checks to see the result of the
previous instruction. If the repeated comparison stopped because it found
two unequal bytes, the Z flag will still be cleared and no branch is made;
however, if the comparisons stopped because ECX became zero, the Z flag
will still be set and the code branches to the {\code equal} label.

\subsection{Example}

This section contains an assembly source file with several functions that
implement array operations using string instructions. Many of the functions
duplicate familiar C library functions.

\index{memory.asm|(}
\begin{AsmCodeListing}[label=memory.asm]
global _asm_copy, _asm_find, _asm_strlen, _asm_strcpy

segment .text
; function _asm_copy
; copies blocks of memory
; C prototype
; void asm_copy( void * dest, const void * src, unsigned sz);
; parameters:
;   dest - pointer to buffer to copy to
;   src  - pointer to buffer to copy from
;   sz   - number of bytes to copy

; next, some helpful symbols are defined

%define dest [ebp+8]
%define src  [ebp+12]
%define sz   [ebp+16]
_asm_copy:
        enter   0, 0
        push    esi
        push    edi

        mov     esi, src        ; esi = address of buffer to copy from
        mov     edi, dest       ; edi = address of buffer to copy to
        mov     ecx, sz         ; ecx = number of bytes to copy

        cld                     ; clear direction flag 
        rep     movsb           ; execute movsb ECX times

        pop     edi
        pop     esi
        leave
        ret


; function _asm_find
; searches memory for a given byte
; void * asm_find( const void * src, char target, unsigned sz);
; parameters:
;   src    - pointer to buffer to search
;   target - byte value to search for
;   sz     - number of bytes in buffer
; return value:
;   if target is found, pointer to first occurrence of target in buffer
;   is returned
;   else
;     NULL is returned
; NOTE: target is a byte value, but is pushed on stack as a dword value.
;       The byte value is stored in the lower 8-bits.
; 
%define src    [ebp+8]
%define target [ebp+12]
%define sz     [ebp+16]

_asm_find:
        enter   0,0
        push    edi

        mov     eax, target     ; al has value to search for
        mov     edi, src
        mov     ecx, sz
        cld

        repne   scasb           ; scan until ECX == 0 or [ES:EDI] == AL

        je      found_it        ; if zero flag set, then found value
        mov     eax, 0          ; if not found, return NULL pointer
        jmp     short quit
found_it:
        mov     eax, edi          
        dec     eax              ; if found return (DI - 1)
quit:
        pop     edi
        leave
        ret


; function _asm_strlen
; returns the size of a string
; unsigned asm_strlen( const char * );
; parameter:
;   src - pointer to string
; return value:
;   number of chars in string (not counting, ending 0) (in EAX)

%define src [ebp + 8]
_asm_strlen:
        enter   0,0
        push    edi

        mov     edi, src        ; edi = pointer to string
        mov     ecx, 0FFFFFFFFh ; use largest possible ECX
        xor     al,al           ; al = 0
        cld

        repnz   scasb           ; scan for terminating 0

;
; repnz will go one step too far, so length is FFFFFFFE - ECX,
; not FFFFFFFF - ECX
;
        mov     eax,0FFFFFFFEh
        sub     eax, ecx        ; length = 0FFFFFFFEh - ecx

        pop     edi
        leave
        ret

; function _asm_strcpy
; copies a string
; void asm_strcpy( char * dest, const char * src);
; parameters:
;   dest - pointer to string to copy to
;   src  - pointer to string to copy from
; 
%define dest [ebp + 8]
%define src  [ebp + 12]
_asm_strcpy:
        enter   0,0
        push    esi
        push    edi

        mov     edi, dest
        mov     esi, src
        cld
cpy_loop:
        lodsb                   ; load AL & inc si
        stosb                   ; store AL & inc di
        or      al, al          ; set condition flags
        jnz     cpy_loop        ; if not past terminating 0, continue

        pop     edi
        pop     esi
        leave
        ret
\end{AsmCodeListing}

\LabelLine{memex.c}
\begin{lstlisting}{}
#include <stdio.h>

#define STR_SIZE 30
/* prototypes */

void asm_copy( void *, const void *, unsigned ) __attribute__((cdecl));
void * asm_find( const void *, 
                 char target, unsigned ) __attribute__((cdecl));
unsigned asm_strlen( const char * ) __attribute__((cdecl));
void asm_strcpy( char *, const char * ) __attribute__((cdecl));

int main()
{
  char st1[STR_SIZE] = "test string";
  char st2[STR_SIZE];
  char * st;
  char   ch;

  asm_copy(st2, st1, STR_SIZE);   /* copy all 30 chars of string */
  printf("%s\n", st2);

  printf("Enter a char: ");  /* look for byte in string */
  scanf("%c%*[^\n]", &ch);
  st = asm_find(st2, ch, STR_SIZE);
  if ( st )
    printf("Found it: %s\n", st);
  else
    printf("Not found\n");

  st1[0] = 0;
  printf("Enter string:");
  scanf("%s", st1);
  printf("len = %u\n", asm_strlen(st1));

  asm_strcpy( st2, st1);     /* copy meaningful data in string */
  printf("%s\n", st2 );

  return 0;
}
\end{lstlisting}
\LabelLine{memex.c}
\index{memory.asm|)}
\index{string instructions|)}
\index{arrays|)}














% -*-latex-*-
\chapter{Floating Point\index{floating point|(}}

\section{Floating Point Representation\index{floating point!representation|(}}

\subsection{Non-integral binary numbers}

When number systems were discussed in the first chapter, only integer values
were discussed. Obviously, it must be possible to represent non-integral
numbers in other bases as well as decimal. In decimal, digits to the right
of the decimal point have associated negative powers of ten:
\[ 0.123 = 1 \times 10^{-1} + 2 \times 10^{-2} + 3 \times 10^{-3} \]

Not surprisingly, binary numbers work similarly:
\[ 0.101_2 = 1 \times 2^{-1} + 0 \times 2^{-2} + 1 \times 2^{-3} = 0.625 \]
This idea can be combined with the integer methods of Chapter~1 to convert
a general number:
\[ 110.011_2 = 4 + 2 + 0.25 + 0.125 = 6.375 \]

Converting from decimal to binary is not very difficult either. In general,
divide the decimal number into two parts: integer and fraction. Convert the
integer part to binary using the methods from Chapter~1. The fractional part
is converted using the method described below.

\begin{figure}[t]
\centering
\fbox{
\begin{tabular}{p{2in}p{2in}}
\begin{eqnarray*}
0.5625 \times 2 & = & 1.125 \\
0.125 \times 2 & = & 0.25 \\
0.25 \times 2 & = & 0.5 \\
0.5 \times 2 & = & 1.0 \\
\end{eqnarray*}
&
\begin{eqnarray*}
\mbox{first bit} & = & 1 \\
\mbox{second bit} & = & 0 \\
\mbox{third bit} & = & 0 \\
\mbox{fourth bit} & = & 1 \\
\end{eqnarray*}
\end{tabular}
}
\caption{Converting 0.5625 to binary\label{fig:binConvert1}}
\end{figure}

Consider a binary fraction with the bits labeled $a, b, c, \ldots$ The number
in binary then looks like:
\[ 0.abcdef\ldots \]
Multiply the number by two. The binary representation of the new number will
be:
\[ a.bcdef\ldots \]
Note that the first bit is now in the one's place. Replace the $a$ with $0$
to get:
\[ 0.bcdef\ldots \]
and multiply by two again to get:
\[ b.cdef\ldots \]
Now the second bit ($b$) is in the one's position. This procedure can be 
repeated until as many bits are needed are found. Figure~\ref{fig:binConvert1}
shows a real example that converts 0.5625 to binary. The method stops when
a fractional part of zero is reached.

\begin{figure}[t]
\centering
\fbox{\parbox{2in}{
\begin{eqnarray*}
0.85 \times 2 & = & 1.7 \\
0.7 \times 2 & = &  1.4 \\
0.4 \times 2 & = &  0.8 \\
0.8 \times 2 & = &  1.6 \\
0.6 \times 2 & = &  1.2 \\
0.2 \times 2 & = &  0.4 \\
0.4 \times 2 & = &  0.8 \\
0.8 \times 2 & = &  1.6 \\
\end{eqnarray*}
}}
\caption{Converting 0.85 to binary\label{fig:binConvert2}}
\end{figure}

As another example, consider converting 23.85 to binary. It is easy to 
convert the integral part ($23 = 10111_2$), but what about the fractional
part ($0.85$)? Figure~\ref{fig:binConvert2} shows the beginning of this
calculation. If one looks at the numbers carefully, an infinite loop is
found! This means that 0.85 is a repeating binary (as opposed to a 
repeating decimal in base 10)\footnote{It should not be so surprising that
a number might repeat in one base, but not another. Think about 
$\frac{1}{3}$, it repeats in decimal, but in ternary (base 3) it would be
$0.1_3$.}. There is a pattern to the numbers in the calculation. Looking
at the pattern, one can see that $0.85 = 0.11\overline{0110}_2$. Thus,
$23.85 = 10111.11\overline{0110}_2$.

One important consequence of the above calculation is that 23.85 can 
not be represented \emph{exactly} in binary using a finite number of bits.
(Just as $\frac{1}{3}$ can not be represented in decimal with a finite
number of digits.) As this chapter shows, {\code float} and {\code double}
variables in C are stored in binary. Thus, values like 23.85 can not be
stored exactly into these variables. Only an approximation of 23.85 can be
stored.

To simplify the hardware, floating point numbers are stored in a consistent
format. This format uses scientific notation (but in binary, using powers of
two, not ten). For example, 23.85 or $10111.11011001100110\ldots_2$ would be
stored as:
\[ 1.011111011001100110\ldots \times 2^{100} \]
(where the exponent (100) is in binary). A \emph{normalized} floating point
number has the form:
\[ 1.ssssssssssssssss \times 2^{eeeeeee} \]
where $1.sssssssssssss$ is the \emph{significand} and $eeeeeeee$ is the
\emph{exponent}.

\subsection{IEEE floating point representation\index{floating point!representation!IEEE|(}}

The IEEE (Institute of Electrical and Electronic Engineers) is an
international organization that has designed specific binary formats
for storing floating point num\-bers. This format is used on most (but
not all!)  computers made today. Often it is supported by the hardware
of the computer itself. For example, Intel's numeric (or math)
coprocessors (which are built into all its CPUs since the Pentium)
use it. The IEEE defines two different formats with different
precisions: single and double precision. Single precision is used by
{\code float} variables in C and double precision is used by {\code
double} variables.

Intel's math coprocessor also uses a third, higher precision called
\emph{extended precision}. In fact, all data in the coprocessor itself is
in this precision. When it is stored in memory from the coprocessor it
is converted to either single or double precision automatically.\footnote{
Some compiler's (such as Borland) {\code long double} type uses this
extended precision. However, other compilers use double precision for
both {\code double} and {\code long double}. (This is allowed by ANSI C.)}
Extended precision uses a slightly different general format than the
IEEE float and double formats and so will not be discussed here.

\subsubsection{IEEE single precision\index{floating point!representation!single precision|(}}

\begin{figure}[t]
\fbox{
\centering
\parbox{5in}{
\begin{tabular}{|c|c|c|}
\multicolumn{1}{p{0.3cm}}{31} &
\multicolumn{1}{p{2.5cm}}{30 \hfill 23} &
\multicolumn{1}{p{6cm}}{22 \hfill 0} \\
\hline
s & e & f \\
\hline
\end{tabular}
\\[0.4cm]
\begin{tabular}{cp{4.5in}}
s & sign bit - 0 = positive, 1 = negative \\
e & biased exponent (8-bits) = true exponent + 7F (127 decimal). The
    values 00 and FF have special meaning (see text). \\
f & fraction - the first 23-bits after the 1. in the significand.
\end{tabular}
}}
\caption{IEEE single precision\label{fig:IEEEsingle}}
\end{figure}

Single precision floating point uses 32 bits to encode the number. It is
usually accurate to 7 significant decimal digits. Floating point num\-bers are
stored in a much more complicated format than integers. 
Figure~\ref{fig:IEEEsingle} shows the basic format of a IEEE single precision
number. There are several quirks to the format. Floating point numbers do
not use the two's complement representation for negative numbers. They use
a signed magnitude representation. Bit 31 determines the sign of the number
as shown.

The binary exponent is not stored directly. Instead, the sum of the
exponent and 7F is stored from bit 23 to 30. This
\emph{biased exponent} is always non-negative.

The fraction part assumes a normalized significand (in the form 
$1.sssssssss$). Since the first bit is always a one, the leading one is
\emph{not stored!} This allows the storage of an additional bit at the end
and so increases the precision slightly. This idea is know as the
\emph{hidden one representation}\index{floating point!representation!hidden one}.

How would 23.85 be stored? \MarginNote{One should always keep in mind
that the bytes 41 BE CC CD can be interpreted different ways depending
on what a program does with them! As as single precision floating
point number, they represent 23.850000381, but as a double word
integer, they represent 1,103,023,309! The CPU does not know which is
the correct interpretation!} First, it is positive so the sign bit is
0. Next the true exponent is 4, so the biased exponent is $7\mathrm{F}
+ 4 = 83_{16}$. Finally, the fraction is 01111101100110011001100
(remember the leading one is hidden). Putting this all together
(to help clarify the different sections of the floating point
format, the sign bit and the fraction have been underlined and the
bits have been grouped into 4-bit nibbles):
\[ \underline{0}\,100\;0001\;1
   \,\underline{011\;1110\;1100\;1100\;1100\;1100}_2 = 41 \mathrm{BE} 
\mathrm{CC} \mathrm{CC}_{16} \]
This is not exactly 23.85 (since it is a repeating binary). If one converts
the above back to decimal, one finds that it is approximately 
23.849998474. This number is very close to 23.85, but it is not exact. 
Actually, in C, 23.85 would not be represented exactly as above. Since
the left-most bit that was truncated from the exact representation is 1,
the last bit is rounded up to 1. So 23.85 would be represented as
41 BE CC CD in hex using single precision. Converting this to decimal
results in 23.850000381 which is a slightly better approximation of 23.85.

How would -23.85 be represented? Just change the sign bit: C1 BE CC
CD. Do \emph{not} take the two's complement!

\begin{table}[t]
\fbox{
\begin{tabular}{lp{3.1in}}
$e=0 \quad\mathrm{and}\quad f=0$ & denotes the number zero (which can not be 
                         normalized) Note that there is a +0 and -0. \\
$e=0 \quad\mathrm{and}\quad f \neq 0$ & denotes a \emph{denormalized number}. These
                              are discussed in the next section. \\
$e=\mathrm{FF} \quad\mathrm{and}\quad f=0$ 
& denotes infinity ($\infty$). There are both positive and negative 
infinities. \\
$e=\mathrm{FF} \quad\mathrm{and}\quad f\neq 0$ 
& denotes an undefined result, known as \emph{NaN} (Not a Number).
\end{tabular}
}
\caption{Special values of \emph{f} and \emph{e}\label{tab:floatSpecials}}
\end{table}

Certain combinations of \emph{e} and \emph{f} have special meanings for
IEEE floats. Table~\ref{tab:floatSpecials} describes these special values.
An infinity is produced by an overflow or by division by zero. An undefined
result is produced by an invalid operation such as trying to find the
square root of a negative number, adding two infinities, \emph{etc.}

Normalized single precision numbers can range in magnitude from 
$1.0 \times 2^{-126}$ ($\approx 1.1755 \times 10^{-35}$) to 
$1.11111\ldots \times 2^{127}$ ($\approx 3.4028 \times 10^{35}$).

\subsubsection{Denormalized numbers\index{floating point!representation!denormalized|(}}

Denormalized numbers can be used to represent numbers with magnitudes too
small to normalize (\emph{i.e.} below $1.0 \times 2^{-126}$). For example,
consider the number $1.001_2 \times 2^{-129}$ 
($\approx 1.6530 \times 10^{-39}$). In the given normalized form, the 
exponent is too small. However, it can be represented in the unnormalized 
form: $0.01001_2 \times 2^{-127}$. To store this number, the biased exponent
is set to 0 (see Table~\ref{tab:floatSpecials}) and the fraction is the 
complete significand of the number written as a product with $2^{-127}$ 
({\emph{i.e.} all bits are stored including the one to the left of the 
decimal point). The representation of $1.001 \times 2^{-129}$ is then:
\[ \underline{0}\,000\;0000\;0
   \,\underline{001\;0010\;0000\;0000\;0000\;0000} \]
\index{floating point!representation!denormalized|)}
\index{floating point!representation!single precision|)}


\subsubsection{IEEE double precision\index{floating point!representation!double precision|(}}

\begin{figure}[t]
\centering
\begin{tabular}{|c|c|c|}
\multicolumn{1}{p{0.3cm}}{63} &
\multicolumn{1}{p{3cm}}{62 \hfill 52} &
\multicolumn{1}{p{7cm}}{51 \hfill 0} \\
\hline
s & e & f \\
\hline
\end{tabular}
\caption{IEEE double precision\label{fig:IEEEdouble}}
\end{figure}

IEEE double precision uses 64 bits to represent numbers and is usually
accurate to about 15 significant decimal digits. As 
Figure~\ref{fig:IEEEdouble} shows, the basic format is very similar to 
single precision. More bits are used for the biased exponent (11) and the
fraction (52) than for single precision.

The larger range for the biased exponent has two consequences. The first is
that it is calculated as the sum of the true exponent and 3FF (1023) (not
7F as for single precision). Secondly, a large range of true exponents (and
thus a larger range of magnitudes) is allowed. Double precision magnitudes
can range from approximately $10^{-308}$ to $10^{308}$.

It is the larger field of the fraction that is responsible for the increase
in the number of significant digits for double values.

As an example, consider 23.85 again. The biased exponent will be 
$4 + \mathrm{3FF} = 403$ in hex. Thus, the double representation would be:
\[ \underline{0}\,100\;0000\;0011\;\underline{0111\;1101\;1001\;1001\;1001\;
   1001\;1001\;1001\;1001\;1001\;1001\;1001\;1010} \]
or 40 37 D9 99 99 99 99 9A in hex. If one converts this back to decimal, one
finds 23.8500000000000014 (there are 12 zeros!) which is a much better 
approximation of 23.85.

The double precision has the same special values as single
precision\footnote{The only difference is that for the infinity and
undefined values, the biased exponent is 7FF not FF.}. Denormalized
numbers are also very similar. The only main difference is that double
denormalized numbers use $2^{-1023}$ instead of $2^{-127}$.
\index{floating point!representation!double precision|)}
\index{floating point!representation!IEEE|)}
\index{floating point!representation|)}

\section{Floating Point Arithmetic\index{floating point!arithmetic|(}}

Floating point arithmetic on a computer is different than in 
continuous mathematics. In mathematics, all numbers can be considered exact.
As shown in the previous section, on a computer many numbers can not be
represented exactly with a finite number of bits. All calculations are
performed with limited precision. In the examples of this section, numbers
with an 8-bit significand will be used for simplicity.

\subsection{Addition}
To add two floating point numbers, the exponents must be equal. If
they are not already equal, then they must be made equal by shifting
the significand of the number with the smaller exponent. For example,
consider $10.375 + 6.34375 = 16.71875$ or in binary:
\[
\begin{array}{rr}
 & 1.0100110 \times 2^3 \\
+& 1.1001011 \times 2^2 \\ \hline
\end{array}
\]
These two numbers do not have the same exponent so shift the significand to
make the exponents the same and then add:
\[
\begin{array}{rr@{.}l}
 &  1&0100110 \times 2^3 \\
+&  0&1100110 \times 2^3 \\ \hline
 & 10&0001100 \times 2^3
\end{array}
\]
Note that the shifting of $1.1001011 \times 2^2$ drops off the trailing one
and after rounding results in $0.1100110 \times 2^3$. The result of the
addition, $10.0001100 \times 2^3$ (or $1.00001100 \times 2^4$) is equal to
$10000.110_2$ or 16.75. This is \emph{not} equal to the exact answer
(16.71875)! It is only an approximation due to the round off errors of the
addition process. 

It is important to realize that floating point arithmetic on a
computer (or calculator) is always an approximation. The laws of
mathematics do not always work with floating point numbers on a
computer.  Mathematics assumes infinite precision which no computer
can match. For example, mathematics teaches that $(a + b) - b = a$;
however, this may not hold true exactly on a computer!

\subsection{Subtraction}
Subtraction works very similarly and has the same problems as addition. As
an example, consider $16.75 - 15.9375 = 0.8125$:
\[
\begin{array}{rr}
 & 1.0000110 \times 2^4 \\
-& 1.1111111 \times 2^3 \\ \hline
\end{array}
\]
Shifting $1.1111111 \times 2^3$ gives (rounding up) $1.0000000 \times 2^4$
\[
\begin{array}{rr}
 & 1.0000110 \times 2^4 \\
-& 1.0000000 \times 2^4 \\ \hline
 & 0.0000110 \times 2^4
\end{array}
\]
$0.0000110 \times 2^4 = 0.11_2 = 0.75$ which is not exactly correct.

\subsection{Multiplication and division}

For multiplication, the significands are multiplied and the exponents are
added. Consider $10.375 \times 2.5 = 25.9375$:
\[
\begin{array}{rr@{}l}
 &  1.0&100110 \times 2^3 \\
\times &  1.0&100000 \times 2^1 \\ \hline
 &     &10100110 \\
+&   10&100110   \\ \hline
 &   1.1&0011111000000 \times 2^4
\end{array}
\]
Of course, the real result would be rounded to 8-bits to give:
\[1.1010000 \times 2^4 = 11010.000_2 = 26 \]

Division is more complicated, but has similar problems with round off errors.

\subsection{Ramifications for programming}

The main point of this section is that floating point calculations are
not exact. The programmer needs to be aware of this. A common mistake that
programmers make with floating point numbers is to compare them assuming
that a calculation is exact. For example, consider a function named
\lstinline|f(x)| that makes a complex calculation and a program is
trying to find the function's roots\footnote{A root of a function is a 
value $x$ such that $f(x) = 0$}. One might be tempted to use the following
statement to check to see if \lstinline|x| is a root:
\begin{lstlisting}[stepnumber=0]{}
  if ( f(x) == 0.0 )
\end{lstlisting}
But, what if \lstinline|f(x)| returns $1 \times 10^{-30}$? This very
likely means that \lstinline|x| is a \emph{very} good approximation of 
a true root; however, the equality will be false. There may not be any
IEEE floating point value of \lstinline|x| that returns exactly zero, due
to round off errors in \lstinline|f(x)|. 

A much better method would be to use:
\begin{lstlisting}[stepnumber=0]{}
  if ( fabs(f(x)) < EPS )
\end{lstlisting}
where \lstinline|EPS| is a macro defined to be a very small positive value
(like $1 \times 10^{-10}$). This is true whenever \lstinline|f(x)| is very
close to zero. In general, to compare a floating point value (say 
\lstinline|x|) to another (\lstinline|y|) use:
\begin{lstlisting}[stepnumber=0]{}
  if ( fabs(x - y)/fabs(y) < EPS )
\end{lstlisting}
\index{floating point!arithmetic|)}

\section{The Numeric Coprocessor}
\index{floating point coprocessor|(}
\subsection{Hardware}
\index{floating point coprocessor!hardware|(}
The earliest Intel processors had no hardware support for floating
point operations. This does not mean that they could not perform float
operations.  It just means that they had to be performed by procedures
composed of many non-floating point instructions. For these early
systems, Intel did provide an additional chip called a \emph{math
coprocessor}. A math coprocessor has machine instructions that perform
many floating point operations much faster than using a software
procedure (on early processors, at least 10 times faster!). The
coprocessor for the 8086/8088 was called the 8087. For the 80286,
there was a 80287 and for the 80386, a 80387. The 80486DX processor
integrated the math coprocessor into the 80486
itself.\footnote{However, the 80486SX did \emph{not} have have an
integrated coprocessor.  There was a separate 80487SX chip for these
machines.}  Since the Pentium, all generations of 80x86 processors
have a built-in math coprocessor; however, it is still programmed as if
it was a separate unit. Even earlier systems without a coprocessor can
install software that emulates a math coprocessor. These emulator
packages are automatically activated when a program executes a
coprocessor instruction and run a software procedure that produces the
same result as the coprocessor would have (though much slower, of
course).

The numeric coprocessor has eight floating point registers. Each
register holds 80 bits of data. Floating point numbers are
\emph{always} stored as 80-bit extended precision numbers in these
registers. The registers are named {\code ST0}, {\code ST1}, {\code
ST2}, $\ldots$ {\code ST7}.  The floating point registers are used
differently than the integer registers of the main CPU. The floating
point registers are organized as a \emph{stack}.  Recall that a stack
is a \emph{Last-In First-Out} (LIFO) list. {\code ST0} always refers
to the value at the top of the stack. All new numbers are added to the
top of the stack. Existing numbers are pushed down on the stack to
make room for the new number.

There is also a status register in the numeric coprocessor. It has several
flags. Only the 4 flags used for comparisons will be covered: C$_0$,
C$_1$, C$_2$ and C$_3$. The use of these is discussed later.
\index{floating point coprocessor!hardware|)}

\subsection{Instructions}

To make it easy to distinguish the normal CPU instructions from coprocessor
ones, all the coprocessor mnemonics start with an {\code F}.

\subsubsection{Loading and storing\index{floating point coprocessor!loading and storing data|(}}
There are several instructions that load data onto the top of the coprocessor
register stack:\\
\begin{tabular}{lp{4in}}
{\code FLD \emph{source}} \index{FLD} & 
loads a floating point number from memory onto the top of the stack. The 
\emph{source} may be a single, double or extended precision number or
a coprocessor register. \\ 
{\code FILD \emph{source}} \index{FILD} &
reads an \emph{integer} from memory, converts it to floating point and
stores the result on top of the stack. The \emph{source} may be either
a word, double word or quad word. \\
{\code FLD1} \index{FLD1} &
stores a one on the top of the stack. \\
{\code FLDZ} \index{FLDZ} &
stores a zero on the top of the stack. \\
\end{tabular}

There are also several instructions that store data from the stack into
memory. Some of these instructions also \emph{pop} (\emph{i.e.} remove)
the number from the stack as it stores it.\\
\begin{tabular}{lp{4in}}
{\code FST \emph{dest}} \index{FST} &
stores the top of the stack ({\code ST0}) into memory. The 
\emph{destination} may either be a single or double precision number or a
coprocessor register.\\
{\code FSTP \emph{dest}} \index{FSTP} &
stores the top of the stack into memory just as {\code FST};
however, after the number is stored, its value is popped from the stack. The 
\emph{destination} may either a single, double or extended precision number or
a coprocessor register.\\
{\code FIST \emph{dest}} \index{FIST} &
stores the value of the top of the stack converted to an integer into memory. 
The \emph{destination} may either a word or a double word. The
stack itself is unchanged. How the floating point number is converted to
an integer depends on some bits in the coprocessor's \emph{control word}.
This is a special (non-floating point) word register that controls how the
coprocessor works. By default, the control word is initialized so that
it rounds to the nearest integer when it converts to integer. However, the
{\code FSTCW} (Store Control Word) and {\code FLDCW} (Load Control Word)
instructions can be used to change this behavior. \index{FSTCW} \index{FLDCW} \\
{\code FISTP \emph{dest}} \index{FIST} &
Same as {\code FIST} except for two things. The top of the stack is popped
and the \emph{destination} may also be a quad word.
\end{tabular}

There are two other instructions that can move or remove data on the
stack itself.\\
\begin{tabular}{lp{4in}}
{\code FXCH ST\emph{n}} \index{FXCH}  &
exchanges the values in {\code ST0} and {\code ST\emph{n}} on the stack
(where \emph{n} is register number from 1 to 7). \\
{\code FFREE ST\emph{n}} \index{FFREE} &
frees up a register on the stack by marking the register as unused or empty.
\end{tabular}
\index{floating point coprocessor!loading and storing data|)}

\subsubsection{Addition and subtraction\index{floating point coprocessor!addition and subtraction|(}}

Each of the addition instructions compute the sum of {\code ST0} and another
operand. The result is always stored in a coprocessor register.\\
\begin{tabular}{p{1.5in}p{3.5in}}
{\code FADD \emph{src}} \index{FADD} &
{\code ST0 += \emph{src}}. The \emph{src} may be any coprocessor register
or a single or double precision number in memory. \\
{\code FADD \emph{dest}, ST0} &
{\code \emph{dest} += ST0}. The \emph{dest} may be any coprocessor register. \\
{\code FADDP \emph{dest}} or \newline {\code FADDP \emph{dest}, STO} \index{FADDP} &
{\code \emph{dest} += ST0} then pop stack. The \emph{dest} may be any
coprocessor register. \\
{\code FIADD \emph{src}} \index{FIADD} &
{\code ST0 += (float) \emph{src}}. Adds an integer to {\code ST0}. The
\emph{src} must be a word or double word in memory.
\end{tabular}

\begin{figure}[t]
\begin{AsmCodeListing}[frame=single]
segment .bss
array        resq SIZE
sum          resq 1

segment .text
      mov    ecx, SIZE
      mov    esi, array
      fldz                  ; ST0 = 0
lp:
      fadd   qword [esi]    ; ST0 += *(esi)
      add    esi, 8         ; move to next double
      loop   lp
      fstp   qword sum      ; store result into sum
\end{AsmCodeListing}
\caption{Array sum example\label{fig:addEx}}
\end{figure}

There are twice as many subtraction instructions than addition because
the order of the operands is important for subtraction (\emph{i.e.}
$a + b = b + a$, but $a - b \neq b - a$!). For each instruction, there is an 
alternate one that subtracts in the reverse order. These reverse instructions
all end in either {\code R} or {\code RP}. Figure~\ref{fig:addEx} shows
a short code snippet that adds up the elements of an array of doubles. On
lines~10 and 13, one must specify the size of the memory operand. 
Otherwise the assembler would not know whether the memory operand was a
float (dword) or a double (qword).

\begin{tabular}{p{1.5in}p{3.5in}}
{\code FSUB \emph{src}} \index{FSUB} &
{\code ST0 -= \emph{src}}. The \emph{src} may be any coprocessor register
or a single or double precision number in memory. \\
{\code FSUBR \emph{src}} \index{FSUBR} &
{\code ST0 = \emph{src} - ST0}. The \emph{src} may be any coprocessor register
or a single or double precision number in memory. \\
{\code FSUB \emph{dest}, ST0} &
{\code \emph{dest} -= ST0}. The \emph{dest} may be any coprocessor register. \\
{\code FSUBR \emph{dest}, ST0} &
{\code \emph{dest} = ST0 - \emph{dest}}. The \emph{dest} may be any 
coprocessor register. \\
{\code FSUBP \emph{dest}} or \newline {\code FSUBP \emph{dest}, STO} \index{FSUBP} &
{\code \emph{dest} -= ST0} then pop stack. The \emph{dest} may be any
coprocessor register. \\
{\code FSUBRP \emph{dest}} or \newline {\code FSUBRP \emph{dest}, STO} \index{FSUBRP} &
{\code \emph{dest} = ST0 - \emph{dest}} then pop stack. The \emph{dest} may 
be any coprocessor register. \\
{\code FISUB \emph{src}} \index{FISUB} &
{\code ST0 -= (float) \emph{src}}. Subtracts an integer from {\code ST0}. The
\emph{src} must be a word or double word in memory. \\
{\code FISUBR \emph{src}} \index{FISUBR} &
{\code ST0 = (float) \emph{src} - ST0}. Subtracts {\code ST0} from an integer.
 The \emph{src} must be a word or double word in memory.
\end{tabular}

\index{floating point coprocessor!addition and subtraction|)}

\subsubsection{Multiplication and division\index{floating point coprocessor!multiplication and division|(}}

The multiplication instructions are completely analogous to the addition
instructions.\\
\begin{tabular}{p{1.5in}p{3.5in}}
{\code FMUL \emph{src}} \index{FMUL} &
{\code ST0 *= \emph{src}}. The \emph{src} may be any coprocessor register
or a single or double precision number in memory. \\
{\code FMUL \emph{dest}, ST0} &
{\code \emph{dest} *= ST0}. The \emph{dest} may be any coprocessor register. \\
{\code FMULP \emph{dest}} or \newline {\code FMULP \emph{dest}, STO} \index{FMULP} &
{\code \emph{dest} *= ST0} then pop stack. The \emph{dest} may be any
coprocessor register. \\
{\code FIMUL \emph{src}} \index{FMUL} &
{\code ST0 *= (float) \emph{src}}. Multiplies an integer to {\code ST0}. The
\emph{src} must be a word or double word in memory.
\end{tabular}

Not surprisingly, the division instructions are analogous to the subtraction
instructions. Division by zero results in an infinity. \\
\begin{tabular}{p{1.5in}p{3.5in}}
{\code FDIV \emph{src}} \index{FDIV} &
{\code ST0 /= \emph{src}}. The \emph{src} may be any coprocessor register
or a single or double precision number in memory. \\
{\code FDIVR \emph{src}} \index{FDIVR} &
{\code ST0 = \emph{src} / ST0}. The \emph{src} may be any coprocessor register
or a single or double precision number in memory. \\
{\code FDIV \emph{dest}, ST0} &
{\code \emph{dest} /= ST0}. The \emph{dest} may be any coprocessor register. \\
{\code FDIVR \emph{dest}, ST0} &
{\code \emph{dest} = ST0 / \emph{dest}}. The \emph{dest} may be any 
coprocessor register. \\
{\code FDIVP \emph{dest}} or \newline {\code FDIVP \emph{dest}, STO} \index{FDIVP} &
{\code \emph{dest} /= ST0} then pop stack. The \emph{dest} may be any
coprocessor register. \\
{\code FDIVRP \emph{dest}} or \newline {\code FDIVRP \emph{dest}, STO} \index{FDIVRP} &
{\code \emph{dest} = ST0 / \emph{dest}} then pop stack. The \emph{dest} may 
be any coprocessor register. \\
{\code FIDIV \emph{src}} \index{FIDIV} &
{\code ST0 /= (float) \emph{src}}. Divides {\code ST0} by an integer. The
\emph{src} must be a word or double word in memory. \\
{\code FIDIVR \emph{src}} \index{FIDIVR} &
{\code ST0 = (float) \emph{src} / ST0}. Divides an integer by {\code ST0}.
 The \emph{src} must be a word or double word in memory.
\end{tabular}
\index{floating point coprocessor!multiplication and division|)}

\subsubsection{Comparisons\index{floating point coprocessor!comparisons|(}}

The coprocessor also performs comparisons of floating point numbers. The
{\code FCOM} family of instructions does this operation. \\
\begin{tabular}{lp{4in}}
{\code FCOM \emph{src}} \index{FCOM} & 
compares {\code ST0} and {\code \emph{src}}. The \emph{src} can be a 
coprocessor register or a float or double in memory. \\
{\code FCOMP \emph{src}} \index{FCOMP} & 
compares {\code ST0} and {\code \emph{src}}, then pops stack. The \emph{src} 
can be a coprocessor register or a float or double in memory. \\
{\code FCOMPP} \index{FCOMPP} & 
compares {\code ST0} and {\code ST1}, then pops stack twice. \\
{\code FICOM \emph{src}} \index{FICOM} & 
compares {\code ST0} and {\code (float) \emph{src}}. The \emph{src} can be a 
word or dword integer in memory. \\
{\code FICOMP \emph{src}} \index{FICOMP} & 
compares {\code ST0} and {\code (float)\emph{src}}, then pops stack. 
The \emph{src} can be a word or dword integer in memory. \\
{\code FTST } \index{FTST} &
compares {\code ST0} and 0.
\end{tabular}

\begin{figure}[t]
\begin{AsmCodeListing}[frame=single]
;     if ( x > y )
;
      fld    qword [x]       ; ST0 = x
      fcomp  qword [y]       ; compare STO and y
      fstsw  ax              ; move C bits into FLAGS
      sahf
      jna    else_part       ; if x not above y, goto else_part
then_part:
      ; code for then part
      jmp    end_if
else_part:
      ; code for else part
end_if:
\end{AsmCodeListing}
\caption{Comparison example\label{fig:compEx}}
\end{figure}

These instructions change the C$_0$, C$_1$, C$_2$ and C$_3$ bits of
the coprocessor status register.  Unfortunately, it is not possible
for the CPU to access these bits directly. The conditional branch 
instructions use the FLAGS register, not the coprocessor status register.
However, it is relatively simple to transfer the bits of the status word
into the corresponding bits of the FLAGS register using some new 
instructions:\\
\begin{tabular}{lp{4in}}
{\code FSTSW \emph{dest}} \index{FSTSW} & 
Stores the coprocessor status word into either a word in memory or the AX
register. \\
{\code SAHF} \index{SAHF} & 
Stores the AH register into the FLAGS register. \\
{\code LAHF} \index{LAHF} & 
Loads the AH register with the bits of the FLAGS register. \\
\end{tabular}

\begin{figure}[t]
\begin{AsmCodeListing}[frame=single]
global _dmax

segment .text
; function _dmax
; returns the larger of its two double arguments
; C prototype
; double dmax( double d1, double d2 )
; Parameters:
;   d1   - first double
;   d2   - second double
; Return value:
;   larger of d1 and d2 (in ST0)
%define d1   ebp+8
%define d2   ebp+16
_dmax:
        enter   0, 0

        fld     qword [d2]
        fld     qword [d1]          ; ST0 = d1, ST1 = d2
        fcomip  st1                 ; ST0 = d2
        jna     short d2_bigger
        fcomp   st0                 ; pop d2 from stack
        fld     qword [d1]          ; ST0 = d1
        jmp     short exit
d2_bigger:                          ; if d2 is max, nothing to do
exit:
        leave
        ret
\end{AsmCodeListing}
\caption{{\code FCOMIP} example \label{fig:fcomipEx}}
\index{FCOMIP}
\end{figure}

Figure~\ref{fig:compEx} shows a short example code snippet. Lines~5
and 6 transfer the C$_0$, C$_1$, C$_2$ and C$_3$ bits of the
coprocessor status word into the FLAGS register. The bits are
transfered so that they are analogous to the result of a comparison
of two \emph{unsigned} integers. This is why line~7 uses a {\code JNA}
instruction.

The Pentium Pro (and later processors (Pentium II and III)) support two
new comparison operators that directly modify the CPU's FLAGS register.

\begin{tabular}{lp{4in}}
{\code FCOMI \emph{src}} \index{FCOMI} & 
compares {\code ST0} and {\code \emph{src}}. The \emph{src} must be a 
coprocessor register. \\
{\code FCOMIP \emph{src}} \index{FCOMIP} & 
compares {\code ST0} and {\code \emph{src}}, then pops stack. The \emph{src} 
must be a coprocessor register. \\
\end{tabular}
Figure~\ref{fig:fcomipEx} shows an example subroutine that finds the
maximum of two doubles using the {\code FCOMIP} instruction. Do not confuse
these instructions with the integer comparison functions ({\code FICOM}
and {\code FICOMP}).
\index{floating point coprocessor!comparisons|)}

\subsubsection{Miscellaneous instructions}
%FINIT?

\begin{figure}[t]
\begin{AsmCodeListing}[frame=single]
segment .data
x            dq  2.75          ; converted to double format
five         dw  5

segment .text
      fild   dword [five]      ; ST0 = 5
      fld    qword [x]         ; ST0 = 2.75, ST1 = 5
      fscale                   ; ST0 = 2.75 * 32, ST1 = 5
\end{AsmCodeListing}
\caption{{\code FSCALE} example\label{fig:fscaleEx}}
\index{FSCALE}
\end{figure}

This section covers some other miscellaneous instructions that the 
coprocessor provides.

\begin{tabular}{lp{4in}}
{\code FCHS} \index{FCHS} & 
{\code ST0 = - ST0} Changes the sign of {\code ST0}  \\
{\code FABS} \index{FABS} & 
$\mathtt{ST0} = |\mathtt{ST0}|$ Takes the absolute value of {\code ST0}\\
{\code FSQRT} \index{FSQRT} &
$\mathtt{ST0} = \sqrt{\mathtt{STO}}$ Takes the square root of {\code ST0} \\
{\code FSCALE} \index{FSCALE} &
$\mathtt{ST0} = \mathtt{ST0} \times 2^{\lfloor \mathtt{ST1} \rfloor}$
multiples {\code ST0} by a power of 2 quickly. {\code ST1} is not
removed from the coprocessor stack. Figure~\ref{fig:fscaleEx} shows an
example of how to use this instruction.
\end{tabular}


\subsection{Examples}

\subsection{Quadratic formula\index{quad.asm|(}}

The first example shows how the quadratic formula can be encoded in assembly.
Recall that the quadratic formula computes the solutions to the quadratic
equation:
\[ a x^2 + b x + c = 0 \]
The formula itself gives two solutions for $x$: $x_1$ and $x_2$.
\[ x_1, x_2 = \frac{-b \pm \sqrt{b^2 - 4 a c}}{2 a} \]
The expression inside the square root ($b^2 - 4 a c$) is called the
\emph{discriminant}. Its value is useful in determining which of the
following three possibilities are true for the solutions.
\begin{enumerate}
\item There is only one real degenerate solution. $b^2 - 4 a c = 0$
\item There are two real solutions. $b^2 - 4 a c > 0$
\item There are two complex solutions. $b^2 - 4 a c < 0$
\end{enumerate}

Here is a small C program that uses the assembly subroutine:
\LabelLine{quadt.c}
\begin{lstlisting}{}
#include <stdio.h>

int quadratic( double, double, double, double *, double *);

int main()
{
  double a,b,c, root1, root2;

  printf("Enter a, b, c: ");
  scanf("%lf %lf %lf", &a, &b, &c);
  if (quadratic( a, b, c, &root1, &root2) )
    printf("roots: %.10g %.10g\n", root1, root2);
  else
    printf("No real roots\n");
  return 0;
}
\end{lstlisting}
\LabelLine{quadt.c}

Here is the assembly routine:
\begin{AsmCodeListing}[label=quad.asm,commentchar=$]
; function quadratic
; finds solutions to the quadratic equation: 
;       a*x^2 + b*x + c = 0
; C prototype:
;   int quadratic( double a, double b, double c,
;                  double * root1, double *root2 )
; Parameters:
;   a, b, c - coefficients of powers of quadratic equation (see above)
;   root1   - pointer to double to store first root in
;   root2   - pointer to double to store second root in
; Return value:
;   returns 1 if real roots found, else 0

%define a               qword [ebp+8]
%define b               qword [ebp+16]
%define c               qword [ebp+24]
%define root1           dword [ebp+32]
%define root2           dword [ebp+36]
%define disc            qword [ebp-8]
%define one_over_2a     qword [ebp-16]

segment .data
MinusFour       dw      -4

segment .text
        global  _quadratic
_quadratic:
        push    ebp
        mov     ebp, esp
        sub     esp, 16         ; allocate 2 doubles (disc & one_over_2a)
        push    ebx             ; must save original ebx

        fild    word [MinusFour]; stack -4
        fld     a               ; stack: a, -4
        fld     c               ; stack: c, a, -4
        fmulp   st1             ; stack: a*c, -4
        fmulp   st1             ; stack: -4*a*c
        fld     b
        fld     b               ; stack: b, b, -4*a*c
        fmulp   st1             ; stack: b*b, -4*a*c
        faddp   st1             ; stack: b*b - 4*a*c
        ftst                    ; test with 0
        fstsw   ax
        sahf
        jb      no_real_solutions ; if disc < 0, no real solutions
        fsqrt                   ; stack: sqrt(b*b - 4*a*c)
        fstp    disc            ; store and pop stack
        fld1                    ; stack: 1.0
        fld     a               ; stack: a, 1.0
        fscale                  ; stack: a * 2^(1.0) = 2*a, 1
        fdivp   st1             ; stack: 1/(2*a)
        fst     one_over_2a     ; stack: 1/(2*a)
        fld     b               ; stack: b, 1/(2*a)
        fld     disc            ; stack: disc, b, 1/(2*a)
        fsubrp  st1             ; stack: disc - b, 1/(2*a)
        fmulp   st1             ; stack: (-b + disc)/(2*a)
        mov     ebx, root1
        fstp    qword [ebx]     ; store in *root1
        fld     b               ; stack: b
        fld     disc            ; stack: disc, b
        fchs                    ; stack: -disc, b
        fsubrp  st1             ; stack: -disc - b
        fmul    one_over_2a     ; stack: (-b - disc)/(2*a)
        mov     ebx, root2
        fstp    qword [ebx]     ; store in *root2
        mov     eax, 1          ; return value is 1
        jmp     short quit

no_real_solutions:
        mov     eax, 0          ; return value is 0

quit:
        pop     ebx
        mov     esp, ebp
        pop     ebp
        ret
\end{AsmCodeListing}
\index{quad.asm|)}

\subsection{Reading array from file\index{read.asm|(}}

In this example, an assembly routine reads doubles from a file. Here is
a short C test program:
\LabelLine{readt.c}
\begin{lstlisting}{}
/*
 * This program tests the 32-bit read_doubles() assembly procedure.
 * It reads the doubles from stdin. (Use redirection to read from file.)
 */
#include <stdio.h>
extern int read_doubles( FILE *, double *, int );
#define MAX 100

int main()
{
  int i,n;
  double a[MAX];

  n = read_doubles(stdin, a, MAX);

  for( i=0; i < n; i++ )
    printf("%3d %g\n", i, a[i]);
  return 0;
}
\end{lstlisting}
\LabelLine{readt.c}

Here is the assembly routine
\begin{AsmCodeListing}[label=read.asm]
segment .data
format  db      "%lf", 0        ; format for fscanf()

segment .text
        global  _read_doubles
        extern  _fscanf

%define SIZEOF_DOUBLE   8
%define FP              dword [ebp + 8]
%define ARRAYP          dword [ebp + 12]
%define ARRAY_SIZE      dword [ebp + 16]
%define TEMP_DOUBLE     [ebp - 8]

;
; function _read_doubles
; C prototype:
;   int read_doubles( FILE * fp, double * arrayp, int array_size );
; This function reads doubles from a text file into an array, until
; EOF or array is full.
; Parameters:
;   fp         - FILE pointer to read from (must be open for input)
;   arrayp     - pointer to double array to read into
;   array_size - number of elements in array
; Return value:
;   number of doubles stored into array (in EAX)

_read_doubles:
        push    ebp
        mov     ebp,esp
        sub     esp, SIZEOF_DOUBLE      ; define one double on stack

        push    esi                     ; save esi
        mov     esi, ARRAYP             ; esi = ARRAYP
        xor     edx, edx                ; edx = array index (initially 0)

while_loop:
        cmp     edx, ARRAY_SIZE         ; is edx < ARRAY_SIZE?
        jnl     short quit              ; if not, quit loop
;
; call fscanf() to read a double into TEMP_DOUBLE
; fscanf() might change edx so save it
;
        push    edx                     ; save edx
        lea     eax, TEMP_DOUBLE
        push    eax                     ; push &TEMP_DOUBLE
        push    dword format            ; push &format
        push    FP                      ; push file pointer
        call    _fscanf
        add     esp, 12
        pop     edx                     ; restore edx
        cmp     eax, 1                  ; did fscanf return 1?
        jne     short quit              ; if not, quit loop

;
; copy TEMP_DOUBLE into ARRAYP[edx]
; (The 8-bytes of the double are copied by two 4-byte copies)
;
        mov     eax, [ebp - 8]
        mov     [esi + 8*edx], eax      ; first copy lowest 4 bytes
        mov     eax, [ebp - 4]
        mov     [esi + 8*edx + 4], eax  ; next copy highest 4 bytes

        inc     edx
        jmp     while_loop

quit:
        pop     esi                     ; restore esi

        mov     eax, edx                ; store return value into eax

        mov     esp, ebp
        pop     ebp
        ret 
\end{AsmCodeListing}
\index{read.asm|)}

\subsection{Finding primes\index{prime2.asm|(}}

This final example looks at finding prime numbers again. This implementation
is more efficient than the previous one. It stores the primes it has found
in an array and only divides by the previous primes it has found instead of
every odd number to find new primes.

One other difference is that it computes the square root of the guess for the
next prime to determine at what point it can stop searching for factors. It
alters the coprocessor control word so that when it stores the 
square root as an integer, it truncates instead of rounding. This is 
controlled by bits 10 and 11 of the control word. These bits are called the
RC (Rounding Control) bits. If they are both 0 (the default), the coprocessor
rounds when converting to integer. If they are both 1, the coprocessor
truncates integer conversions. Notice that the routine is careful to save
the original control word and restore it before it returns.

Here is the C driver program:
\LabelLine{fprime.c}
\begin{lstlisting}{}
#include <stdio.h>
#include <stdlib.h>
/*
 * function find_primes
 * finds the indicated number of primes
 * Parameters:
 *   a - array to hold primes
 *   n - how many primes to find
 */
extern void find_primes( int * a, unsigned n );

int main()
{
  int status;
  unsigned i;
  unsigned max;
  int * a;

  printf("How many primes do you wish to find? ");
  scanf("%u", &max);

  a = calloc( sizeof(int), max);

  if ( a ) {

    find_primes(a,max);

    /* print out the last 20 primes found */
    for(i= ( max > 20 ) ? max - 20 : 0; i < max; i++ )
      printf("%3d %d\n", i+1, a[i]);

    free(a);
    status = 0;
  }
  else {
    fprintf(stderr, "Can not create array of %u ints\n", max);
    status = 1;
  }

  return status;
}
\end{lstlisting}
\LabelLine{fprime.c}

Here is the assembly routine:


\begin{AsmCodeListing}[label=prime2.asm]
segment .text
        global  _find_primes
;
; function find_primes
; finds the indicated number of primes
; Parameters:
;   array  - array to hold primes
;   n_find - how many primes to find
; C Prototype:
;extern void find_primes( int * array, unsigned n_find )
;
%define array         ebp + 8
%define n_find        ebp + 12
%define n             ebp - 4           ; number of primes found so far
%define isqrt         ebp - 8           ; floor of sqrt of guess
%define orig_cntl_wd  ebp - 10          ; original control word
%define new_cntl_wd   ebp - 12          ; new control word

_find_primes:
        enter   12,0                    ; make room for local variables

        push    ebx                     ; save possible register variables
        push    esi

        fstcw   word [orig_cntl_wd]     ; get current control word
        mov     ax, [orig_cntl_wd]
        or      ax, 0C00h               ; set rounding bits to 11 (truncate)
        mov     [new_cntl_wd], ax
        fldcw   word [new_cntl_wd]

        mov     esi, [array]            ; esi points to array
        mov     dword [esi], 2          ; array[0] = 2
        mov     dword [esi + 4], 3      ; array[1] = 3
        mov     ebx, 5                  ; ebx = guess = 5
        mov     dword [n], 2            ; n = 2
;
; This outer loop finds a new prime each iteration, which it adds to the
; end of the array. Unlike the earlier prime finding program, this function
; does not determine primeness by dividing by all odd numbers. It only
; divides by the prime numbers that it has already found. (That's why they
; are stored in the array.)
;
while_limit:
        mov     eax, [n]
        cmp     eax, [n_find]           ; while ( n < n_find )
        jnb     short quit_limit

        mov     ecx, 1                  ; ecx is used as array index
        push    ebx                     ; store guess on stack
        fild    dword [esp]             ; load guess onto coprocessor stack
        pop     ebx                     ; get guess off stack
        fsqrt                           ; find sqrt(guess)
        fistp   dword [isqrt]           ; isqrt = floor(sqrt(quess))
;
; This inner loop divides guess (ebx) by earlier computed prime numbers
; until it finds a prime factor of guess (which means guess is not prime)
; or until the prime number to divide is greater than floor(sqrt(guess))
;
while_factor:
        mov     eax, dword [esi + 4*ecx]        ; eax = array[ecx]
        cmp     eax, [isqrt]                    ; while ( isqrt < array[ecx] 
        jnbe    short quit_factor_prime
        mov     eax, ebx
        xor     edx, edx
        div     dword [esi + 4*ecx]     
        or      edx, edx                        ; && guess % array[ecx] != 0 )
        jz      short quit_factor_not_prime
        inc     ecx                             ; try next prime
        jmp     short while_factor

;
; found a new prime !
;
quit_factor_prime:
        mov     eax, [n]
        mov     dword [esi + 4*eax], ebx        ; add guess to end of array
        inc     eax
        mov     [n], eax                        ; inc n

quit_factor_not_prime:
        add     ebx, 2                          ; try next odd number
        jmp     short while_limit

quit_limit:

        fldcw   word [orig_cntl_wd]             ; restore control word
        pop     esi                             ; restore register variables
        pop     ebx

        leave
        ret 
\end{AsmCodeListing}
\index{prime2.asm|)}
\index{floating point coprocessor|)}
\index{floating point|)}

% -*-latex-*-
\chapter{Structures and C++}

\section{Structures\index{structures|(}}

\subsection{Introduction}

Structures are used in C to group together related data into a composite 
variable. This technique has several advantages:
\begin{enumerate}
\item It clarifies the code by showing that the data defined in the structure
      are intimately related.
\item It simplifies passing the data to functions. Instead of passing
      multiple variables separately, they can be passed as a single unit.
\item It increases the \index{locality}\emph{locality}\footnote{See the virtual memory 
management section of any Operating System text book for discussion of
this term.} of the code.
\end{enumerate}

From the assembly standpoint, a structure can be considered as an
array with elements of \emph{varying} size. The elements of real
arrays are always the same size and type. This property is what allows
one to calculate the address of any element by knowing the starting
address of the array, the size of the elements and the desired
element's index.

A structure's elements do not have to be the same size (and usually
are not). Because of this each element of a structure must be
explicitly specified and is given a \emph{tag} (or name) instead of a
numerical index.

In assembly, the element of a structure will be accessed in a similar
way as an element of an array. To access an element, one must know the
starting address of the structure and the \emph{relative offset} of
that element from the beginning of the structure. However, unlike an
array where this offset can be calculated by the index of the element, 
the element of a structure is assigned an offset by the compiler.

For example, consider the following structure:
\begin{lstlisting}[stepnumber=0]{}
struct S {
  short int x;    /* 2-byte integer */
  int       y;    /* 4-byte integer */
  double    z;    /* 8-byte float   */
};
\end{lstlisting}

\begin{figure}
\centering
\begin{tabular}{r|c|}
\multicolumn{1}{c}{Offset} & \multicolumn{1}{c}{ Element } \\
\cline{2-2}
0 & {\code x} \\
\cline{2-2}
2 & \\
  & {\code y} \\
\cline{2-2}
6 & \\
  & \\
  & {\code z} \\
  & \\
\cline{2-2}
\end{tabular}
\caption{Structure S \label{fig:structPic1}}
\end{figure}

Figure~\ref{fig:structPic1} shows how a variable of type {\code S}
might look in the computer's memory. The ANSI C standard states that
the elements of a structure are arranged in the memory in the same
order as they are defined in the {\code struct} definition. It also
states that the first element is at the very beginning of the
structure (\emph{i.e.} offset zero). It also defines another useful
macro in the {\code stddef.h} header file named {\code
offsetof()}. \index{structures!offsetof()} This macro computes and
returns the offset of any element of a structure. The macro takes two
parameters, the first is the name of the \emph{type} of the structure,
the second is the name of the element to find the offset of. Thus, the
result of {\code offsetof(S, y)} would be 2 from
Figure~\ref{fig:structPic1}.

%TODO: talk about definition of offsetof() ??

\subsection{Memory alignment}

\begin{figure}
\centering
\begin{tabular}{r|c|}
\multicolumn{1}{c}{Offset} & \multicolumn{1}{c}{ Element } \\
\cline{2-2}
0 & {\code x} \\
\cline{2-2}
2 & \emph{unused} \\
\cline{2-2}
4 & \\
  & {\code y} \\
\cline{2-2}
8 & \\
  & \\
  & {\code z} \\
  & \\
\cline{2-2}
\end{tabular}
\caption{Structure S \label{fig:structPic2}}

\end{figure}
\index{structures!alignment|(}
If one uses the {\code offsetof} macro to find the offset of {\code y}
using the \emph{gcc} compiler, they will find that it returns 4, not
2!  Why?  \MarginNote{Recall that an address is on a double word
boundary if it is divisible by 4} Because \emph{gcc} (and many other
compilers) align variables on double word boundaries by default. In
32-bit protected mode, the CPU reads memory faster if the data starts
at a double word boundary. Figure~\ref{fig:structPic2} shows how the
{\code S} structure really looks using \emph{gcc}. The compiler
inserts two unused bytes into the structure to align {\code y} (and
{\code z}) on a double word boundary. This shows why it is a good idea
to use {\code offsetof} to compute the offsets instead of calculating
them oneself when using structures defined in C.

Of course, if the structure is only used in assembly, the programmer can 
determine the offsets himself. However, if one is interfacing C and assembly,
it is very important that both the assembly code and the C code agree on
the offsets of the elements of the structure! One complication is that 
different C compilers may give different offsets to the elements. For example,
as we have seen, the \emph{gcc} compiler creates an {\code S} structure that
looks like Figure~\ref{fig:structPic2}; however, Borland's compiler would
create a structure that looks like Figure~\ref{fig:structPic1}. C compilers
provide ways to specify the alignment used for data. However, the ANSI C
standard does not specify how this will be done and thus, different compilers
do it differently.  



%Borland's compiler has a flag, {\code -a}, that can be
%used to define the alignment used for all data. Compiling with {\code -a 4}
%tells \emph{bcc} to use double word alignment. Microsoft's compiler 
%provides a {\code \#pragma pack} directive that can be used to set
%the alignment (consult Microsoft's documentation for details). Borland's
%compiler also supports Microsoft's pragma 

The \emph{gcc}\index{compiler!gcc!\_\_attribute\_\_} compiler has a flexible and complicated method of
specifying the alignment. The compiler allows one to specify the
alignment of any type using a special syntax. For example, the
following line:
\begin{lstlisting}[stepnumber=0]{}
  typedef short int unaligned_int __attribute__((aligned(1)));
\end{lstlisting}
\noindent defines a new type named {\code unaligned\_int} that
is aligned on byte boundaries. (Yes, all the parenthesis after {\code
\_\_attribute\_\_} are required!)  The 1 in the {\code aligned}
parameter can be replaced with other powers of two to specify other
alignments. (2 for word alignment, 4 for double word alignment,
\emph{etc.}) If the {\code y} element of the structure was changed to be an
{\code unaligned\_int} type, \emph{gcc} would put {\code y} at offset 2.
However, {\code z} would still be at offset 8 since doubles are also double
word aligned by default. The definition of {\code z}'s type would have
to be changed as well for it to put at offset 6.

\begin{figure}[t]
\begin{lstlisting}[frame=tlrb,stepnumber=0]{}
struct S {
  short int x;    /* 2-byte integer */
  int       y;    /* 4-byte integer */
  double    z;    /* 8-byte float   */
} __attribute__((packed));
\end{lstlisting}
\caption{Packed struct using \emph{gcc} \label{fig:packedStruct}\index{compiler!gcc!\_\_attribute\_\_}}
\end{figure}

The \emph{gcc} compiler also allows one to \emph{pack} a structure. This
tells the compiler to use the minimum possible space for the structure.
Figure~\ref{fig:packedStruct} shows how {\code S} could be rewritten this way.
This form of {\code S} would use the minimum bytes possible, 14 bytes.

Microsoft's and Borland's compilers both support the same method of specifying
alignment using a {\code \#pragma} directive.\index{compiler!Microsoft!pragma pack}
\begin{lstlisting}[stepnumber=0]{}
#pragma pack(1)
\end{lstlisting}
The directive above tells the compiler to pack elements of structures
on byte boundaries (\emph{i.e.}, with no extra padding). The one can
be replaced with two, four, eight or sixteen to specify alignment on
word, double word, quad word and paragraph boundaries,
respectively. The directive stays in effect until overridden by
another directive. This can cause problems since these directives are
often used in header files. If the header file is included before
other header files with structures, these structures may be laid out
differently than they would by default. This can lead to a very hard to
find error. Different modules of a program might lay out the elements
of the structures in \emph{different} places!

\begin{figure}[t]
\begin{lstlisting}[frame=tlrb,stepnumber=0]{}
#pragma pack(push)    /* save alignment state */
#pragma pack(1)       /* set byte alignment   */

struct S {
  short int x;    /* 2-byte integer */
  int       y;    /* 4-byte integer */
  double    z;    /* 8-byte float   */
};

#pragma pack(pop)     /* restore original alignment */
\end{lstlisting}
\caption{Packed struct using Microsoft or Borland \label{fig:msPacked}\index{compiler!Microsoft!pragma pack}}
\end{figure}

There is a way to avoid this problem. Microsoft and Borland support a
way to save the current alignment state and restore it
later. Figure~\ref{fig:msPacked} shows how this would be done.
\index{structures!alignment|)}

\subsection{Bit Fields\index{structures!bit fields|(}}

\begin{figure}[t]
\begin{lstlisting}[frame=tlrb,stepnumber=0]{}
struct S {
  unsigned f1 : 3;   /* 3-bit field  */
  unsigned f2 : 10;  /* 10-bit field */
  unsigned f3 : 11;  /* 11-bit field */
  unsigned f4 : 8;   /* 8-bit field  */
};
\end{lstlisting}
\caption{Bit Field Example \label{fig:bitStruct}}
\end{figure}

Bit fields allow one to specify members of a struct that only use a specified
number of bits. The size of bits does not have to be a multiple of eight. A
bit field member is defined like an \lstinline|unsigned int| or \lstinline|int|
member with a colon and bit size appended to it. Figure~\ref{fig:bitStruct}
shows an example. This defines a 32-bit variable that is decomposed in the 
following parts:
\begin{center}
\begin{tabular}{|c|c|c|c|}
\multicolumn{1}{c}{8 bits} & \multicolumn{1}{c}{11 bits} 
& \multicolumn{1}{c}{10 bits} & \multicolumn{1}{c}{3 bits} \\ \hline
\hspace{2em} f4 \hspace{2em} & \hspace{3em} f3 \hspace{3em}
& \hspace{3em} f2 \hspace{3em} & f1 \\
\hline
\end{tabular}
\end{center}
The first bitfield is assigned to the least significant bits of its
double word.\footnote{Actually, the ANSI/ISO C standard gives the
compiler some flexibility in exactly how the bits are laid
out. However, common C compilers (\emph{gcc}, \emph{Microsoft} and
\emph{Borland}) will lay the fields out like this.}

However, the format is not so simple if one looks at how the bits are actually
stored in memory. The difficulty occurs when bitfields span byte boundaries.
Because the bytes on a little endian processor will be reversed in memory. For
example, the {\code S} struct bitfields will look like this in memory:
\begin{center}
\begin{tabular}{|c|c||c|c||c||c|}
\multicolumn{1}{c}{5 bits} & \multicolumn{1}{c}{3 bits} 
& \multicolumn{1}{c}{3 bits} & \multicolumn{1}{c}{5 bits} 
& \multicolumn{1}{c}{8 bits} & \multicolumn{1}{c}{8 bits} \\ \hline
f2l & f1 &  f3l  & f2m & \hspace{1em} f3m \hspace{1em} 
& \hspace{1.5em} f4 \hspace{1.5em} \\
\hline
\end{tabular}
\end{center}
The \emph{f2l} label refers to the last five bits (\emph{i.e.}, the five least
significant bits) of the \emph{f2} bit field. The \emph{f2m} label refers to the
five most significant bits of \emph{f2}. The double vertical lines show the byte
boundaries. If one reverses all the bytes, the pieces of the \emph{f2} and \emph{f3}
fields will be reunited in the correct place.

\begin{figure}[t]
\centering
\begin{tabular}{|c*{8}{|p{1.3em}}|}
\hline
Byte $\backslash$ Bit & 7 & 6 & 5 & 4 & 3 & 2 & 1 & 0 \\ \hline
0 & \multicolumn{8}{c|}{Operation Code (08h) } \\ \hline
1 & \multicolumn{3}{c|}{Logical Unit \# } & \multicolumn{5}{c|}{msb of LBA} \\ \hline
2 & \multicolumn{8}{c|}{middle of Logical Block Address} \\ \hline
3 & \multicolumn{8}{c|}{lsb of Logicial Block Address} \\ \hline
4 & \multicolumn{8}{c|}{Transfer Length} \\ \hline
5 & \multicolumn{8}{c|}{Control} \\ \hline
\end{tabular}
\caption{SCSI Read Command Format \label{fig:scsi-read}}
\end{figure}

\begin{figure}[t]
\begin{lstlisting}[frame=lrtb]{}
#define MS_OR_BORLAND (defined(__BORLANDC__) \
                        || defined(_MSC_VER))

#if MS_OR_BORLAND
#  pragma pack(push)
#  pragma pack(1)
#endif

struct SCSI_read_cmd {
  unsigned opcode : 8;
  unsigned lba_msb : 5;
  unsigned logical_unit : 3;
  unsigned lba_mid : 8;    /* middle bits */
  unsigned lba_lsb : 8;
  unsigned transfer_length : 8;
  unsigned control : 8;
}
#if defined(__GNUC__)
   __attribute__((packed))
#endif
;

#if MS_OR_BORLAND
#  pragma pack(pop)
#endif
\end{lstlisting}
\caption{SCSI Read Command Format Structure\label{fig:scsi-read-struct}\index{compiler!gcc!\_\_attribute\_\_}
         \index{compiler!Microsoft!pragma pack}}
\end{figure}

The physical memory layout is not usually important unless the data is being 
transfered in or out of the program (which is actually quite common with bit fields).
It is common for hardware devices interfaces to use odd number of bits that 
bitfields could be useful to represent. 


\begin{figure}[t]
\centering
\begin{tabular}{|c||c||c||c||c|c||c|}
\multicolumn{1}{c}{8 bits} & \multicolumn{1}{c}{8 bits} 
& \multicolumn{1}{c}{8 bits} & \multicolumn{1}{c}{8 bits} 
& \multicolumn{1}{c}{3 bits} & \multicolumn{1}{c}{5 bits} 
& \multicolumn{1}{c}{8 bits} \\ \hline
control & transfer\_length & lba\_lsb  & lba\_mid &  
logical\_unit  & lba\_msb & opcode \\
\hline
\end{tabular}
\caption{Mapping of {\code SCSI\_read\_cmd} fields \label{fig:scsi-read-map}}
\end{figure}
\index{SCSI|(}
One example is SCSI\footnote{Small Computer Systems Interface, an industry standard 
for hard disks, \emph{etc.}}. A direct read command for a SCSI device is specified
by sending a six byte message to the device in the format specified in 
Figure~\ref{fig:scsi-read}. The difficulty representing this using bitfields is the
\emph{logical block address} which spans 3 different bytes of the command. From
Figure~\ref{fig:scsi-read}, one sees that the data is stored in big endian format.
Figure~\ref{fig:scsi-read-struct} shows a definition that attempts to work with
all compilers. The first two lines define a macro that is true if the code is
compiled with the Microsoft or Borland compilers. The potentially confusing parts
are lines 11 to 14. First one might wonder why the \lstinline|lba_mid| and
\lstinline|lba_lsb| fields are defined separately and not as a single 16-bit
field? The reason is that the data is in big endian order. A 16-bit
field would be stored in little endian order by the compiler. Next,
the \lstinline|lba_msb| and \lstinline|logical_unit| fields appear to
be reversed; however, this is not the case. They have to be put in
this order. Figure~\ref{fig:scsi-read-map} shows how the fields are
mapped as a 48-bit entity. (The byte boundaries are again denoted by
the double lines.) When this is stored in memory in little endian
order, the bits are arranged in the desired format
(Figure~\ref{fig:scsi-read}).

\begin{figure}[t]
\begin{lstlisting}[frame=lrtb]{}
struct SCSI_read_cmd {
  unsigned char opcode;
  unsigned char lba_msb : 5;
  unsigned char logical_unit : 3;
  unsigned char lba_mid;    /* middle bits */
  unsigned char lba_lsb;
  unsigned char transfer_length;
  unsigned char control;
}
#if defined(__GNUC__)
   __attribute__((packed))
#endif
;
\end{lstlisting}
\caption{Alternate SCSI Read Command Format Structure\label{fig:scsi-read-struct2}
         \index{compiler!gcc!\_\_attribute\_\_}\index{compiler!Microsoft!pragma pack}}
\end{figure}

To complicate matters more, the definition for the
\lstinline|SCSI_read_cmd| does not quite work correctly for Microsoft
C. If the \lstinline|sizeof(SCSI_read_cmd)| expression is evalutated,
Microsoft C will return 8, not 6! This is because the Microsoft
compiler uses the type of the bitfield in determining how to map the
bits. Since all the bit fields are defined as \lstinline|unsigned|
types, the compiler pads two bytes at the end of the structure to make
it an integral number of double words. This can be remedied by making
all the fields \lstinline|unsigned short| instead. Now, the Microsoft
compiler does not need to add any pad bytes since six bytes is an
integral number of two-byte words.\footnote{Mixing different types of
bit fields leads to very confusing behavior! The reader is invited to
experiment.} The other compilers also work correctly with this
change. Figure~\ref{fig:scsi-read-struct2} shows yet another definition
that works for all three compilers. It avoids all but two of the bit
fields by using \lstinline|unsigned char|.
\index{SCSI|)}

The reader should not be discouraged if he found the previous
discussion confusing.  It is confusing! The author often finds it less
confusing to avoid bit fields altogether and use bit operations to
examine and modify the bits manually.

\index{structures!bit fields|)}

%TODO:discuss alignment issues and struct size issues

\subsection{Using structures in assembly}

As discussed above, accessing a structure in assembly is very much
like accessing an array. For a simple example, consider how one would
write an assembly routine that would zero out the {\code y} element
of an {\code S} structure. Assuming the prototype of the routine would be:
\begin{lstlisting}[stepnumber=0]{}
void zero_y( S * s_p );
\end{lstlisting}
\noindent the assembly routine would be:
\begin{AsmCodeListing}
%define      y_offset  4
_zero_y:
      enter  0,0
      mov    eax, [ebp + 8]      ; get s_p (struct pointer) from stack
      mov    dword [eax + y_offset], 0
      leave
      ret
\end{AsmCodeListing}

C allows one to pass a structure by value to a function; however, this is
almost always a bad idea. When passed by value, the entire data in the
structure must be copied to the stack and then retrieved by the routine.
It is much more efficient to pass a pointer to a structure instead.

C also allows a structure type to be used as the return value of a function.
Obviously a structure can not be returned in the {\code EAX} register. Different
compilers handle this situation differently. A common solution that compilers
use is to internally rewrite the function as one that takes a structure pointer
as a parameter. The pointer is used to put the return value into a structure
defined outside of the routine called.

Most assemblers (including NASM) have built-in support for defining structures
in your assembly code. Consult your documentation for details.

% add section on structure return values for functions

\index{structures|)}

\section{Assembly and C++\index{C++|(}}

The C++ programming language is an extension of the C language. Many of the
basic rules of interfacing C and assembly language also apply to C++.
However, some rules need to be modified. Also, some of the extensions
of C++ are easier to understand with a knowledge of assembly language.
This section assumes a basic knowledge of C++.

\subsection{Overloading and Name Mangling\index{C++!name mangling|(}}
\label{subsec:mangling}
\begin{figure}
\centering
\begin{lstlisting}[frame=tlrb]{}
#include <stdio.h>

void f( int x )
{
  printf("%d\n", x);
}

void f( double x )
{
  printf("%g\n", x);
}
\end{lstlisting}
\caption{Two {\code f()} functions \label{fig:twof}}
\end{figure}

C++ allows different functions (and class member functions) with the same
name to be defined. When more than one function share the same name, the
functions are said to be \emph{overloaded}. If two functions are defined
with the same name in C, the linker will produce an error because it will
find two definitions for the same symbol in the object files it is linking.
For example, consider the code in Figure~\ref{fig:twof}. The equivalent
assembly code would define two labels named {\code \_f} which will obviously
be an error.

C++ uses the same linking process as C, but avoids this error by
performing \emph{name mangling} or modifying the symbol used to label
the function. In a way, C already uses name mangling, too. It adds an
underscore to the name of the C function when creating the label for
the function. However, C will mangle the name of both functions in
Figure~\ref{fig:twof} the same way and produce an error. C++ uses a
more sophisticated mangling process that produces two different labels
for the functions. For example, the first function in
Figure~\ref{fig:twof} would be assigned by DJGPP the label {\code
\_f\_\_Fi} and the second function, {\code \_f\_\_Fd}. This avoids any
linker errors.
% check to make sure that DJGPP does still but an _ at beginning for C++

Unfortunately, there is no standard for how to manage names in C++ and
different compilers mangle names differently. For example, Borland C++ would
use the labels {\code @f\$qi} and {\code @f\$qd} for the two functions
in Figure~\ref{fig:twof}. However, the rules are not completely arbitrary.
The mangled name encodes the \emph{signature} of the function. The signature
of a function is defined by the order and the type of its parameters. 
Notice that the function that takes a single {\code int} argument has an
\emph{i} at the end of its mangled name (for both DJGPP and Borland) and that
the one that takes a {\code double} argument has a \emph{d} at the end of
its mangled name. If there was a function named {\code f} with the
prototype:
\begin{lstlisting}[stepnumber=0]{}
  void f( int x, int y, double z);
\end{lstlisting}
\noindent DJGPP would mangle its name to be {\code \_f\_\_Fiid} and Borland to
{\code @f\$qiid}.

The return type of the function is \emph{not} part of a function's
signature and is not encoded in its mangled name. This fact explains a
rule of overloading in C++. Only functions whose signatures are unique
may be overloaded. As one can see, if two functions with the same name
and signature are defined in C++, they will produce the same mangled
name and will create a linker error. By default, all C++ functions are
name mangled, even ones that are not overloaded. When it is compiling
a file, the compiler has no way of knowing whether a particular
function is overloaded or not, so it mangles all names. In fact, it
also mangles the names of global variables by encoding the type of the
variable in a similar way as function signatures. Thus, if one defines
a global variable in one file as a certain type and then tries to use
it in another file as the wrong type, a linker error will be
produced. This characteristic of C++ is known as \emph{typesafe
linking}. \index{C++!typesafe linking}It also exposes another type of
error, inconsistent prototypes. This occurs when the definition of a
function in one module does not agree with the prototype used by
another module. In C, this can be a very difficult problem to debug. C
does not catch this error. The program will compile and link, but will
have undefined behavior as the calling code will be pushing different
types on the stack than the function expects. In C++, it will produce
a linker error.

When the C++ compiler is parsing a function call, it looks for a
matching function by looking at the types of the arguments passed to the
function\footnote{The match does not have to be an exact match, the compiler
will consider matches made by casting the arguments. The rules for this
process are beyond the scope of this book. Consult a C++ book for details.}.
If it finds a match, it then creates a {\code CALL} to the correct function
using the compiler's name mangling rules.

Since different compilers use different name mangling rules, C++ code
compiled by different compilers may not be able to be linked
together. This fact is important when considering using a precompiled
C++ library! If one wishes to write a function in assembly that will
be used with C++ code, she must know the name mangling rules for the
C++ compiler to be used (or use the technique explained below).

The astute student may question whether the code in Figure~\ref{fig:twof}
will work as expected. Since C++ name mangles all functions, then the
{\code printf} function will be mangled and the compiler will not produce
a {\code CALL} to the label {\code \_printf}. This is a valid concern!
If the prototype for {\code printf} was simply placed at the top of the file,
this would happen. The prototype is:
\begin{lstlisting}[stepnumber=0]{}
  int printf( const char *, ...);
\end{lstlisting}
\noindent DJGPP would mangle this to be {\code
\_printf\_\_FPCce}. (The {\code F} is for \emph{function}, {\code P}
for \emph{pointer}, {\code C} for \emph{const}, {\code c} for
\emph{char} and {\code e} for ellipsis.) This would not call the
regular C library's {\code printf} function! Of course, there must be
a way for C++ code to call C code. This is very important because
there is \emph{a lot} of useful old C code around.  In addition to
allowing one to call legacy C code, C++ also allows one to call
assembly code using the normal C mangling conventions.

\index{C++!extern ""C""|(}
C++ extends the {\code extern} keyword to allow it to specify that the
function or global variable it modifies uses the normal C conventions.
In C++ terminology, the function or global variable uses \emph{C
linkage}. For example, to declare {\code printf} to have C linkage,
use the prototype:
\begin{lstlisting}[language=C++,stepnumber=0]{}
extern "C" int printf( const char *, ... );
\end{lstlisting}
\noindent This instructs the compiler not to use the C++ name mangling
rules on this function, but instead to use the C rules. However, by
doing this, the {\code printf} function may not be overloaded. This provides
the easiest way to interface C++ and assembly, define the function to
use C linkage and then use the C calling convention.

For convenience, C++ also allows the linkage of a block of functions
and global variables to be defined. The block is denoted by the
usual curly braces.
\begin{lstlisting}[stepnumber=0,language=C++]{}
extern "C" {
  /* C linkage global variables and function prototypes */
}
\end{lstlisting}

If one examines the ANSI C header files that come with C/C++ compilers
today, they will find the following near the top of each header file:
\begin{lstlisting}[stepnumber=0,language=C++]{}
#ifdef __cplusplus
extern "C" {
#endif
\end{lstlisting}
\noindent And a similar construction near the bottom containing a
closing curly brace.  C++ compilers define the {\code \_\_cplusplus}
macro (with \emph{two} leading underscores). The snippet above
encloses the entire header file within an {\code extern~"C"} block if
the header file is compiled as C++, but does nothing if compiled as C
(since a C compiler would give a syntax error for {\code extern~"C"}).
This same technique can be used by any programmer to create a header
file for assembly routines that can be used with either C or C++.
\index{C++!extern ""C""|)}
\index{C++!name mangling|)}

\begin{figure}
\begin{lstlisting}[language=C++,frame=tlrb]{}
void f( int & x )     // the & denotes a reference parameter
{ x++; }

int main()
{
  int y = 5;
  f(y);               // reference to y is passed, note no & here!
  printf("%d\n", y);  // prints out 6!
  return 0;
}
\end{lstlisting}
\caption{Reference example \label{fig:refex}}
\end{figure}

\subsection{References\index{C++!references|(}}

\emph{References} are another new feature of C++. They allow one to
pass parameters to functions without explicitly using pointers. For
example, consider the code in Figure~\ref{fig:refex}. Actually,
reference parameters are pretty simple, they really are just pointers.
The compiler just hides this from the programmer (just as Pascal compilers
implement {\code var} parameters as pointers). When the compiler generates
assembly for the function call on line~7, it passes the \emph{address} of 
{\code y}. If one was writing function {\code f} in assembly, they would
act as if the prototype was\footnote{Of course, they might want to
declare the function with C linkage to avoid name mangling as discussed
in Section~\ref{subsec:mangling}}:
\begin{lstlisting}[stepnumber=0]{}
  void f( int * xp);
\end{lstlisting}

References are just a convenience that are especially useful for
operator overloading. This is another feature of C++ that allows one
to define meanings for common operators on structure or class
types. For example, a common use is to define the plus ({\code +})
operator to concatenate string objects. Thus, if {\code a} and {\code
b} were strings, {\code a~+~b} would return the concatenation of the
strings {\code a} and {\code b}. C++ would actually call a function to
do this (in fact, this expression could be rewritten in function
notation as {\code operator~+(a,b)}).  For efficiency, one would like
to pass the address of the string objects instead of passing them by
value. Without references, this could be done as {\code 
operator~+(\&a,\&b)}, but this would require one to write in operator
syntax as {\code \&a~+~\&b}. This would be very awkward and confusing.
However, by using references, one can write it as {\code a~+~b}, which
looks very natural.
\index{C++!references|)}

\subsection{Inline functions\index{C++!inline functions|(}}

\emph{Inline functions} are yet another feature of C++\footnote{
C compilers often support this feature as an extension
of ANSI C.}. Inline functions are meant to replace the error-prone,
preprocessor-based macros that take parameters. Recall from C, that
writing a macro that squares a number might look like:
\begin{lstlisting}[stepnumber=0]{}
#define SQR(x) ((x)*(x))
\end{lstlisting}
\noindent Because the preprocessor does not understand C and does
simple substitutions, the parenthesis are required to compute the correct
answer in most cases. However, even this version will not give the correct
answer for {\code SQR(x++)}.

\begin{figure}
\begin{lstlisting}[language=C++,frame=tlrb]{}
inline int inline_f( int x ) 
{ return x*x; }

int f( int x ) 
{ return x*x; }

int main()
{
  int y, x = 5;
  y = f(x);
  y = inline_f(x);
  return 0;
}
\end{lstlisting}
\caption{Inlining example \label{fig:InlineFun}}
\end{figure}


Macros are used because they eliminate the overhead of making a
function call for a simple function. As the chapter on subprograms
demonstrated, performing a function call involves several steps. For a
very simple function, the time it takes to make the function call may
be more than the time to actually perform the operations in the
function! Inline functions are a much more friendly way to write code
that looks like a normal function, but that does \emph{not} {\code
CALL} a common block of code. Instead, calls to inline functions are
replaced by code that performs the function.  C++ allows a function to
be made inline by placing the keyword {\code inline} in front of the
function definition. For example, consider the functions declared in
Figure~\ref{fig:InlineFun}. The call to function {\code f} on line~10
does a normal function call (in assembly, assuming {\code x} is at
address {\code ebp-8} and {\code y} is at {\code ebp-4}):
\begin{AsmCodeListing}
      push   dword [ebp-8]
      call   _f
      pop    ecx
      mov    [ebp-4], eax
\end{AsmCodeListing}
However, the call to function {\code inline\_f} on line~11 would look like:
\begin{AsmCodeListing}
      mov    eax, [ebp-8]
      imul   eax, eax
      mov    [ebp-4], eax
\end{AsmCodeListing}

In this case, there are two advantages to inlining. First, the inline function
is faster. No parameters are pushed on the stack, no stack frame is
created and then destroyed, no branch is made. Secondly, the inline function
call uses less code! This last point is true for this example, but does not
hold true in all cases.

The main disadvantage of inlining is that inline code is not linked
and so the code of an inline function must be available to \emph{all}
files that use it. The previous example assembly code shows this. The
call of the non-inline function only requires knowledge of the
parameters, the return value type, calling convention and the name of
the label for the function.  All this information is available from
the prototype of the function. However, using the inline function
requires knowledge of the all the code of the function. This means
that if \emph{any} part of an inline function is changed, \emph{all}
source files that use the function must be recompiled. Recall that for
non-inline functions, if the prototype does not change, often the
files that use the function need not be recompiled. For all these
reasons, the code for inline functions are usually placed in header
files. This practice is contrary to the normal hard and fast rule in C
that executable code statements are \emph{never} placed in header
files.
\index{C++!inline functions|)}

\begin{figure}[t]
\begin{lstlisting}[language=C++,frame=tlrb]{}
class Simple {
public:
  Simple();                // default constructor
  ~Simple();               // destructor
  int get_data() const;    // member functions
  void set_data( int );
private:
  int data;                // member data
};

Simple::Simple()
{ data = 0; }

Simple::~Simple()
{ /* null body */ }

int Simple::get_data() const
{ return data; }

void Simple::set_data( int x )
{ data = x; }
\end{lstlisting}
\caption{A simple C++ class\label{fig:SimpleClass}}
\end{figure}

\subsection{Classes\index{C++!classes|(}}

A C++ class describes a type of \emph{object}. An object has both data
members and function members\footnote{Often called \emph{member
functions} in C++ or more generally \emph{methods}\index{methods}.}. In other words,
it's a {\code struct} with data and functions associated with
it. Consider the simple class defined in
Figure~\ref{fig:SimpleClass}. A variable of {\code Simple} type would
look just like a normal C {\code struct} with a single {\code int}
member. \MarginNote{Actually, C++ uses the {\code this} keyword to
access the pointer to the object acted on from inside the member
function.}  The functions are \emph{not} stored in memory assigned to
the structure. However, member functions are different from other
functions. They are passed a \emph{hidden} parameter. This parameter
is a pointer to the object that the member function is acting on.

\begin{figure}[t]
\begin{lstlisting}[stepnumber=0]{}
void set_data( Simple * object, int x )
{
  object->data = x;
}
\end{lstlisting}
\caption{C Version of Simple::set\_data()\label{fig:SimpleCVer}}
\end{figure}


\begin{figure}[t]
\begin{AsmCodeListing}
_set_data__6Simplei:           ; mangled name
      push   ebp
      mov    ebp, esp

      mov    eax, [ebp + 8]   ; eax = pointer to object (this)
      mov    edx, [ebp + 12]  ; edx = integer parameter
      mov    [eax], edx       ; data is at offset 0

      leave
      ret
\end{AsmCodeListing}
\caption{Compiler output of Simple::set\_data( int ) \label{fig:SimpleAsm}}
\end{figure}


For example, consider the {\code set\_data} method of the {\code
Simple} class of Figure~\ref{fig:SimpleClass}. If it was written in C,
it would look like a function that was explicitly passed a
pointer to the object being acted on as the code in
Figure~\ref{fig:SimpleCVer} shows.  The {\code -S} switch on the
\emph{DJGPP} compiler (and the \emph{gcc} and Borland compilers as
well) tells the compiler to produce an assembly file containing the
equivalent assembly language for the code produced.  For \emph{DJGPP}
and \emph{gcc} the assembly file ends in an {\code .s} extension and
unfortunately uses AT\&T assembly language syntax which is quite
different from NASM and MASM syntaxes\footnote{The \emph{gcc}
compiler system includes its own assembler called \emph{gas}\index{gas}. The
\emph{gas} assembler uses AT\&T syntax and thus the compiler outputs
the code in the format for \emph{gas}. There are several pages on the
web that discuss the differences in INTEL and AT\&T formats. There is
also a free program named {\code a2i}
({http://www.multimania.com/placr/a2i.html}), that converts AT\&T
format to NASM format.}. (Borland and MS compilers generate a file
with a {\code .asm} extension using MASM syntax.)
Figure~\ref{fig:SimpleAsm} shows the output of \emph{DJGPP} converted
to NASM syntax and with comments added to clarify the purpose of the
statements. On the very first line, note that the {\code set\_data}
method is assigned a mangled label that encodes the name of the
method, the name of the class and the parameters. The name of the
class is encoded because other classes might have a method named
{\code set\_data} and the two methods \emph{must} be assigned
different labels. The parameters are encoded so that the class can
overload the {\code set\_data} method to take other parameters just as
normal C++ functions. However, just as before, different compilers
will encode this information differently in the mangled label.

Next on lines~2 and 3, the familiar function prologue appears. On
line~5, the first parameter on the stack is stored into {\code
EAX}. This is \emph{not} the {\code x} parameter! Instead it is the
hidden parameter\footnote{As usual, \emph{nothing} is hidden in the
assembly code!} that points to the object being acted on. Line~6
stores the {\code x} parameter into {\code EDX} and line~7 stores
{\code EDX} into the double word that {\code EAX} points to. This is
the {\code data} member of the {\code Simple} object being acted on,
which being the only data in the class, is stored at offset 0 in the
{\code Simple} structure.

\begin{figure}[tp]
\begin{lstlisting}[frame=tlrb,language=C++]{}
class Big_int {
public:
   /* 
   * Parameters:
   *   size           - size of integer expressed as number of 
   *                    normal unsigned int's
   *   initial_value  - initial value of Big_int as a normal unsigned int
   */
  explicit Big_int( size_t   size,
                    unsigned initial_value = 0);
  /*
   * Parameters:
   *   size           - size of integer expressed as number of 
   *                    normal unsigned int's
   *   initial_value  - initial value of Big_int as a string holding
   *                    hexadecimal representation of value. 
   */
  Big_int( size_t       size,
           const char * initial_value);

  Big_int( const Big_int & big_int_to_copy);
  ~Big_int();

  // returns size of Big_int (in terms of unsigned int's)
  size_t size() const;

  const Big_int & operator = ( const Big_int & big_int_to_copy);
  friend Big_int operator + ( const Big_int & op1,
                              const Big_int & op2 );
  friend Big_int operator - ( const Big_int & op1,
                              const Big_int & op2);
  friend bool operator == ( const Big_int & op1,
                            const Big_int & op2 );
  friend bool operator < ( const Big_int & op1,
                           const Big_int & op2);
  friend ostream & operator << ( ostream &       os,
                                 const Big_int & op );
private:
  size_t      size_;    // size of unsigned array
  unsigned *  number_;  // pointer to unsigned array holding value
};
\end{lstlisting}
\caption{Definition of Big\_int class\label{fig:BigIntClass}}
\end{figure}

\begin{figure}[tp]
\begin{lstlisting}[frame=tlrb,language=C++]{}
// prototypes for assembly routines
extern "C" {
  int add_big_ints( Big_int &       res, 
                    const Big_int & op1, 
                    const Big_int & op2);
  int sub_big_ints( Big_int &       res, 
                    const Big_int & op1, 
                    const Big_int & op2);
}

inline Big_int operator + ( const Big_int & op1, const Big_int & op2)
{
  Big_int result(op1.size());
  int res = add_big_ints(result, op1, op2);
  if (res == 1)
    throw Big_int::Overflow();
  if (res == 2)
    throw Big_int::Size_mismatch();
  return result;
}

inline Big_int operator - ( const Big_int & op1, const Big_int & op2)
{
  Big_int result(op1.size());
  int res = sub_big_ints(result, op1, op2);
  if (res == 1)
    throw Big_int::Overflow();
  if (res == 2)
    throw Big_int::Size_mismatch();
  return result;
}
\end{lstlisting}
\caption{Big\_int Class Arithmetic Code\label{fig:BigIntAdd}}
\end{figure}

\subsubsection{Example}
\index{C++!Big\_int example|(}
This section uses the ideas of the chapter to create a C++ class that
represents an unsigned integer of arbitrary size. Since the integer
can be any size, it will be stored in an array of unsigned integers
(double words). It can be made any size by using dynamical
allocation. The double words are stored in reverse order\footnote{Why?
Because addition operations will then always start processing at the
beginning of the array and move forward.}  (\emph{i.e.} the least
significant double word is at index 0).  Figure~\ref{fig:BigIntClass}
shows the definition of the {\code Big\_int} class\footnote{See the
code example source for the complete code for this example. The text
will only refer to some of the code.}. The size of a {\code Big\_int}
is measured by the size of the {\code unsigned} array that is used to
store its data. The {\code size\_} data member of the class is
assigned offset zero and the {\code number\_} member is assigned
offset 4.

To simplify these example, only object instances with the same size
arrays can be added to or subtracted from each other.

The class has three constructors: the first (line~9) initializes the
class instance by using a normal unsigned integer; the second
(line~18) initializes the instance by using a string that contains a
hexadecimal value. The third constructor (line~21) is the \emph{copy
constructor}\index{C++!copy constructor}.

This discussion focuses on how the addition and subtraction operators
work since this is where the assembly language is
used. Figure~\ref{fig:BigIntAdd} shows the relevant parts of the
header file for these operators. They show how the operators are set
up to call the assembly routines. Since different compilers use
radically different mangling rules for operator functions, inline
operator functions are used to set up calls to C linkage assembly
routines. This makes it relatively easy to port to different compilers
and is just as fast as direct calls. This technique also eliminates the
need to throw an exception from assembly!

Why is assembly used at all here? Recall that to perform multiple
precision arithmetic, the carry must be moved from one dword to be
added to the next significant dword. C++ (and C) do not allow the
programmer to access the CPU's carry flag. Performing the addition could
only be done by having C++ independently recalculate the carry flag
and conditionally add it to the next dword. It is much more efficient
to write the code in assembly where the carry flag can be accessed and
using the {\code ADC} instruction which automatically adds the carry
flag in makes a lot of sense.

For brevity, only the {\code add\_big\_ints} assembly routine will be discussed
here. Below is the code for this routine (from {\code big\_math.asm}):
\begin{AsmCodeListing}[label=big\_math.asm]
segment .text
        global  add_big_ints, sub_big_ints
%define size_offset 0
%define number_offset 4

%define EXIT_OK 0
%define EXIT_OVERFLOW 1
%define EXIT_SIZE_MISMATCH 2

; Parameters for both add and sub routines
%define res ebp+8
%define op1 ebp+12
%define op2 ebp+16

add_big_ints:
        push    ebp
        mov     ebp, esp
        push    ebx
        push    esi
        push    edi
        ;
        ; first set up esi to point to op1
        ;              edi to point to op2
        ;              ebx to point to res
        mov     esi, [op1]
        mov     edi, [op2]
        mov     ebx, [res]
        ;
        ; make sure that all 3 Big_int's have the same size
        ;
        mov     eax, [esi + size_offset]
        cmp     eax, [edi + size_offset]
        jne     sizes_not_equal                 ; op1.size_ != op2.size_
        cmp     eax, [ebx + size_offset]
        jne     sizes_not_equal                 ; op1.size_ != res.size_

        mov     ecx, eax                        ; ecx = size of Big_int's
        ;
        ; now, set registers to point to their respective arrays
        ;      esi = op1.number_
        ;      edi = op2.number_
        ;      ebx = res.number_
        ;
        mov     ebx, [ebx + number_offset]
        mov     esi, [esi + number_offset]
        mov     edi, [edi + number_offset]
        
        clc                                     ; clear carry flag
        xor     edx, edx                        ; edx = 0
        ;
        ; addition loop
add_loop:
        mov     eax, [edi+4*edx]
        adc     eax, [esi+4*edx]
        mov     [ebx + 4*edx], eax
        inc     edx                             ; does not alter carry flag
        loop    add_loop

        jc      overflow
ok_done:
        xor     eax, eax                        ; return value = EXIT_OK
        jmp     done
overflow:
        mov     eax, EXIT_OVERFLOW
        jmp     done
sizes_not_equal:
        mov     eax, EXIT_SIZE_MISMATCH
done:
        pop     edi
        pop     esi
        pop     ebx
        leave
        ret
\end{AsmCodeListing}

Hopefully, most of this code should be straightforward to the reader
by now. Lines~25 to 27 store pointers to the {\code Big\_int} objects
passed to the function into registers. Remember that references really
are just pointers.  Lines~31 to 35 check to make sure that the sizes
of the three objects's arrays are the same. (Note that the offset of
{\code size\_} is added to the pointer to access the data member.)
Lines~44 to 46 adjust the registers to point to the array used by the
respective objects instead of the objects themselves. (Again, the
offset of the {\code number\_} member is added to the object pointer.)

\begin{figure}[tp]
\begin{lstlisting}[language=C++, frame=tlrb]{}
#include "big_int.hpp"
#include <iostream>
using namespace std;

int main()
{
  try {
    Big_int b(5,"8000000000000a00b");
    Big_int a(5,"80000000000010230");
    Big_int c = a + b;
    cout << a << " + " << b << " = " << c << endl;
    for( int i=0; i < 2; i++ ) {
      c = c + a;
      cout << "c = " << c << endl;
    }
    cout << "c-1 = " << c - Big_int(5,1) << endl;
    Big_int d(5, "12345678");
    cout << "d = " << d << endl;
    cout << "c == d " << (c == d) << endl;
    cout << "c > d " << (c > d) << endl;
  }
  catch( const char * str ) {
    cerr << "Caught: " << str << endl;
  }
  catch( Big_int::Overflow ) {
    cerr << "Overflow" << endl;
  }
  catch( Big_int::Size_mismatch ) {
    cerr << "Size mismatch" << endl;
  }
  return 0;
}
\end{lstlisting}
\caption{ Simple Use of {\code Big\_int} \label{fig:BigIntEx}}
\end{figure}

The loop in lines~52 to 57 adds the integers stored in the arrays together
by adding the least significant dword first, then the next least significant
dwords, \emph{etc.} The addition must be done in this sequence for extended
precision arithmetic (see Section~\ref{sec:ExtPrecArith}). Line~59 checks
for overflow, on overflow the carry flag will be set by the last addition
of the most significant dword. Since the dwords in the array are stored in
little endian order, the loop starts at the beginning of the array and
moves forward toward the end.

Figure~\ref{fig:BigIntEx} shows a short example using the {\code Big\_int}
class. Note that {\code Big\_int} constants must be declared explicitly as
on line~16. This is necessary for two reasons. First, there is no conversion
constructor that will convert an unsigned int to a {\code Big\_int}. Secondly,
only {\code Big\_int}'s of the same size can be added. This makes conversion
problematic since it would be difficult to know what size to convert to. A
more sophisticated implementation of the class would allow any size to be
added to any other size. The author did not wish to over complicate this
example by implementing this here. (However, the reader is encouraged to
do this.)
\index{C++!Big\_int example|)}

\begin{figure}[tp]
\begin{lstlisting}[language=C++, frame=tlrb]{}
#include <cstddef>
#include <iostream>
using namespace std;

class A {
public:
  void __cdecl m() { cout << "A::m()" << endl; }
  int ad;
};

class B : public A {
public:
  void __cdecl m() { cout << "B::m()" << endl; }
  int bd;
};

void f( A * p )
{
  p->ad = 5;
  p->m();
}

int main()
{
  A a;
  B b;
  cout << "Size of a: " << sizeof(a)
       << " Offset of ad: " << offsetof(A,ad) << endl;
  cout << "Size of b: " << sizeof(b)
       << " Offset of ad: " << offsetof(B,ad)
       << " Offset of bd: " << offsetof(B,bd) << endl;
  f(&a);
  f(&b);
  return 0;
}
\end{lstlisting}
\caption{ Simple Inheritance \label{fig:SimpInh}}
\end{figure}


\subsection{Inheritance and Polymorphism\index{C++!inheritance|(}}}


\begin{figure}[tp]
\begin{AsmCodeListing}
_f__FP1A:                       ; mangled function name
      push   ebp
      mov    ebp, esp
      mov    eax, [ebp+8]       ; eax points to object
      mov    dword [eax], 5     ; using offset 0 for ad
      mov    eax, [ebp+8]       ; passing address of object to A::m()
      push   eax
      call   _m__1A             ; mangled method name for A::m()
      add    esp, 4
      leave
      ret
\end{AsmCodeListing}
\caption{Assembly Code for Simple Inheritance \label{fig:FAsm1}}
\end{figure}

\emph{Inheritance} allows one class to inherit the data and methods of another.
For example, consider the code in Figure~\ref{fig:SimpInh}. It shows two 
classes, {\code A} and {\code B}, where class {\code B} inherits from {\code A}.
The output of the program is:
\begin{verbatim}
Size of a: 4 Offset of ad: 0
Size of b: 8 Offset of ad: 0 Offset of bd: 4
A::m()
A::m()
\end{verbatim}
Notice that the {\code ad} data members of both classes ({\code B}
inherits it from {\code A}) are at the same offset. This is important
since the {\code f} function may be passed a pointer to either an
{\code A} object or any object of a type derived (\emph{i.e.}
inherited from) {\code A}.  Figure~\ref{fig:FAsm1} shows the (edited)
asm code for the function (generated by \emph{gcc}).

\begin{figure}[tp]
\begin{lstlisting}[language=C++, frame=tlrb]{}
class A {
public:
  virtual void __cdecl m() { cout << "A::m()" << endl; }
  int ad;
};

class B : public A {
public:
  virtual void __cdecl m() { cout << "B::m()" << endl; }
  int bd;
};
\end{lstlisting}
\caption{ Polymorphic Inheritance \label{fig:VirtInh}}
\end{figure}

\index{C++!polymorphism|(}
Note that in the output that {\code A}'s {\code m} method was called
for both the {\code a} and {\code b} objects. From the assembly, one
can see that the call to {\code A::m()} is hard-coded into the
function. For true object-oriented programming, the method called
should depend on what type of object is passed to the function. This
is known as \emph{polymorphism}. C++ turns this feature off by
default. One uses the \emph{virtual} \index{C++!virtual} keyword to enable
it. Figure~\ref{fig:VirtInh} shows how the two classes would be
changed. None of the other code needs to be changed.  Polymorphism can
be implemented many ways. Unfortunately, \emph{gcc}'s implementation
is in transition at the time of this writing and is becoming
significantly more complicated than its initial implementation.  In
the interest of simplifying this discussion, the author will only
cover the implementation of polymorphism which the Windows based
Microsoft and Borland compilers use. This implementation has not
changed in many years and probably will not change in the foreseeable
future. 

With these changes, the output of the program changes:
\begin{verbatim}
Size of a: 8 Offset of ad: 4
Size of b: 12 Offset of ad: 4 Offset of bd: 8
A::m()
B::m()
\end{verbatim}


\begin{figure}[tp]
\begin{AsmCodeListing}[commentchar=!]
?f@@YAXPAVA@@@Z:
      push   ebp
      mov    ebp, esp

      mov    eax, [ebp+8]
      mov    dword [eax+4], 5  ; p->ad = 5;

      mov    ecx, [ebp + 8]    ; ecx = p
      mov    edx, [ecx]        ; edx = pointer to vtable
      mov    eax, [ebp + 8]    ; eax = p
      push   eax               ; push "this" pointer
      call   dword [edx]       ; call first function in vtable
      add    esp, 4            ; clean up stack

      pop    ebp
      ret
\end{AsmCodeListing}
\caption{Assembly Code for {\code f()} Function \label{fig:FAsm2}}
\end{figure}

\begin{figure}[tp]
\begin{lstlisting}[language=C++, frame=tlrb]{}
class A {
public:
  virtual void __cdecl m1() { cout << "A::m1()" << endl; }
  virtual void __cdecl m2() { cout << "A::m2()" << endl; }
  int ad;
};

class B : public A {    // B inherits A's m2()
public:
  virtual void __cdecl m1() { cout << "B::m1()" << endl; }
  int bd;
};
/* prints the vtable of given object */
void print_vtable( A * pa )
{
  // p sees pa as an array of dwords
  unsigned * p = reinterpret_cast<unsigned *>(pa);
  // vt sees vtable as an array of pointers
  void ** vt = reinterpret_cast<void **>(p[0]);
  cout << hex << "vtable address = " << vt << endl;
  for( int i=0; i < 2; i++ )
    cout << "dword " << i << ": " << vt[i] << endl;

  // call virtual functions in EXTREMELY non-portable way!
  void (*m1func_pointer)(A *);   // function pointer variable
  m1func_pointer = reinterpret_cast<void (*)(A*)>(vt[0]);
  m1func_pointer(pa);            // call method m1 via function pointer

  void (*m2func_pointer)(A *);   // function pointer variable
  m2func_pointer = reinterpret_cast<void (*)(A*)>(vt[1]);
  m2func_pointer(pa);            // call method m2 via function pointer
}

int main()
{
  A a;   B b1;  B b2;
  cout << "a: " << endl;   print_vtable(&a);
  cout << "b1: " << endl;  print_vtable(&b1);
  cout << "b2: " << endl;  print_vtable(&b2);
  return 0;
}
\end{lstlisting}
\caption{ More complicated example \label{fig:2mEx}}
\end{figure}


\begin{figure}[tp]
\centering
%\epsfig{file=vtable}
\input{vtable.latex}
\caption{Internal representation of {\code b1}\label{fig:vtable}}
\end{figure}

Now the second call to {\code f} calls the {\code B::m()} method because it
is passed a {\code B} object. This is not the only change however. The size
of an {\code A} is now 8 (and {\code B} is 12). Also, the offset of {\code
ad} is 4, not 0. What is at offset 0? The answer to these questions are related
to how polymorphism is implemented. 

\index{C++!vtable|(} A C++ class that has any virtual methods is given
an extra hidden field that is a pointer to an array of method
pointers\footnote{For classes without virtual methods C++ compilers
always make the class compatible with a normal C struct with the same
data members.}. This table is often called the \emph{vtable}. For the
{\code A} and {\code B} classes this pointer is stored at offset 0.
The Windows compilers always put this pointer at the beginning of the
class at the top of the inheritance tree. Looking at the assembly code
(Figure~\ref{fig:FAsm2}) generated for function {\code f} (from
Figure~\ref{fig:SimpInh}) for the virtual method version of the
program, one can see that the call to method {\code m} is not to a
label. Line~9 finds the address of the vtable from the object. The
address of the object is pushed on the stack in line~11. Line~12 calls
the virtual method by branching to the first address in the
vtable\footnote{Of course, this value is already in the {\code ECX}
register. It was put there in line~8 and line~10 could be removed and
the next line changed to push {\code ECX}. The code is not very
efficient because it was generated without compiler optimizations
turned on.}. This call does not use a label, it branches to the code
address pointed to by {\code EDX}. This type of call is an example of
\emph{late binding}\index{C++!late binding}. Late binding delays the
decision of which method to call until the code is running. This
allows the code to call the appropriate method for the object. The
normal case (Figure~\ref{fig:FAsm1}) hard-codes a call to a certain
method and is called \emph{early binding}\index{C++!early binding}
(since here the method is bound early, at compile time).

The attentive reader will be wondering why the class methods in
Figure~\ref{fig:VirtInh} are explicitly declared to use the C calling
convention by using the {\code \_\_cdecl} keyword. By default, Microsoft
uses a different calling convention for C++ class methods than the
standard C convention. It passes the pointer to the object acted on by
the method in the {\code ECX} register instead of using the stack. The
stack is still used for the other explicit parameters of the
method. The {\code \_\_cdecl} modifier tells it to use the standard C
calling convention. Borland~C++ uses the C calling convention by default.

\begin{figure}[tp]
\fbox{ \parbox{\textwidth}{\code
a: \\
vtable address = 004120E8\\
dword 0: 00401320\\
dword 1: 00401350\\
A::m1()\\
A::m2()\\
b1:\\
vtable address = 004120F0\\
dword 0: 004013A0\\
dword 1: 00401350\\
B::m1()\\
A::m2()\\
b2:\\
vtable address = 004120F0\\
dword 0: 004013A0\\
dword 1: 00401350\\
B::m1()\\
A::m2()\\
} }
\caption{Output of program in Figure~\ref{fig:2mEx} \label{fig:2mExOut}}
\end{figure}


Next let's look at a slightly more complicated example
(Figure~\ref{fig:2mEx}). In it, the classes {\code A} and {\code B}
each have two methods: {\code m1} and {\code m2}. Remember that since
class {\code B} does not define its own {\code m2} method, it
inherits the {\code A} class's method.  Figure~\ref{fig:vtable} shows
how the {\code b} object appears in memory. Figure~\ref{fig:2mExOut}
shows the output of the program. First, look at the address of the
vtable for each object.  The two {\code B} objects's addresses are the
same and thus, they share the same vtable.  A vtable is a property of
the class not an object (like a {\code static} data member). Next,
look at the addresses in the vtables. From looking at assembly output,
one can determine that the {\code m1} method pointer is at offset~0
(or dword~0) and {\code m2} is at offset~4 (dword~1). The {\code m2}
method pointers are the same for the {\code A} and {\code B} class
vtables because class {\code B} inherits the {\code m2} method from
the {\code A} class.

Lines~25 to 32 show how one could call a virtual function by reading
its address out of the vtable for the object\footnote{Remember this
code only works with the MS and Borland compilers, not \emph{gcc}.}.
The method address is stored into a C-type function pointer with an
explicit \emph{this} pointer.  From the output in
Figure~\ref{fig:2mExOut}, one can see that it does work. However,
please do \emph{not} write code like this! This is only used to
illustrate how the virtual methods use the vtable.

%Looking at the output of Figure~\ref{fig:2mExOut} does demonstrate several
%features of the implementation of polymorphism.  The {\code b1} and {\code b2}
%variables have the same vtable address; however the {\code a} variable
%has a different vtable address. The vtable is a property of the class not
%a variable of the class. All class variables share a common vtable. The two
%{\code dword} values in the table are the pointers to the virtual methods.
%The first one (number 0) is for {\code m1}. Note that it is different for the
%{\code A} and {\code B} classes. This makes sense since the A and B classes
%have different {\code m1} methods. However, the second method pointer is 
%the same for both classes, since class {\code B} inherits the {\code m2}
%method from its base class, {\code A}.

There are some practical lessons to learn from this. One important
fact is that one would have to be very careful when reading and writing
class variables to a binary file. One can not just use a binary read
or write on the entire object as this would read or write out the
vtable pointer to the file!  This is a pointer to where the vtable
resides in the program's memory and will vary from program to
program. This same problem can occur in C with structs, but in C,
structs only have pointers in them if the programmer explicitly puts
them in. There are no obvious pointers defined in either the {\code A}
or {\code B} classes.


Again, it is important to realize that different compilers implement
virtual methods differently. In Windows, COM (Component Object Model)
\index{COM} class objects use vtables to implement COM
interfaces\footnote{COM classes also use the {\code \_\_stdcall}
\index{calling convention!stdcall} calling convention, not the
standard C one.}. Only compilers that implement virtual method vtables
as Microsoft does can create COM classes. This is why Borland uses the
same implementation as Microsoft and one of the reasons why \emph{gcc}
can not be used to create COM classes.

The code for the virtual method looks exactly like a non-virtual one. Only
the code that calls it is different. If the compiler can be absolutely sure
of what virtual method will be called, it can ignore the vtable and call
the method directly (\emph{e.g.}, use early binding).
\index{C++!vtable|)}
\index{C++!polymorphism|)}
\index{C++!inheritance|)}
\index{C++!classes|)}
\index{C++|)}

\subsection{Other C++ features}

The workings of other C++ features (\emph{e.g.}, RunTime Type
Information, exception handling and multiple inheritance) are beyond
the scope of this text. If the reader wishes to go further, a good
starting point is \emph{The Annotated C++ Reference Manual} by Ellis
and Stroustrup and \emph{The Design and Evolution of C++} by
Stroustrup.


%%-*- latex -*-
\chapter{Dynamic Link Libraries}

\section{Using the Window's API and Dynamic Link Libraries}

UNIX systems provide a simple C based Application Programming
Interface (API).  In contrast, Microsoft Windows\texttrademark
\hspace{0.5em} packages its API in Dynamic Link Libraries that load in
each executables address space before its use. From the C/C++
programmers perspective, it appears as a normal C function call;
however, at the assembly level, it is different.

\subsection{Standard Call Calling Convention}
The Windows API uses the \emph{Standard Call}\index{calling
convention!standard call} calling convention. As stated eariler, this
convention pushes the arguments in reverse order just as the standard
C calling convention. However, there are two important
differences. First, the subroutine is responsible for clearing the
parameters from the stack. Secondly, the label of the function is
generated differently. An underscore is prepended the name as before,
but in addition the \emph{@} character is added to the end of the
function name along a number equal to the number of bytes on the stack
for the parameters of the function (in decimal).

\index{CloseHandle|(}
For example, consider the {\code CloseHandle} Windows API function. It's prototype
looks like:
\begin{lstlisting}[stepnumber=0]{}
BOOL WINAPI CloseHandle( HANDLE hObject );
\end{lstlisting}
Since a {\code HANDLE} is a double word in 32-bit Windows, the label
for this function would be {\code \_CloseHandle@4}. The {\code WINAPI}\index{WINAPI}
in the prototype is a C macro defined to be {\code \_\_stdcall}. Below
is a sample call to the function assuming that the {\code hObject}
value is in {\code EBX}.
\begin{AsmCodeListing}[frame=single]
  push  ebx             ; Push hObject on stack
  call  _CloseHandle@4
  mov   esi, eax        ; Save return value in ESI
\end{AsmCodeListing}
The stack does not have to adjusted after the function call, {\code CloseHandle}
fixes the stack automatically.
\index{CloseHandle|)}

\subsection{Static and Shared Libraries}

\index{static library|(}
A \emph{static library} is a collection of object files that can be
linked to an executable when it is created. The object code is
inserted directly into the executable just like an ordinary object
file. The library file is just a convenience. It allows a single file
to be included in the link step of the build process. All the object
code is probably not stored in the final executable. The linker will
look at which object modules are needed and only include the required
ones. Static libraries are created using the {\code LIB} program under
Windows or the {\code ar} program under UNIX. Windows libraries end in
a {\code .lib} extension and UNIX libraries end in a {\code .a}
extension.
\index{static library|)}

\index{shared library|(}
A shared library (or DLL in Windows) as its name implies shares code
among executables. When the executable is run, the OS finds all the 
shared libraries it requires and loads them into the processes memory
so the executable can use the code in them. Using the virtual memory
mapping capabilities on a protected mode operating system, shared
library code used by two or more concurrently running processes is
only loaded into physical memory once. 

There are advantages and disadvantages to shared libraries. The first
advantage is that executable sizes can be greatly reduced. If a large
library is used by many executables. Only one copy of the code (in the
shared library) is on the system (and in physical memory). If the
library was included statically, each executable would include a copy
of the code.

Another advantage is that if a bug is found in the shared library,
then the library can be replaced with one with the bug fixed. The
executables will automatically use the new fixed code without
recompiling the executable. However, this only works if the interface
of the functions in the shared library remain unchanged.

Shared libraries are also used to allow code written in different
languages to interoperate. For example, C++ can be interfaced to
Visual Basic\index{Visual Basic}, DotNet\index{DotNet} and
Java\index{Java} using shared libraries under Windows.

The disadvantages of shared libraries are that they can be more complicated
to maintain. One of the most common problems is that a new version of the
library breaks older code because it behaves slightly differently. Then
some executables need one version and others need a different one. In Windows,
this condition is known as \emph{DLL hell}\index{DLL hell}.
\index{shared library|)}

\index{DLL|(}
\subsection{Windows DLLs}

A Windows DLL is constructed much like an executable program. Object files are linked
together to create a DLL file. Unlike an executable, a DLL can have many entry points.
An entry point is just a function that can be called externally to the DLL. Only
functions that have been \emph{exported}\index{export} can be called externally.

A function (or global variable) may be exported by either entering its
name in the definition file\index{DLL!definition file} for the DLL or
by using Microsoft specific keywords.

\begin{figure}[t]
\begin{Verbatim}[frame=single,commandchars=\\\{\}]
LIBRARY \textit{library root name}
DESCRIPTION '\textit{short text description}'
EXPORTS
\textit{list of exported functions (one per line)}
\end{Verbatim}
\caption{Windows DLL definition file\label{fig:DefFile}}
\end{figure}

\subsubsection{Definition file}
This is a text file with a {\code .def} extension that lists all the
functions that the DLL exports. This file is used during the link step
of the DLL creation process. Figure~\ref{fig:DefFile} shows the
general format of a definition file.


\index{DLL|)}

\begin{appendix}
%appendix
\chapter{80x86 Instructions}
\section{Non-floating Point Instructions}
This section lists and describes the actions and formats of the 
non-floating point instructions of the Intel 80x86 CPU family.

The formats use the following abbreviations:
\begin{center}
\begin{tabular}{|l|l|}
\hline
R   & general register \\
R8  & 8-bit register \\
R16 & 16-bit register \\
R32 & 32-bit register \\
SR  & segment register \\
M   & memory \\
M8  & byte \\
M16 & word \\
M32 & double word \\
I   & immediate value \\
\hline
\end{tabular}
\end{center}
These can be combined for the multiple operand instructions. For example,
the format \emph{R, R} means that the instruction takes two register operands.
Many of the two operand instructions allow the same operands. The abbreviation
\emph{O2} is used to represent these operands: \emph{R,R R,M R,I M,R M,I}. If
a 8-bit register or memory can be used for an operand, the abbreviation,
\emph{R/M8} is used.

The table also shows how various bits of the FLAGS register are affected by
each instruction. If the column is blank, the corresponding bit is not
affected at all. If the bit is always changed to a particular value, a 1 or
0 is shown in the column. If the bit is changed to a value that depends on
the operands of the instruction, a \emph{C} is placed in the column. Finally,
if the bit is modified in some undefined way a \emph{?} appears in the
column. Because the only instructions that change the direction flag are 
{\code CLD} and {\code STD}, it is not listed under the FLAGS columns.

\begin{longtable}{||l|p{1.5in}|p{0.75in}|c|c|c|c|c|c||}
\hline \hline
\multicolumn{1}{||c}{} & 
   \multicolumn{1}{c}{} &
   \multicolumn{1}{c}{} &
  \multicolumn{6}{c||}{\textbf{Flags}} \\ \cline{4-9}
\multicolumn{1}{||c}{\textbf{Name}} & 
   \multicolumn{1}{c}{\textbf{Description}} &
   \multicolumn{1}{c}{\textbf{Formats}} &
   \multicolumn{1}{c}{\textbf{O}} &
   \multicolumn{1}{c}{\textbf{S}} &
   \multicolumn{1}{c}{\textbf{Z}} &
   \multicolumn{1}{c}{\textbf{A}} &
   \multicolumn{1}{c}{\textbf{P}} &
   \multicolumn{1}{c||}{\textbf{C}} \\ \hline \endhead
\hline \hline \endfoot
%                                              O   S   Z   A   P   C
{\code ADC} & Add with Carry & O2            & C & C & C & C & C & C \\
{\code ADD} & Add Integers   & O2            & C & C & C & C & C & C \\
{\code AND} & Bitwise AND    & O2            & 0 & C & C & ? & C & 0 \\
{\code BSWAP} & Byte Swap    & R32           &   &   &   &   &   &  \\
{\code CALL} & Call Routine  & R M I         &   &   &   &   &   &   \\
{\code CBW} & Convert Byte to Word &         &   &   &   &   &   & \\
{\code CDQ} & Convert Dword to Qword &       &   &   &   &   &   & \\
{\code CLC} & Clear Carry &                  &   &   &   &   &   & 0 \\
{\code CLD} & Clear Direction Flag &         &   &   &   &   &   & \\
{\code CMC} & Complement Carry &             &   &   &   &   &   & C \\
{\code CMP} & Compare Integers & O2          & C & C & C & C & C & C \\
{\code CMPSB} & Compare Bytes &              & C & C & C & C & C & C \\
{\code CMPSW} & Compare Words &              & C & C & C & C & C & C \\
{\code CMPSD} & Compare Dwords &             & C & C & C & C & C & C \\
{\code CWD} & Convert Word to Dword into DX:AX & &   &   &   &   &   & \\
{\code CWDE} & Convert Word to Dword into EAX & &   &   &   &   &   & \\
{\code DEC} & Decrement Integer & R M        & C & C & C & C & C & \\
{\code DIV} & Unsigned Divide & R M          & ? & ? & ? & ? & ? & ? \\
{\code ENTER} & Make stack frame & I,0       &   &   &   &   &   & \\
{\code IDIV} & Signed Divide & R M           & ? & ? & ? & ? & ? & ? \\
{\code IMUL} & Signed Multiply & R M R16,R/M16 R32,R/M32 R16,I R32,I 
                                       {\small R16,R/M16,I R32,R/M32,I}
                                             & C & ? & ? & ? & ? & C \\
{\code INC} & Increment Integer & R M        & C & C & C & C & C & \\
{\code INT} & Generate Interrupt & I         &   &   &   &   &   & \\
{\code JA } & Jump Above & I                 &   &   &   &   &   & \\
{\code JAE } & Jump Above or Equal & I       &   &   &   &   &   & \\
{\code JB } & Jump Below & I                 &   &   &   &   &   & \\
{\code JBE } & Jump Below or Equal  & I      &   &   &   &   &   & \\
{\code JC } & Jump Carry & I                 &   &   &   &   &   & \\
{\code JCXZ } & Jump if CX = 0 & I           &   &   &   &   &   & \\
{\code JE } & Jump Equal & I                 &   &   &   &   &   & \\
{\code JG } & Jump Greater & I               &   &   &   &   &   & \\
{\code JGE } & Jump Greater or Equal & I     &   &   &   &   &   & \\
{\code JL } & Jump Less & I                  &   &   &   &   &   & \\
{\code JLE } & Jump Less or Equal & I        &   &   &   &   &   & \\
{\code JMP } & Unconditional Jump & R M I    &   &   &   &   &   & \\
{\code JNA } & Jump Not Above & I            &   &   &   &   &   & \\
{\code JNAE } & Jump Not Above or Equal& I   &   &   &   &   &   & \\
{\code JNB } & Jump Not Below & I            &   &   &   &   &   & \\
{\code JNBE } & Jump Not Below or Equal & I  &   &   &   &   &   & \\
{\code JNC } & Jump No Carry & I             &   &   &   &   &   & \\
{\code JNE } & Jump Not Equal & I            &   &   &   &   &   & \\
{\code JNG } & Jump Not Greater & I          &   &   &   &   &   & \\
{\code JNGE } & Jump Not Greater or Equal & I&   &   &   &   &   & \\
{\code JNL } & Jump Not Less & I             &   &   &   &   &   & \\
{\code JNLE } & Jump Not Less or Equal & I   &   &   &   &   &   & \\
{\code JNO } & Jump No Overflow & I          &   &   &   &   &   & \\
{\code JNS } & Jump No Sign & I              &   &   &   &   &   & \\
{\code JNZ } & Jump Not Zero & I             &   &   &   &   &   & \\
{\code JO } & Jump Overflow & I              &   &   &   &   &   & \\
{\code JPE } & Jump Parity Even & I          &   &   &   &   &   & \\
{\code JPO } & Jump Parity Odd & I           &   &   &   &   &   & \\
{\code JS } & Jump Sign & I                  &   &   &   &   &   & \\
{\code JZ } & Jump Zero & I                  &   &   &   &   &   & \\
{\code LAHF} & Load FLAGS into AH &          &   &   &   &   &   & \\
{\code LEA} & Load Effective Address & R32,M &   &   &   &   &   & \\
{\code LEAVE} & Leave Stack Frame &          &   &   &   &   &   & \\
{\code LODSB} & Load Byte &                  &   &   &   &   &   & \\
{\code LODSW} & Load Word &                  &   &   &   &   &   & \\
{\code LODSD} & Load Dword &                 &   &   &   &   &   & \\
{\code LOOP}  & Loop       & I               &   &   &   &   &   & \\
{\code LOOPE/LOOPZ} & Loop If Equal & I     &   &   &   &   &   & \\
{\code LOOPNE/LOOPNZ} & Loop If Not Equal & I  &   &   &   &   &   & \\
{\code MOV} & Move Data & O2 \mbox{SR,R/M16} R/M16,SR
                                             &   &   &   &   &   & \\
{\code MOVSB} & Move Byte &                  &   &   &   &   &   & \\
{\code MOVSW} & Move Word &                  &   &   &   &   &   & \\
{\code MOVSD} & Move Dword &                 &   &   &   &   &   & \\
{\code MOVSX} & Move Signed & R16,R/M8 R32,R/M8 R32,R/M16
                                             &   &   &   &   &   & \\
{\code MOVZX} & Move Unsigned & R16,R/M8 R32,R/M8 R32,R/M16
                                             &   &   &   &   &   & \\
{\code MUL} & Unsigned Multiply & R M        & C & ? & ? & ? & ? & C \\
{\code NEG} & Negate & R M                   & C & C & C & C & C & C \\
{\code NOP} & No Operation &                 &   &   &   &   &   & \\
{\code NOT} & 1's Complement & R M           &   &   &   &   &   & \\
{\code OR} & Bitwise OR    & O2              & 0 & C & C & ? & C & 0 \\
{\code POP} & Pop From Stack & R/M16 R/M32   &   &   &   &   &   & \\
{\code POPA} & Pop All &                     &   &   &   &   &   & \\
{\code POPF} & Pop FLAGS &                   & C & C & C & C & C & C \\
{\code PUSH} & Push to Stack & R/M16 R/M32 I &   &   &   &   &   & \\
{\code PUSHA} & Push All &                   &   &   &   &   &   & \\
{\code PUSHF} & Push FLAGS &                 &   &   &   &   &   & \\
{\code RCL} & Rotate Left with Carry & R/M,I R/M,CL
                                             & C &   &   &   &   & C \\
{\code RCR} & Rotate Right with Carry & R/M,I R/M,CL
                                             & C &   &   &   &   & C \\
{\code REP} & Repeat &                       &   &   &   &   &   & \\
{\code REPE/REPZ} & Repeat If Equal&        &   &   &   &   &   & \\
{\code REPNE/REPNZ} & Repeat If Not Equal&  &   &   &   &   &   & \\
{\code RET} & Return &                       &   &   &   &   &   & \\
{\code ROL} & Rotate Left & R/M,I R/M,CL     & C &   &   &   &   & C \\
{\code ROR} & Rotate Right & R/M,I R/M,CL    & C &   &   &   &   & C \\
{\code SAHF} & Copies AH into FLAGS &        &   & C & C & C & C & C \\
{\code SAL} & Shifts to Left & R/M,I R/M, CL &   &   &   &   &   & C \\
{\code SBB}  & Subtract with Borrow & O2     & C & C & C & C & C & C \\
{\code SCASB} & Scan for Byte &              & C & C & C & C & C & C \\
{\code SCASW} & Scan for Word &              & C & C & C & C & C & C \\
{\code SCASD} & Scan for Dword &             & C & C & C & C & C & C \\
{\code SETA } & Set Above & R/M8                 &   &   &   &   &   & \\
{\code SETAE } & Set Above or Equal & R/M8       &   &   &   &   &   & \\
{\code SETB } & Set Below & R/M8                 &   &   &   &   &   & \\
{\code SETBE } & Set Below or Equal  & R/M8      &   &   &   &   &   & \\
{\code SETC } & Set Carry & R/M8                 &   &   &   &   &   & \\
{\code SETE } & Set Equal & R/M8                 &   &   &   &   &   & \\
{\code SETG } & Set Greater & R/M8               &   &   &   &   &   & \\
{\code SETGE } & Set Greater or Equal & R/M8     &   &   &   &   &   & \\
{\code SETL } & Set Less & R/M8                  &   &   &   &   &   & \\
{\code SETLE } & Set Less or Equal & R/M8        &   &   &   &   &   & \\
{\code SETNA } & Set Not Above & R/M8            &   &   &   &   &   & \\
{\code SETNAE } & Set Not Above or Equal& R/M8   &   &   &   &   &   & \\
{\code SETNB } & Set Not Below & R/M8            &   &   &   &   &   & \\
{\code SETNBE } & Set Not Below or Equal & R/M8  &   &   &   &   &   & \\
{\code SETNC } & Set No Carry & R/M8             &   &   &   &   &   & \\
{\code SETNE } & Set Not Equal & R/M8            &   &   &   &   &   & \\
{\code SETNG } & Set Not Greater & R/M8          &   &   &   &   &   & \\
{\code SETNGE } & Set Not Greater or Equal & R/M8&   &   &   &   &   & \\
{\code SETNL } & Set Not Less & R/M8             &   &   &   &   &   & \\
{\code SETNLE } & Set Not LEss or Equal & R/M8   &   &   &   &   &   & \\
{\code SETNO } & Set No Overflow & R/M8          &   &   &   &   &   & \\
{\code SETNS } & Set No Sign & R/M8              &   &   &   &   &   & \\
{\code SETNZ } & Set Not Zero & R/M8             &   &   &   &   &   & \\
{\code SETO } & Set Overflow & R/M8              &   &   &   &   &   & \\
{\code SETPE } & Set Parity Even & R/M8          &   &   &   &   &   & \\
{\code SETPO } & Set Parity Odd & R/M8           &   &   &   &   &   & \\
{\code SETS } & Set Sign & R/M8                  &   &   &   &   &   & \\
{\code SETZ } & Set Zero & R/M8                  &   &   &   &   &   & \\

{\code SAR} & Arithmetic Shift to Right & R/M,I R/M, CL 
                                             &   &   &   &   &   & C \\
{\code SHR} & Logical Shift to Right & R/M,I R/M, CL 
                                             &   &   &   &   &   & C \\
{\code SHL} & Logical Shift to Left & R/M,I R/M, CL 
                                             &   &   &   &   &   & C \\
{\code STC} & Set Carry &                    &   &   &   &   &   & 1 \\
{\code STD} & Set Direction Flag &           &   &   &   &   &   & \\
{\code STOSB} & Store Btye &                 &   &   &   &   &   & \\
{\code STOSW} & Store Word &                 &   &   &   &   &   & \\
{\code STOSD} & Store Dword &                &   &   &   &   &   & \\
{\code SUB} & Subtract & O2                  & C & C & C & C & C & C\\
{\code TEST} & Logical Compare & R/M,R R/M,I & 0 & C & C & ? & C & 0\\
{\code XCHG} & Exchange & R/M,R R,R/M        &   &   &   &   &   & \\
{\code XOR} & Bitwise XOR    & O2            & 0 & C & C & ? & C & 0 \\

\end{longtable}

\newpage
\section{Floating Point Instructions}

\renewcommand{\thefootnote}{\fnsymbol{footnote}} In this section, many
of the 80x86 math coprocessor instructions are described. The
description section briefly describes the operation of the
instruction. To save space, information about whether the instruction
pops the stack is not given in the description. 

The format column shows what type of operands can be used with each
instruction. The following abbreviations are used:
\begin{center}
\begin{tabular}{|l|l|}
\hline
ST\emph{n} & A coprocessor register \\
F          & Single precision number in memory \\
D          & Double precision number in memory \\
E          & Extended precision number in memory \\
I16        & Integer word in memory \\
I32        & Integer double word in memory \\
I64        & Integer quad word in memory \\
\hline
\end{tabular}
\end{center}

Instructions requiring a Pentium Pro or better are marked with an 
asterisk(\footnotemark[1]).

\begin{longtable}{||l|l|l||}
\hline \hline
\multicolumn{1}{||c}{\textbf{Instruction}} & 
  \multicolumn{1}{c}{\textbf{Description}} &
\multicolumn{1}{c||}{\textbf{Format}} \\
\hline
\endhead
\hline \hline \endfoot
{\code FABS} & $\mathtt{ST0} = |\mathtt{ST0}|$ & \\
{\code FADD \emph{src}} & {\code ST0 += \emph{src}} & ST\emph{n} F D \\
{\code FADD \emph{dest}, ST0} & {\code \emph{dest} += STO} & ST\emph{n} \\
{\code FADDP \emph{dest}[,ST0]} & {\code \emph{dest} += ST0} & ST\emph{n} \\
{\code FCHS} & $\mathtt{ST0} = - \mathtt{ST0}$ & \\
{\code FCOM \emph{src}} & Compare {\code ST0} and {\code \emph{src}} &
ST\emph{n} F D \\
{\code FCOMP \emph{src}} & Compare {\code ST0} and {\code \emph{src}} &
ST\emph{n} F D \\
{\code FCOMPP \emph{src}} & Compares {\code ST0} and {\code ST1} & \\
{\code FCOMI\footnotemark[1] \emph{src}} & Compares into FLAGS 
& ST\emph{n} \\
{\code FCOMIP\footnotemark[1] \emph{src}} & Compares into FLAGS 
& ST\emph{n} \\
{\code FDIV \emph{src}} & {\code ST0 /= \emph{src}} & ST\emph{n} F D \\
{\code FDIV \emph{dest}, ST0} & {\code \emph{dest} /= STO} & ST\emph{n} \\
{\code FDIVP \emph{dest}[,ST0]} & {\code \emph{dest} /= ST0} & ST\emph{n} \\
{\code FDIVR \emph{src}} & {\code ST0 = \emph{src}/ST0} & ST\emph{n} F D \\
{\code FDIVR \emph{dest}, ST0} & {\code \emph{dest} = ST0/\emph{dest}} 
& ST\emph{n} \\
{\code FDIVRP \emph{dest}[,ST0]} & {\code \emph{dest} = ST0/\emph{dest}} 
& ST\emph{n} \\
{\code FFREE \emph{dest}} & Marks as empty & ST\emph{n} \\
{\code FIADD \emph{src}} & {\code ST0 += \emph{src}} & I16 I32 \\
{\code FICOM \emph{src}} & Compare {\code ST0} and {\code \emph{src}} &
I16 I32 \\
{\code FICOMP \emph{src}} & Compare {\code ST0} and {\code \emph{src}} &
I16 I32 \\
{\code FIDIV \emph{src}} & {\code STO /= \emph{src}} & I16 I32 \\
{\code FIDIVR \emph{src}} & {\code STO = \emph{src}/ST0} & I16 I32 \\
{\code FILD \emph{src}} & Push \emph{src} on Stack & I16 I32 I64 \\
{\code FIMUL \emph{src}} & {\code ST0 *= \emph{src}} & I16 I32 \\
{\code FINIT} & Initialize Coprocessor & \\
{\code FIST \emph{dest}} & Store {\code ST0} & I16 I32 \\
{\code FISTP \emph{dest}} & Store {\code ST0} & I16 I32 I64\\
{\code FISUB \emph{src}} & {\code ST0 -= \emph{src}} & I16 I32 \\
{\code FISUBR \emph{src}} & {\code ST0 = \emph{src} - ST0} & I16 I32 \\
{\code FLD \emph{src}} & Push \emph{src} on Stack & ST\emph{n} F D E \\
{\code FLD1} & Push 1.0 on Stack & \\
{\code FLDCW \emph{src}} & Load Control Word Register & I16 \\
{\code FLDPI} & Push $\pi$ on Stack & \\
{\code FLDZ} & Push 0.0 on Stack & \\
{\code FMUL \emph{src}} & {\code ST0 *= \emph{src}} & ST\emph{n} F D \\
{\code FMUL \emph{dest}, ST0} & {\code \emph{dest} *= STO} & ST\emph{n} \\
{\code FMULP \emph{dest}[,ST0]} & {\code \emph{dest} *= ST0} & ST\emph{n} \\
{\code FRNDINT} & Round {\code ST0} & \\
{\code FSCALE} & $\mathtt{ST0} = \mathtt{ST0} \times 2^{\lfloor \mathtt{ST1} \rfloor}$ & \\
{\code FSQRT} & $\mathtt{ST0} = \sqrt{\mathtt{STO}}$ & \\
{\code FST \emph{dest}} & Store {\code ST0} & ST\emph{n} F D \\
{\code FSTP \emph{dest}} & Store {\code ST0} & ST\emph{n} F D E \\
{\code FSTCW \emph{dest}} & Store Control Word Register & I16 \\
{\code FSTSW \emph{dest}} & Store Status Word Register & I16 AX \\
{\code FSUB \emph{src}} & {\code ST0 -= \emph{src}} & ST\emph{n} F D \\
{\code FSUB \emph{dest}, ST0} & {\code \emph{dest} -= STO} & ST\emph{n} \\
{\code FSUBP \emph{dest}[,ST0]} & {\code \emph{dest} -= ST0} & ST\emph{n} \\
{\code FSUBR \emph{src}} & {\code ST0 = \emph{src}-ST0} & ST\emph{n} F D \\
{\code FSUBR \emph{dest}, ST0} & {\code \emph{dest} = ST0-\emph{dest}} 
& ST\emph{n} \\
{\code FSUBP \emph{dest}[,ST0]} & {\code \emph{dest} = ST0-\emph{dest}} 
& ST\emph{n} \\
{\code FTST} & Compare {\code ST0} with 0.0 & \\
{\code FXCH \emph{dest}} & Exchange {\code ST0} and {\code \emph{dest}} 
& ST\emph{n} \\
\end{longtable}

\renewcommand{\thefootnote}{\arabic{footnote}}



\end{appendix}
\clearpage
\ifmypdf
\phantomsection % fixes link anchor
\fi
\addcontentsline{toc}{chapter}{Index}
\printindex
\end{document}

